\subsection{Case study: auto async-signal safety}
\label{sec:assafe}

We have now described the role of \textit{libgotcha}, and how \textit{libinger} uses
it to handle nonreentrancy.  Before concluding our discussion, however, we note that
\textit{libgotcha} has other interesting uses in its own right.

As an example, we have used it to implement a small library, \textit{libas-safe},
that transparently allows an application's signal handlers to call functions that
are not async-signal safe, which is forbidden by POSIX because it is normally unsafe.

Written in 127 lines of C, \textit{libas-safe} works by injecting code before
\texttt{main()} to switch the program away from its starting libset.  It shadows
the system's \texttt{sigaction()}, providing an implementation that:
\begin{itemize}
\item Provides copy-based library isolation for signal handlers by switching the
	thread's next libset to the starting libset while a signal handler is running.
\item Allows use of uncopyable code such as \texttt{malloc()} from a signal
	handler by deferring signal arrival whenever the thread is already executing
	in the starting libset, then delivering the deferred signal when the
	interruptible callback fires.
\end{itemize}

In addition to making signal handlers a lot easier to write, \textit{libas-safe} can
be used to automatically ``fix'' deadlocks and other misbehaviors in misbehaved
signal-handling programs just by loading it via \texttt{LD\_PRELOAD}.

We can imagine extending \textit{libgotcha} to support other use cases, such as
simultaneously using different versions or build configurations of the same library
from a single application.

\thesis{Walk through example(s) of programs that are automatically repaired by
\textit{libas-safe}.}

\thesis{Explore \textit{libac-safe} idea?}

\thesis{Use \textit{libas-safe} to prototype a library that enables safe signal
handling in Rust.}

\thesis{It's been suggested that \texttt{libgotcha} could let you switch
between multiple versions of the same library (e.g., different revisions, feature
sets, or even release vs. debug).  This probably merits additional thought.}

\thesis{Another possible use for (an enhanced) \textit{libgotcha} is some kind
of scary runtime aspect-oriented programming thing (i.e., a generalization of
ltrace).}
