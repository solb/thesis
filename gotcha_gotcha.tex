\section{Shared state: \textit{libgotcha}}
\label{sec:libgotcha}

Despite the name, it is more like a runtime that isolates hidden shared state
within an application.  Although the rest of the
program does not interact directly with \textit{libgotcha}, its presence has a
global effect:\@ once loaded into the process image, it employs a technique we call
\textbf{selective relinking} to dynamically intercept and reroute many of the
program's function
calls and global variable accesses.

The goal of \textit{libgotcha} is to establish around every preemptible function a
memory isolation boundary encompassing whatever third-party libraries that function
interacts with (Section~\ref{sec:libinger:reentrancy}).  The result is that the only
state shared across the boundary is that explicitly passed via arguments,
return value, or closure---the same state the application programmer is responsible
for protecting from concurrency violations (Section~\ref{sec:libinger:concurrency}).
Listing~\ref{lst:exmplstate} shows the impact on an example program, and
Figure~\ref{fig:progsupport} classifies libraries by how \textit{libgotcha} supports
them.

\begin{figure}
\begin{lstlisting}[label=lst:exmplstate,caption=Demo of isolated \textnormal{(1)} vs.\ shared \textnormal{(2\&3)} state]
static bool two;
bool three;

linger_t caller(const char *s, u64 timeout) {
	stdout = NULL;
	two = true;
	three = true;
	return launch(timed, timeout, s);
}

void timed(void *s) {
	assert(stdout); // (1)
	assert(two); // (2)
	assert(three); // (3)
}
\end{lstlisting}
\end{figure}

Note that \textit{libgotcha} operates at runtime; this constrains its visibility
into the program, and therefore the granularity of its operation, to shared
libraries.  It therefore
assumes that the programmer will dynamically link all third-party libraries, since
otherwise there is no way to tell them apart from the rest of the program at runtime.
We feel this restriction is reasonable because a programmer wishing to use
\textit{libinger} or \textit{libgotcha} must already have control over their
project's build in order to add the dependency.

Before introducing the \textit{libgotcha} API and explaining selective relinking, we
now briefly motivate the need for \textit{libgotcha} by demonstrating how existing
system interfaces fail to provide the required type of isolation.
