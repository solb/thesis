\chapter{Preemptible remote procedure calls: \\ the \textit{strobelight} caching RPC server}
\label{chap:strobelight}

\ifdefined\chapquotes
\vspace{-1in}
\begin{chapquote}[1.5in]{Philip Reeve, \textit{Mortal Engines}}
`Do you really think I am so shortsighted?  The Guild \\
of Engineers plans further ahead than you suspect. \\
London will never stop moving.  Movement is life.'
\end{chapquote}
\fi

Whereas local function calls with timeouts have yet to become a mainstream idea, it
has long been standard to use timeouts when making RPCs (remote procedure calls)
where a program on one machine invokes a function that executes on another.  Because
the connection to the remote machine could fail, it is common for the caller to
specify a timeout after which the call should return an error code, allowing the
client program to continue running.  What these systems usually do not do is use this
timeout to reduce wasted work on the server side.  In their most classic formulation,
RPCs do not even inform the server of the timeout; if the client times out, the
server continues to work on the request, only to find once it has finished the work
that the client is no longer listening for an answer.  This is particularly wasteful
if an ill-behaved client repeats a failing request with the same timeout, in which
case the server duplicates the same work with the same likely fate.  Newer systems
might share the timeout with the server, but generally use it only for cooperative
scheduling.  Therefore, it is up to the developer of each server-side function to
implement cancellation to prevent it from wasting resources serving doomed requests.

We had a first-year undergraduate student prototype a novel caching RPC server that
addresses this limitation transparently to the server-side programmer.  Over the
course of two months, he built the \textit{strobelight} RPC server, so called because
it processes each request only while the client is listening for a response and
pauses in the intervening intervals to avoid wasting server compute time.  The system
serves as an example of the new capabilities possible by using preemptible functions,
and also a demonstration that preemptible functions are not only usable by expert
programmers.


\section{State-of-the-art RPC systems}

The motivation for this work is that contemporary RPC systems do not generally
support preemption.  We begin with a brief survey of well-known systems and their
approach to timeouts.

\paragraph{ONC RPC}
In the 1980s, Sun Microsystems developed a custom RPC protocol as part of the NFS
(Network File System) project.  The protocol has come to be known as ONC (Open
Network Computing) RPC, and was standardized in a series of RFCs, most recently
updated in 2009.  The protocol is widely implemented, shipping with many Unix systems
and Microsoft Windows.  It is minimal and leaves features such as timeouts to the
transport protocol.  The RFC notes that clients and servers must carefully handle the
case of retrying timed out or otherwise failed requests to avoid executing the task
multiple times~\cite{www-onc-rpc-rfc}.

\paragraph{ZooKeeper}
The Apache Software Foundation maintains a more modern distributed RPC system called
ZooKeeper, which grew out of the Hadoop project.  ZooKeeper runs RPCs on a cluster of
servers known as an ensemble.  Although the authors make use of the term timeout,
they mean something different by it than Sun:\@ in ZooKeeper, a timeout is a
heartbeat used to diagnose connection or server problems, and is associated with the
client session rather than any specific request.  Unrequited heartbeats cause the
session to expire, at which point requests that had not yet started computing return
an error code~\cite{Hunt:zookeeper:2010}.

\paragraph{eRPC}
The eRPC project achieves lower RPC latencies by providing low-level primitives
tailored to userland networking drivers.  Like ZooKeeper, it uses heartbeats to
detect broken connections.  In such a case, it immediately notifies the client of the
failure, but the server continues waiting for associated request(s) to finish
executing~\cite{Kalia:nsdi2019}.

\paragraph{gRPC}
In 2015, Google open sourced a new RPC system designed for use in modern datacenters.
The gRPC documentation encourages clients to include a timeout, which it refers to as
a deadline, with their requests.  This timeout is transmitted to the server along
with the request, where it can be used to cancel the associated request.
Unfortunately, each server-side function is responsible for detecting such
cancellations by periodically checking a flag and responding
accordingly~\cite{www-grpc}.

\paragraph{ZIO gRPC}
ZIO gRPC is an experimental RPC system based on gRPC.  It adds asynchronous
cancellation, which it accomplishes by requiring that server-side functions are
written in purely functional Scala.  Upon a timeout, the system transparently cancels
any partial work the server has done and cleans up its associated
resources~\cite{www-zio-grpc}.


\section{Language support for asynchronous cancellation}
\label{sec:strobelight:cancellation}

Today's RPC systems force the server-side programmer to handle timeouts and
cancellation manually.  The only exception is ZIO gRPC, which only supports purely
functional services.

This state of affairs becomes less surprising---albeit no less disappointing---if we
recall that operating systems have long failed to support asynchronous cancellation
(Chapter~\ref{chap:functions}).  Unsurprisingly, most programming languages that
support shared state have struggled with the same problem.

C\# used to provide the \texttt{Thread.Abort()} method for asynchronously cancelling
a thread.  It has since marked it obsolete because it suffers from the same safety
problems as POSIX and Windows threads.  As the documentation remarks:
\begin{quote}
The \texttt{Thread.Abort()} method should be used with caution.  Particularly when
you call it to abort a thread other than the current thread, you don't know what code
has executed or failed to execute when the \texttt{ThreadAbortException} is thrown.
You also cannot be certain of the state of your application or any application and
user state that it's responsible for preserving.  For example, calling
\texttt{Thread.Abort()} may prevent the execution of static constructors or the
release of unmanaged resources~\cite{www-csharp-abort}.
\end{quote}

The Java standard library initially offered a \texttt{Thread.stop()} method for
asynchronous cancellation, but the language was soon forced to deprecate the feature.
Notably, it purports to avoid the resource leak and deadlock problems of competing
systems, but can still corrupt program and library state:
\begin{quote}
Stopping a thread causes it to unlock all the monitors that it has locked.  (The
monitors are unlocked as the \texttt{ThreadDeath} exception propagates up the stack.)
If any of the objects previously protected by these monitors were in an inconsistent
state, other threads may now view these objects in an inconsistent state.  Such
objects are said to be damaged. When threads operate on damaged objects, arbitrary
behavior can result.  This behavior may be subtle and difficult to detect, or it may
be pronounced.  Unlike other unchecked exceptions, \texttt{ThreadDeath} kills threads
silently; thus, the user has no warning that his program may be corrupted.  The
corruption can manifest itself at any time after the actual damage occurs, even hours
or days in the future~\cite{www-java-stop}.
\end{quote}

Even scarier is Ruby, which still offers the \texttt{Thread.raise} interface and
higher-level ``features'' such as \texttt{Timeout}.  These are still marketed as
useful and bear no disclaimers, despite making no effort to address any of the perils
of asynchronous preemption~\cite{www-ruby-danger-terrifying}.


\section{Easier RPCs with \textit{strobelight}}

Our work on preemptible functions paves the way for systems that safely support both
preemption and asynchronous cancellation.  The new concurrency abstraction provides
what the Java commentary identifies as missing:\@ a way to determine that structures
may be in a damaged state.  Furthermore, it contains such damage to those objects
directly used by the cancelled preemptible function itself by isolating library
states via selective relinking.  (As we have noted, resource leaks remain a problem
that the programmer must solve manually, but we sketched an approach to automatic
cleanup in Chapter~\ref{chap:ingerc}.)

The \textit{strobelight} system leverages preemptible functions to improve the
usability of RPCs in two significant ways:\@ preventing wasted work and preserving
salvageable progress.  It does this by extending \textit{libturquoise}'s approach of
wrapping each task in a preemptible function (Chapter~\ref{chap:libturquoise}).  A
\textit{strobelight} client sends a request consisting of a function identifier, set
of arguments, and its timeout.  As each request arrives, the \textit{strobelight}
server invokes a preemptible function to process it.  If the client times out and
stops listening for a response, the server's preemptible function does so at a
similar time, automatically pausing execution.  The server keeps paused preemptible
functions around and ``memoizes'' repeated requests by resuming them when an incoming
function identifier and argument set match those of one that has timed out.  The
server also memoizes computed results, so that repeating a completed request does not
result in duplicate work.

The \textit{strobelight} server is a proof-of-concept system implemented in 128 lines
of Rust.  As such, it has several shortcomings that a production system would need to
address.  For one thing, it currently memoizes all partial and completed requests,
assuming that all server-side functions neither affect one another's results nor
depend on anything other than their arguments.  Lifting these impractical
restrictions would require a configuration or annotation mechanism to indicate which
functions have these properties; ideally, the system would also provide a system for
managing transactions that was aware of dependencies and able to determine when a
restart or recalculation was required.  Another shortcoming of the current system is
that it keeps all past requests around forever.  To avoid a ballooning resource
footprint, a real server would need to use an LRU cache or similar approach.  For
partially-computed functions, this would either require the programmer to write a
specific cleanup handler to accompany each server-side function or the integration of
our automatic cancellation ideas into the preemptible functions stack.  The current
implementation does not provide type safe bindings for the client, but systems such
as gRPC already solve this using Protocol Buffers.  Finally, \textit{strobelight}
does not currently allow install additional functions into a running server, but this
would be easy to support thanks to \textit{libgotcha}'s compatibility with the
dynamic linker's \texttt{dlopen()} interface for runtime loading
(Section~\ref{sec:libgotcha:interpose}).
