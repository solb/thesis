To demonstrate this feature, we consider
decompression bombs, files that expand exponentially when decoded, consuming enormous
computation time in addition to their large memory footprint.
PNG files are vulnerable to such an attack, and although \textit{libpng} now
supports some mitigations~\cite{www-libpng-bombs}, one cannot always expect (or
trust) such functionality from third-party code.

We benchmarked the use of \textit{libpng}'s ``simple API'' to decode an in-memory PNG
file.  We then compared against synchronous isolation using preemptible functions, as
well as the na\"ive alternative mitigations proposed in Section~\ref{sec:intro}.  For
preemptible functions, we wrapped all uses of \textit{libpng} in a call to
\texttt{launch()} and used a dedicated (but blocking) reaper thread to remove the
cost of cancellation from the critical path; for threads, we used
\texttt{pthread\_create()} followed by \texttt{pthread\_timedjoin\_np()} and,
conditionally, \texttt{pthread\_cancel()} and \texttt{pthread\_join()}; and for
processes, we used \texttt{fork()} followed by \texttt{sigtimedwait()}, a
conditional \texttt{kill()}, then a \texttt{waitpid()} to reap the child.  We ran
\texttt{pthread\_cancel()} both with and without asynchronous cancelability enabled,
but the former always deadlocked.  The timeout was 10 ms in all cases.

\begin{figure}
\begin{center}
	\begin{minipage}{0.235\textwidth}
	\includegraphics[width=\textwidth]{figs/cerberus2_nns16_surplus256k_mirjam}
	\subcaption{Benign image}
	\label{fig:libpng:benign}
	\end{minipage}
%
	\begin{minipage}{0.235\textwidth}
	\includegraphics[width=\textwidth]{figs/cerberus2_nns16_surplus256k_10K}
	\subcaption{Malicious image}
	\label{fig:libpng:bomb}
	\end{minipage}
\end{center}
\caption{\textit{libpng} in-memory image decode times}
\end{figure}

Running on the benign RGB image \texttt{mirjam\_meijer\_mirjam\_mei.png} from version
\texttt{1:0.18+dfsg-15} of Debian's \texttt{openclipart-png} package showed
\texttt{launch()} to be both faster and lower-variance than the other approaches,
adding 355 $\mu$s or 5.2\% over the baseline (Figure~\ref{fig:libpng:benign}).
