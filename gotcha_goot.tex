\subsection{Selective relinking}
\label{sec:relinking}

Most of the complexity of \textit{libgotcha} lies in the implementation of selective
relinking, the mechanism underlying libset switches.

Whenever a program uses a dynamic symbol, it looks up its address in a data structure
called the global offset table (GOT).  As it loads the program, \texttt{ld-linux.so}
eagerly resolves the addresses of all global variables and some functions and stores
them in the GOT.

Selective relinking works by shadowing the GOT.\footnote{Hence the name
\textit{lib\textbf{got}cha}.}  As soon as \texttt{ld-linux.so} finishes populating
the GOT, \textit{libgotcha} replaces every entry that should trigger a libset switch
with a fake address, storing the original one in its shadow GOT, which is organized
by the libset that houses the definition.  The fake address used depends upon the
type of symbol:

\thesis{Draw a diagram explaining our custom tables and the series of lookups
performed by \texttt{procedure\_linkage\_override()}.}

\thesis{Talk about PLOTs and pointer comparison.}
