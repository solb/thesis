\subsection{Library copying: namespaces}
\label{sec:namespaces}

Expanding a preemptible function's isolation boundary to include libraries requires
providing it with private copies of those libraries.  POSIX has long provided a
\texttt{dlopen()} interface to the dynamic linker for loading shared objects at
runtime; however, opening an already-loaded library just increments
a reference count, and this function is therefore of no use for making copies.
Fortunately, the GNU dynamic linker (\texttt{ld-linux.so}) also supports
Solaris-style \textbf{namespaces}, or isolated sets of loaded libraries.  For each
namespace, \texttt{ld-linux.so} maintains a separate set of loaded libraries whose
dependency graph and reference counts are tracked independently from the rest of the
program~\cite{dlmopen-manpage}.

It may seem like namespaces provide the isolation we need:\@ whenever we
\texttt{launch(\textnormal{\textit{F}})}, we can initialize a namespace with a copy
of the whole application and transfer control into that namespace's copy of
\textit{F}, rather than the original.  The problem with this approach is that it
breaks the lexical scoping of static variables.  For example,
Listing~\ref{lst:exmplstate} would fail assertion (2).

\thesis{Cover the algorithm for identifying GOT entries that correspond to
cross-library symbols.}
