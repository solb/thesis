\section{Providing Isolation}
\label{sec:isolation}

Of course, consolidating multiple users' jobs into a single process requires
addressing the accompanying security challenges (isolation---in the sense of
confidentiality/integrity as well as resource limits), and doing so in a way that
does not compromise our ambitious performance goals.

Our guiding philosophy for doing so is ``Language-based isolation with defense in depth.''
We draw inspiration from two recently-published systems whose own demanding
performance requirements drove them to perform similar coalescing of traditionally
independent components:  NetBricks~\cite{Panda2016} is a network functions runtime
for providing programmable network capabilities; it is unique among this class of
systems for running the submitted functions in-process rather than in VMs.
Tock~\cite{Levy2017} is an embedded operating system that provides (in addition to a
more traditional process model) a type of lightweight application known as a capsule
that is embedded within the kernel and communicates with it using simple function
calls.  As their primary line of defense against untrusted code, both systems
leverage Rust~\cite{www-rustlang}, a new type-safe systems programming language.

Rust is a strongly-typed, compiled language that reprises C's abstention from a
heavyweight runtime.  Unlike many other modern systems languages, Rust is an
attractive choice when both performance and predictability are critical due to its
lack of a garbage collector.  Still, it manages to provide strong memory safety
guarantees by focusing on ``zero-cost abstractions'' (i.e., those that can be
compiled down to code whose safety is assured without runtime checks).  In
particular, safe Rust code is guaranteed to be free of null or dangling pointer
dereferences, invalid variable values (e.g., casts are checked and unions are
tagged), reads from uninitialized memory, mutations of non-\texttt{mut} data (roughly
the equivalent of C's \texttt{const}), and data races, among other
misbehaviors~\cite{www-rustlang-ub}.

We require each microservice to be written in Rust, which gives us many aspects of
the isolation we need:  It is difficult for microservices to crash the worker process,
since most segmentation faults are prevented, and runtime errors such as integer
overflow generate Rust panics that we can catch.  Microservices cannot get references
to data that does not belong to them thanks to the variable and pointer initialization
rules.  
Given our performance goals, there is a significant aspect of the
required isolation that Rust does not provide: there is nothing to stop users from
monopolizing the CPU\@.
Our system, however, must be preemptive.\footnote{It is surprisingly hard to find lightweight
  threads that can be killed easily.  For example, 
Go~\cite{www-golang} and Erlang~\cite{www-erlang}'s lightweight threads
share a kernel thread among multiple tasks, but use cooperatative multitasking
between tasks on a thread;  there is no way to involuntarily terminate a task
from outside the task.}  We discuss our solution to this in the following
section.

Users thus submit their microservices in the form of Rust source code, allowing the
serverless operator to pass the \texttt{-Funsafe-code} flag while compiling to reject
any \texttt{unsafe} code.  This process does not need to happen on the compute
nodes, provided that the deployment server tasked with compilation runs the same version
of the Rust compiler.\footnote{This restriction exists because, as of the latest
release (1.23.0) of the compiler, Rust does not have a stable ABI.}  The operator
needs to trust the compiler, standard library, and any libraries against which it
will permit the microservice to link, but importantly need not worry about the
microservice itself.  We believe that most users would find it acceptable to be
presented with a list of permissible dependencies.\mk{Not sure if we should keep this sentence; it's probably reasonable,
but might be seen as a bit of conjecture.}  Libraries that do not use
\texttt{unsafe} code can be whitelisted without review.  To approximate how big
such a list would be given the current Rust ecosystem, we turn to a 2017
study~\cite{www-cratesio-unsafe} by the Tock authors that found just under half of
the Rust package manager's top 1000 most-downloaded libraries to be free of
\texttt{unsafe} code.  They caution that many of those packages have unsafe
dependencies, but we suspect that reviewing a relatively small number of popular
libraries would open up the majority of the most popular packages.

Once the microservice is compiled to a shared object file, it is distributed to
each compute node on which it might run.  Then, in order to ensure that invokers will
experience the warm-start latencies discussed in Section~\ref{sec:motive}, those
nodes' host processes should instruct one or more of their workers to preload the
dynamic library.  If the provider experiences too many active
microservices for its available resources, it can unload some libraries; on their
next invocation, they will experience invocation latency greater than the network
latency, but comparable to the \textit{warm-start} latencies of today's serverless
systems.
