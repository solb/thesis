\chapter{Acknowledgments}

I wish to thank my adviser, Dave Andersen, for his unwavering support and for never
giving up on me despite extended periods of little progress.  I especially appreciate
his encouragement to take vacations and backing of my teaching, even when it meant
taking time off from research.

A doctorate can be an emotional roller coaster ride, and I owe many people thanks for
keeping me from derailing on the tricky corners (and corner cases).  The good doctors
Ben Blum and Dominic Chen, both of whom were students in the program when I started,
inspired me with their intense systems projects.  Their perseverance in conducting
challenging low-level research convinced me that operating systems research is far
from ``dead,'' as some professors had tried to tell me when I was seeking an adviser.
On more than one occasion, Dominic rescued me from debugging funks when I spent weeks
unsuccessfully searching for horrible bugs; I am also forever indebted to him for
introducing me to the rr reverse debugger.  Ben's earlier work on Rust was one of the
factors that led me to learn the programming language.  Another colleague, Gabriele
Farina, challenged me to become fluent in it and encouraged me to join him at Rust
conferences and attend the Pittsburgh Rust coffee meetup.  At the latter, I met
Holden Marcsisin, then a star local high school student who helped me reason about
safe unstructured jumps in the context of Rust.  Dave's (now) former student Anuj
Kalia was instrumental in my research pivot toward the more fulfilling area of novel
programming primitives:\@ his suggestion during a meeting about our microservices
work that we could leverage dynamic linking to address the nonreentrancy problems
associated with asynchronous cancellation led directly to my work on selective
relinking, which made everything else in this \thesis possible.  And I owe a great
debt to my many friends in the program who kept things fun (and me sane) through the
hard times with regular board games and other social events, bicycle rides and trips,
and the occasional hike.

This research was also made possible by the labor of countless people whom I do not
know personally.  I would like to thank the contributors to Rust, glibc, GDB, rr,
Valgrind, strace, Git, and the many other free and/or open-source software projects
that make computers worth using.  I am constantly impressed by the quality of today's
libre development tooling, which greatly simplifies the implementation and debugging
of even low-level code.  While I have done little to contribute to the ecosystem of
tools, you have my appreciation.  Your work leaves me in awe of my predecessors who
conducted operating systems research before such tools were available.  I must also
thank the anonymous reviewer who wrote the following kind words about our eponymous
conference paper, words made especially inspiring by my own teaching aspirations:
\begin{quote}
It seems like a very old-school ATC style submission, which is great.  When I was
teaching undergraduates about high-concurrency userspace services and the tradeoffs
between coroutines and event-driven programming, this would have been a nice approach
to present.
\end{quote}

I became a Ph.D. student in order to pursue a career in teaching, and thanks to the
many people who placed their trust in me, I have been had numerous opportunities for
practice throughout my graduate studies.  Charlie Garrod served as a sort of teaching
mentor during the early half of my studies, something that every doctoral student who
longs to teach needs to find.  Bill Scherlis enabled me to take a semester-long leave
of absence to serve as a sabbatical replacement for one of the computer science
teachers at a local private high school, where I had (among other responsibilities)
the amazing experience of designing and teaching a seminar course on computer
architecture to a small group of very motivated students.  Brian Railing twice
invited me to serve as co-instructor of record for Carnegie Mellon's well-known
15-213 Introduction to Computer Systems course.  Tom Cortina asked me to teach a very
compressed six-week instance of the data structures course, 15-122 Principles of
Imperative Computation, during my last summer in the program.  Fellow doctoral
student Kyle Liang volunteered to co-instruct with me; both of us were new to the
course, and I could not have done it without him, our incredibly devoted staff of
nineteen undergraduate teaching assistants, or the regular guidance of Iliano
Cervesato.  Later that year, when the public high school in neighboring Mt.\@ Lebanon
temporarily lost its computer science teacher, Dave was willing to continue paying
for me even as I substitute taught there three days per week, just four months before
my defense and before most of this \thesis was written!  Special thanks go to Deb
Cavlovich, the administrator of our program, for petitioning the department head for
the exception to the rules that made this arrangement possible.  And my longtime
friend Connor Brem has my endless gratitude for subsequently \textit{using his
vacation time} to assume the role and teach the 150 high school students for the next
two months while I wrote and defended this document.

Numerous others have supported my growth as a teacher in other ways.  Michael Hilton
taught a course on computer science pedagogy.  Along with Fran\c{c}eska Xhakaj and
Brian, he went on to found a semiweekly discussion series that now brings together
teaching types from across the School of Computer Science.  Dave, Guy Blelloch, and
Tom funded my three successful---and one deeply unsuccessful, thanks to a global
pandemic---trips to the SIGCSE computer science education conference.  Erica Weng
organized a series of undergraduate research mixers that directly led to a different
sort of teaching experience for me, when then-freshman Yosef Alsuhaibani implemented
the \textit{strobelight} RPC system while serving as my summer research assistant.

Thank you to my entire family for always supporting my education.  I attribute both
my love of learning and my teaching aspirations to a tight-knit extended family that
cherishes education and encourages everyone to follow their dreams rather than money.
Thank you to my many aunts and uncles and older cousins for all the gatherings and
outdoor adventures, and to my younger cousins for tolerating it when I force fed you
computer science concepts from a young age.  I would not have pursued (or
successfully completed) a doctorate without my parents' encouragement, and let the
record show that my father was right to insist that Carnegie Mellon's offer of a
guaranteed stipend was a Big Deal.  Dad, that very provision did rescue me from at
least one funding shortfall, so you officially get to say, ``I told you so.''

Last, but certainly not least, my thanks go out to my partner Yvonne Marcoux, who
selflessly took over my executive function for the first two months of this year
while I locked myself in the apartment and wrote the majority of this \thesis.
Without you, it would have been a miserable existence indeed, and the quality of the
end product would be that much worse.  Thanks for all the bicycling, camping, and
road trips when I had time, and for putting up with and supporting me when I was busy
and boring.  Here is to many more adventures together!
