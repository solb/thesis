\section{Related work}
\label{sec:related}

\begin{table*}
\ifdefined\mytableistoobig
	\scriptsize
\else
	\small
\fi
\begin{tabular}{c||c|c|c|c|c|c}
&&& \multicolumn{2}{c|}{Dependencies} & \multicolumn{2}{c}{Third-party code support} \\
System & Preemptive & Synchronous & In userland & Works without GC & Preemptible & Works without recompiling \\
\hline
\textit{Scheme engines} & \checkmark* & \checkmark & \checkmark && $\dagger$ & \checkmark \\
\textit{Lilt} && \checkmark & \checkmark && $\dagger$* & --- \\
\textit{goroutines} & \schrodingerscheckmark && \checkmark && $\dagger$* & --- \\
$C\forall$ & \checkmark && \checkmark & \checkmark & $\dagger$* & --- \\
\textit{RT library} & \checkmark && \checkmark & \checkmark && \checkmark \\
\textit{Shinjuku} & \checkmark &&& \checkmark & $\dagger$ & \\
\hline
\textit{libinger} & \checkmark & \checkmark & \checkmark & \checkmark & \checkmark & \checkmark
\end{tabular}

\vspace{12pt}
\centering{\checkmark* = the language specification leaves the interaction with blocking system calls unclear} \\
\centering{$\dagger$ = assuming the third-party library is written in a purely functional (stateless) fashion} \\
\centering{$\dagger$* = the third-party code must be written in the language without foreign dependencies} \\
\centering{(beyond simple recompilation, this necessitates porting)}
\vspace{6pt}
\caption{Systems providing timed code at sub-process granularity}
\label{tab:related}
\end{table*}

A number of past projects (Table~\ref{tab:related}) have sought to provide
bounded-time execution of chunks of code at sub-process granularity.
For the purpose of our discussion, we
refer to a portion of the program whose execution should be bounded as \textbf{timed
code} (a generalization of a preemptible function); exactly how such code is
delineated depends on the system's interface.

Interface notwithstanding, the systems' most distinguishing
characteristic is the mechanism by which they enforce execution bounds.  At one end
of the spectrum are \textbf{cooperative} multitasking systems where
timed code voluntarily cedes the CPU to another
task via a runtime check.  (This is often done implicitly; a simple example is a
compiler that injects a conditional branch
at the beginning of any function call from timed code.)
Occupying the other extreme are \textbf{preemptive} systems that externally
pause timed code and transfer control to a scheduler routine (e.g., via
an interrupt service routine or signal handler, possibly within the language's VM).

The cooperative approach tends to be unable to interrupt two classes of timed code:\@
(1) \textbf{blocking-call} code sections that cause
long-running kernel traps (e.g., by making I/O system calls),
thereby preventing the interruption logic from being run; and (2)
\textbf{excessively-tight loops} whose body does not contain any yield points (e.g.,
spin locks or long-running CPU instructions).
Although some cooperative systems refine their approach with mechanisms
to tolerate either blocking-call code sections~\cite{www-golang} or excessively-tight
loops~\cite{vanderwaart:cmucs2006}, we are not aware of any that are capable of
handling both
cases.

One early instance of timed code support was the \textit{engines} feature of
the Scheme 84 language~\cite{haynes:iucs1984}.  Its interface was a new
\texttt{engine}
keyword that behaved similarly to \texttt{lambda}, but created a special ``thunk''
accepting as an argument the number of ticks (abstract time units) it should run for.
The caller also supplied a callback function to receive the
timed code's return value upon successful completion.  Like the rest of the
Scheme language, engines were stateless:\@ whenever one ran out of computation time,
it would return a replacement engine recording the point of interruption.  Engines'
implementation relied heavily on Scheme's managed runtime, with ticks
corresponding to virtual machine instructions and cleanup handled by the garbage
collector.  Although the paper mentions timer interrupts as an alternative, it does
not evaluate such an approach.

\textit{Lilt}~\cite{vanderwaart:cmucs2006} introduced a language for writing
programs with statically-enforced timing policies.
Its compiler tracks the possible duration of each path through a program and
inserts yield operations wherever a timeout could possibly occur.  Although this
approach requires assigning the execution limit at compile time, the compiler is able
to handle excessively-tight loops by instrumenting backward jumps.
Blocking-call functions remained a challenge, however:\@ handling them would have
required
operating system support, reminiscent of \textit{Singularity}'s static language-based
isolation~\cite{hunt:msr2005}.
