\section{Related Work}

\begin{table*}
\begin{tabular}{c||c|c|c|c|c|c}
&& Divorces from & \multicolumn{2}{c|}{Dependencies} & \multicolumn{2}{c}{Third-party code support} \\
System & Unblockable & threading & In userland & Works without GC & No redesign & No recompilation \\
\hline
\textit{Scheme engines} & \checkmark* & \checkmark & \checkmark && $\dagger$ & --- \\
\textit{Lilt} && \checkmark & \checkmark && $\dagger$ & \\
\textit{goroutines} &&& \checkmark &&& \\
\textit{Realtime library} & \checkmark && \checkmark & \checkmark && \\
\textit{Shinjuku} & \checkmark &&& \checkmark && \\
\hline
\textit{libinger} & \checkmark & \checkmark & \checkmark & \checkmark & \checkmark & \checkmark
\end{tabular}
\centering{* = the language specification leaves the interaction with blocking system calls unclear} \\
\centering{$\dagger$ = assuming the third-party library is written in a purely functional (stateless) fashion}
\caption{Systems providing intra-process bounded execution time}
\label{tab:related}
\end{table*}

\dga{Make the word timed function more generic so we can use it for threads too.}

A number of past projects (Table~\ref{tab:related}) have sought to provide
bounded-time execution of chunks of code at a sub-process granularity.
For the purposes of our discussion, we
refer to the portion of the code whose execution should be bounded as a \textbf{timed
function} (a generalization of preemptible functions); exactly what form this takes
depends on the system's interface.

Their interfaces notwithstanding, perhaps the systems' most distinguishing
characteristic is the mechanism by which they enforce execution bounds.  At one end
of the spectrum are \textbf{cooperative} multitasking systems where a
running timed function implicitly but voluntarily cedes use of the CPU to another
task via a runtime check.  (A simple example is a system that injects code to check
whether the timed function has expired at the beginning of every one of its function
calls.)  The other extreme is filled by \textbf{preemptive} systems where the CPU
interrupts the timed function and uses some independent code to reassign the CPU to
another task.  (This is analogous to the clock interrupt--based scheduler provided
by modern operating system kernels.)

\dga{Dave's stack can't handle the I/O system calls parenthetical.}

\dga{Refined approaches is too vague.}

The cooperative approach has difficulty handling two types of timed functions, which
may be able to avoid interruption:\@ (1) \textbf{blocking-call} ones that cause
long-running traps (e.g., by making I/O system calls) transfer control to the kernel,
thereby preventing the interruption logic from being run; and (2)
\textbf{compute-bound} ones (e.g., tight loops), which may simply never call into the
interruption logic.  Nonpreemptive systems may employ refined approaches to mitigate
these problems, and in theory handle all the cases a preemptive one would be able to;
in such a case, we classify them as \textbf{unblockable}, since they do not assume
that the timed function is well behaved.

One early well-known example of timed functions was the \textit{engines} feature of
the Scheme 84 language~\cite{haynes:iucs1984}.  The feature added an \texttt{engine}
keyword that behaved similarly to the \texttt{lambda} one, but created a "thunk"
accepting as an argument the number of ticks (abstract time units) it should run for.
In their interface, the caller also supplied a callback function to receive the
timed function's return value upon successful completion.  Like the rest of the
Scheme language, engines were stateless:\@ whenever one ran out of computation time,
it would return a new engine representing the new execution point.  The
implementation of engines relied heavily on Scheme's heavyweight runtime, with ticks
corresponding to virtual machine instructions and cleanup handled by the garbage
collector.  The VM dependency was not fundamental, however, as the literature notes
that timer interrupts could have been used instead.

In 2006, C.\@ Joseph Vanderwaart introduced a language, \textit{Lilt}, for writing
programs with statically-enforced timing policies~\cite{vanderwaart:cmucs2006}.  The
language's compiler tracks the possible duration of each path through the program and
and inserts yield operations wherever its time might have expired.  Although this
approach requires knowing the execution limit at compile time, the compiler is able
to handle compute-bound timed functions thanks to its awareness of backward jumps.
Blocking-call functions are a challenge, however:\@ handling them would require
operating system support, reminiscent of \textit{Singularity}'s static language-based
isolation~\cite{hunt:msr2005}.

\dga{Topic sentence broken. ``Recent language support for userland threading could be
used to provide support for timed functions.  Once such example is goroutines...}

More recent language support for timed functions is provied by the Go
language~\cite{www-golang} in the form of \textit{goroutines}.  It breaks from
Scheme's interface by presenting such functions as separate threads of executions.
The language's heavyweight runtime is responsible for scheduling multiple such user
threads per kernel thread, which is performed cooperatively by conditionally yielding
at function call points.  This has long caused problems with tight loops, with the
traditional workaround being to manually add calls to the \texttt{runtime.Gosched()}
yield function~\cite{www-golang-tightloop}.

\mk{Just use the authors' names (e.g., M\&A-RT or just RT).}

All the solutions discussed thus far rely on languages with a heavyweight, managed
runtime that includes garbage collection.  However, some recent systems have sought
to provide timed functions in a language-agnostic fashion that minimizes such
dependencies.  One example is the userland C-language thread library for realtime
systems developed by Mollison and Anderson~\cite{mollison:rtas2013}, which performs
preemption using timer interrupts, as proposed in the early Scheme engines
literature.  They install a time-based signal handler responsible for both scheduling
tasks and migrating them between cores.  Thanks to their lightweight runtime, they
are able to achieve average overall scheduling latencies in the tens of microseconds.

\mk{It's not obvious why M\&A-RT isn't our panacea.  And I couldn't tie the idea of
shared state back to the third-party code support column in the table.}

\textit{Shinjuku}~\cite{Kaffes:nsdi2019} is an operating system designed to perform
preemption at microsecond scale.  Built on the Dune framework~\cite{Belay:osdi2012},
it runs tasks on a pool of worker threads controlled by a single centralized
dispatcher thread.  The later thread polices how long each task has been running and
sends an inter-processor interrupt (IPI) to any worker whose task has been running
for longer than permitted.  The authors study the cost of IPIs and the overheads
imposed by performing them within a VT-x virtual machine, as required by Dune.  They
then implement optimizations to reduce these overheads at the expense of Shinjuku's
isolation from the rest of the system.

\dga{The FFI bit is jarring as a dependent clause.  Adding timed functions to a
system that doesn't assume concurrency causes problems with shared state, and is only
saved by the fact that they only support functional code.  Then talk about Go's
specific FFI limitation separately.}

Because Scheme engines and Lilt exist within a functional programming ecosystem, they
are able to assume that timed functions will not interact with shared state.  This
would be an onerous requirement to impose in languages such as C, where it is
common for libraries to rely on such state for error handling (e.g., libc's
\texttt{errno} variable), resource allocation and assignment, caching and
memoization, and countless other purposes.  Indeed, handling shared state turns out
to be the key challenge of providing timed functions, and one that prior work does
not attempt to address:\@ foreign-function interface (FFI) calls to other languages
don't work from Scheme engines or Lilt, are considered nonpreemptible when they
occur in goroutines~\cite{www-golang-fficall}, and are not discussed by either the
realtime threading library or Shinjuku.

The fundamental problem with shared state is that interrupting nonreentrant code can
leave its storage in an undefined state.  This implies that if we have two functions,
\texttt{foo()} and \texttt{bar()}, that access the same state, and we inteleave at
runtime a call to \texttt{bar()} in the middle of the execution of \texttt{foo()},
the result is undefined behavior.  In the C language, it's important to consider
shared state whenever writing a signal handler, since it may be called at any point
in the program's execution.  The POSIX standard defines the term async-signal-safe to
refer to a function that is safe to invoke from a signal
handler~\cite{signal-safety-manpage}; the list of standard library functions required
to exhibit this property notably excludes all heap-allocation functions such as
\texttt{malloc()}.  Note that async-signal-safety is a stronger requirement than
thread safety, since the code may interrupt itself, even within a critical section!

\dga{Stronger requirement point should come earlier.  Addendum: Traditional
approaches to thread safety would result in deadlock under the self-interruption
case.  Circumventing would require substantial interface-level (breaking API)
changes.}

\dga{Note that solutions involving allowing the timed function to continue running
are not acceptable because they give up our preemption property.  Maybe a concrete
example?}

A key observation about timed functions is that, as soon as an application calls one,
its safety properties change.  In particular, because the function might not run to
completion, all code after the call site on that thread of execution may be thought
of as a signal handler.  It is therefore subject to the same limitation of only being
allowed to call async-signal-safe functions!  Needless to say, such a requirement
would make timed functions almost impossible to use correctly, so a large portion of
our work focuses on lifting this requirement in the common case.

In contrast to modern approaches' thread-like model, we argue for a return to Scheme
engines' view of timed functions as procedure calls.  However, we seek to provide
such an abstraction in a language-agnostic manner, without relying on a managed
runtime, virtual machine, or garbage collector.  Furthermore, we seek to support
calls into most third-party libraries without requiring either their redesign (e.g.,
to eliminate global variables or migrate to reentrant interfaces) or their
recompilation (e.g., to allow a compiler to add scheduling checks).
