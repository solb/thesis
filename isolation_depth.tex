Our defense-in-depth comes from using lightweight operating-system protections
to block access to certain system calls, as well as the proposed mechanisms
in Section~\ref{sec:future}.  Some system calls must be blocked to have any
defense at all; otherwise, the microservice could create kernel threads (e.g.,
\texttt{fork()}), create competition between threads (e.g., \texttt{nice()}), or even
terminate the entire worker (e.g., \texttt{exit()}). Finally, user functions should
not have unmonitored file system access.

We propose to block system calls using Linux's \texttt{seccomp()} system
call~\cite{seccomp-manpage}; each worker process should call this during
initialization to limit itself to a \whitelist{ed} set of system calls.
Prior to lockdown, the worker process should install a \texttt{SIGSYS} handler
for regaining control from any microservice that attempts to violate the policy.
