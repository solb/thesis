\chapter{Foreword}

Perhaps it is inevitable that when a nonfiction work reaches a certain length, it
begins to serve multiple purposes; if so, this document is no exception.  Yes, it is
a record of the ideas I have explored over the past years of my life.  But like any
thesis, it is also a lesson:\@ in summarizing my findings from these explorations, it
endeavors to save you from spending years of your own on the same topic.  And like
any good lesson, this one begins with an exercise...

With your permission, we will conduct a brief mindfulness activity.  At this moment,
and for however much longer you focus on this document, you mind will be occupied by
ideas.  Many of these ideas I will have put there.  Shortly, they will be ideas about
computer systems, but first let us consider:  How are the ideas getting to you?  You
are reading, but what does that mean?  Perhaps you are holding a printout or a bound
copy of this manuscript, or perhaps you have loaded it onto your personal computer,
tablet, phone, or hand terminal.  In any case, you have opened it to a particular
page, exposing your eyes to a sea of shapes arranged into nested clusters.  Your eyes
have gravitated to a cluster of particularly large shapes known as a ``chapter
title,'' then they have scanned across the page and sent a compressed representation
of each smaller ``word'' cluster of shapes to your brain, which has matched the
clusters of shapes to entries in your mental lexicon, then parsed them according to a
set of linguistic rules to infer their logical connections.  Then you have moved on
to the large ``paragraph'' clusters and processed each in turn, starting with the
first of its ``sentence'' clusters, and in so doing, learning what the next sentences
will be about.  Occasionally, something will go wrong at one of these steps and you
will backtrack and notice a missed word, or more closely examine a misidentified
word, or try a different parsing of the sentence, or reexamine the logical flow of
the paragraph.  You will usually not realize you are doing any of this, preferring
to think simply that you are \textbf{reading}.

The document you are reading is about computer systems, and like your brain, such
systems have many complexities.  If we as computer users had to describe the full
process for doing everything, we would never accomplish anything, so instead we
build \textbf{abstractions} for performing common tasks without examining the
underlying details.  Some would say that any computer systems research is
fundamentally about abstractions.  This particular work centers around an abstraction
for use by application programmers, who in turn work on top of other abstractions,
the most notable of which is software called the \textbf{operating system}.

In computing, as in life, one's fundamental goal is to accomplish some task using a
set of shared resources.  Someone must decide how to allocate these shared resources,
a role usually filled for a particular resource by a piece of software called its
\textbf{scheduler}.  One major responsibility of the operating system is to share
hardware resources such as the processor, the short- and long-term storage devices,
devices for user interaction, and network interfaces.  Among these, the one most
relevant to our discussion is the CPU scheduler, which manages the processor.

Despite itself being an abstraction that hides enormous complexity, a processor is
conceptually quite simple:\@ it receives a stream of simple \textbf{instructions}
telling it what to do, executing them in the order received and occasionally jumping
to a particular point elsewhere in the stream when so instructed.  The simplicity of
this model belies the infinite expressive power of programs constructed from such
instructions.  Indeed, programming at the instruction level is difficult not only
because the simplicity of the instructions make it verbose, but also because ad-hoc
jumps can be deceptively complicated to reason about.  Modern programmers usually
write software in programming languages that provide so-called ``structured control''
abstractions for performing common, formulaic sequences of jumps.

The most fundamental abstraction composing a structured program is the
\textbf{function}:\@ a section of code that expects zero or more input data, performs
some computation, and generates zero or more output data.  One function can call
another, which automatically transfers the input data and jumps to the start of that
function's code.  Later, when the end of its code is reached, the function
automatically jumps back to the program point just after it was called and transfers
its output data back.  Notice that a function call is \textbf{synchronous}; that is,
the function runs to completion before the calling function continues to run.
Because such sequential execution matches the processor's inherent behavior, sharing
the processor between functions is trivial and requires no scheduler.

Of course, an important feature of modern computers is the ability to work on
multiple tasks alongside each other, such as reading a document and composing an
outline or notes.  Operating systems manage such situations by providing an
abstraction called a \textbf{process}, or independent task.  Each process is isolated
from the others on the system and cannot access their data.  Furthermore, processes
exhibit a property known as \textbf{concurrency} wherein their executions can
interleave such that one process executes some of its code ``in the middle of''
another process's work.  (Think of momentarily putting your notetaking on hold to
scroll down in the document you're reading.)

Because isolation prevents processes from calling each other's functions directly,
switching between processes requires a scheduler to transfer control of the
processor.  Specifically, the processor must stop executing the running process and
start running the operating system's CPU scheduler code, which then performs an
action called a \textbf{context switch}:\@ it saves a checkpoint of that process and
restores the other process, resuming it from the state in which it last left off.
The conceptually simpler way for this transition to happen is \textbf{cooperative}
multitasking, in which the former process voluntarily gives up control of the
processor by explicitly telling the operating system to give someone else a turn.
Unfortunately, it is not safe to assume a process will eventually cede its processor,
as it may never decide to do so, through either misbehavior or malice.  Such a
scenario would render the rest of the programs unusable.

Fortunately, processors have a low-level mechanism for spontaneously changing which
instruction they are executing known as a \textbf{timer interrupt}.  Every so often,
the processor jumps into the OS scheduler from whatever code it is currently
executing.  Since it is now has the use of the processor, the scheduler can decide
whether to jump back to that same program or context switch to a different one, a
decision that is usually made based on how long the former program had been running
since the last context switch.  This style of process scheduling is known as
\textbf{preemptive} multitasking because the operating system initiates it by
actively pausing the running process.

Recent decades have seen the introduction of multicore computers that have more than
one processor, creating the opportunity for the operating system to schedule a
different process on each.  Such processes exhibit \textbf{parallelism}:\@ they
actually run at the same time.  Parallelism is also an attractive feature for
application programmers because by carefully restructuring their programs, they can
route some of their work to each processor, thereby speeding up portions of their
program's run.  Unfortunately, fitting such programs into operating systems' existing
process model was cumbersome.

To better accommodate parallel programs, operating systems introduced a hybrid
abstraction called a \textbf{thread}.  Like processes, threads can be both concurrent
and parallel.  Unlike processes, though, threads must share data to effectively
work together on a single task, so the threads within a process are not isolated from
one another.  It turns out that the simultaneous presence of concurrency and shared
data introduces fundamental challenges that make it difficult to write correct
programs due to a class of bugs informally known as race conditions.  Although safe
concurrency is a popular area of study, challenges remain particularly in systems
containing components that predate the parallel programming paradigm.  More detailed
coverage of safe concurrency and backwards compatibility as they relate to this
thesis work appears in chapters~\ref{chap:libinger} and \ref{chap:libgotcha},
respectively.

The lack of isolation between threads permits the programmer to spawn a thread in
much the same way they would call a function:\@ in most programming languages, they
place the code they want to execute on the new thread in its own function, but
instead of calling it directly, they pass it to a special wrapper function.  The
wrapper sets up the thread and begins running the programmer's custom thread thereon.
However, in an important break from functions, threads are \textbf{asynchronous} like
processes.  The wrapper function returns almost instantly, even if the thread is
still running in the background.  As with processes, this property means there must
be a scheduler to decide which application code each processor should run.  Note
that for the sake of this discussion, we are assuming this is the operating system's
CPU scheduler; however, this is not always the case and sometimes a custom scheduler
runs as part of the application itself, a configuration that is addressed in detail
in chapter~\ref{chap:libturquoise}.

Introducing additional scheduler dependencies on an application has important
functionality and performance ramifications for two fundamental reasons.  First, the
scheduler's placement behind an abstraction decouples it from the program's logic,
thereby imposing one or more levels of communication barrier that reduce its
understanding of the particular application's needs, often resulting in a brittle
policy ill suited to the workload.  For instance, few preemptive schedulers provide a
way for an application to customize the timer interrupt interval, even when supported
by the hardware.  Second, every scheduler works by running its own code to make
decisions about how to allocate a resource.  Because it does not represent useful
work from the application's perspective, time spent this way is pure overhead, and it
follows that introducing unnecessary scheduling necessarily reduces performance.

Threads represent the standard application for exploiting preemption within an
application.  However, reminiscent of how processes were cumbersome to use for
parallel programming, threads are ill suited to some use cases of preemption.  For
one thing, programmers who do not need parallelism are led to build their synchrony
atop asynchrony, thereby introducing a useless scheduler dependency.  For instance,
when calling a helper function but needing a result by a specific deadline is tempted
to spawn the function on its own thread, then immediately wait for the thread to
finish, a task better accomplished on the same thread.  Furthermore, although they
support pausing code mid-execution, threads make it very difficult to cancel
in-progress work that is no longer needed at all.

Fortunately, the tendency to use threads for application-level preemption is not
because the operating system does not expose hardware features such as timers.
Rather, it is because such features are presented as very low-level abstractions that
perform hardware-style unstructured jumps rather than using language-style structured
control and abstracting away the details of context switching.  We therefore propose
a new abstraction for easy preemption within an application, but show that it can be
implemented on top of the existing operating system.
