\chapter{Rethinking POSIX safety: \\ \textit{libas-safe} and \textit{libac-safe}}
\label{chap:safety}

\ifdefined\chapquotes
\vspace{-1in}
\begin{chapquote}[1.75in]{Douglas Adams, \textit{The Hitchhiker's Guide to the Galaxy}}
`Ah!  This is obviously some strange use of the word \textit{safe} \\
that I wasn't previously aware of.'
\end{chapquote}
\fi

\solb{Explain what POSIX means by the term asynchronous (and audit how we've used it before now)}

The reader will probably not be surprised to learn that the POSIX specification
guarantees that most C library functions are thread safe; that is, assuming they are
not explicitly instructed to use the same resources, it is safe to call them from
concurrent kernel threads within the same process.  However, POSIX also defines two
less familiar types of safety that become relevant when an individual thread does
something other than execute a single task to synchronous completion or failure:

\paragraph{Async-signal safety.}
Async-signal-safe functions are those that can be safely called from a signal handler
that has interrupted a non-async-signal-safe function (or conversely, can be safely
interrupted by a signal handler that calls non-async-signal-safe functions).

\begin{sloppypar}
\paragraph{Async-cancel safety.}
Async-cancel-safe functions are those that can be called by
asynchronously-cancellable threads, which we first saw in
Chapter~\ref{chap:functions}.  One can ostensibly cancel such threads at any point in
their execution; however, POSIX marks almost no functions as async-cancel safe, so in
practice the feature is only useful for threads executing a compute-bound loop with
no I/O or other reliance on the OS.
\\
\end{sloppypar}

Along with thread safety, these two classes of safety exist because of nonreentrant
interfaces; that is, functions whose signatures do not expose all the shared state
they use.  To understand the need for these two classes of safety, it is helpful to
consider how one might implement a thread-safe function.  For the sake of this
discussion, consider the nonreentrant pseudorandom-number generator \texttt{rand()},
which takes no arguments but returns a random number.  Clearly, the function needs
some kind of entropy pool to produce such a number, and since the caller doesn't
provide it with any information, the \texttt{rand()} function must manage the entropy
pool itself.  This implies the function has internal state that is implicitly shared
among all its callers.  The simplest way to implement an entropy pool is to feed some
seed value to a one-way function, and use the resulting pseudorandom number as the
new seed value.  Thus, the entropy pool needs only to store the current seed value.

It's easy to see that a trivial implementation of \texttt{rand()} using the
aforementioned approach is subject to data races when used by multiple threads:\@
the function must update the entropy pool on each invocation, and concurrent accesses
may interleave.  Establishing thread safety is as easy as using a thread-local
variable to maintain a separate entropy pool for each thread of execution, thereby
eliminating the shared state.  Unfortunately, this mitigation is neither applicable
to async-signal safety nor to async-cancellation safety:  In the former case, there
is no analogue of thread-local variables capable of retargeting data accesses
depending on whether a signal handler is running.  In the latter case, a function
that mutates state that must be shared \textit{between} threads is likely to corrupt
such state if cancelled in the middle of writing to it, even if the function employs
concurrency control to prevent data races.  (One such example is the
\texttt{malloc()} family of dynamic memory allocation functions, which carve their
allocations out of a fixed heap.  Although they take a lock on a portion of the heap
while reserving each allocation, cancelling them during this critical section will
result in the lock never being released and the affected portion of the heap becoming
unusable.)

In this chapter, we explore how \textit{libgotcha} fills a usability gap in Unix's
userland and allows one to establish either kind of async safety in places where it
would not otherwise exist.  In doing so, we present ordinarily unsound sample
programs that can be automatically fixed through the use of \textit{libgotcha} and
discuss practical applications to real-world systems.  And through its presentation
of two simple control libraries, this chapter also serves as a straightforward
example of how to use \textit{libgotcha}.  The insight from this chapter, both on our
runtime and on POSIX safety itself, will be useful when we later explain the workings
of \textit{libinger}, a more complicated control library.


\section{Establishing async-signal safety:\@ \textit{libas-safe}}

Our approach to artificially establishing async-signal safety is to repurpose
selective relinking to isolate signal handlers from the rest of the program.  We do
so by running the entire program, with the exception of uninterruptible library
functions (Section~\ref{sec:libgotcha:unint}) and custom signal handlers, in a
newly-allocated libset.  Whenever a signal with a handler arrives, we switch to the
starting libset before executing it, then switch back before returning to the rest of
the program.

As a demonstration of our technique, we have implemented \textit{libas-safe}, a tiny
runtime comprising 127 lines of C code that automatically fixes programs whose signal
handlers call functions that are not async-signal safe.  To use it you, either
preload it at load time or link your buggy application directly against it at build
time.  Note that it only fixes bugs truly arising from async-signal safety:\@ it will
neutralize most resulting undefined behavior, but it will not address logic errors in
the program itself (e.g., a handler's attempt to traverse a corrupt or otherwise
inconsistent data structure).  Furthermore, it is a proof of concept and there are
cases it does not bother to handle.  Most notably, it does not isolate handlers for
different signals from one another, so programs that handle multiple signals must
ensure the other(s) are masked while any handler that calls unsafe functions is
running.\footnote{Technically, this stipulation is slightly stronger than necessary,
both in terms of scope (all other handlers) and enforcement mechanism (signal masks).
The exact requirement is that no two handlers that both use unsafe functions can be
allowed to interleave their execution.}

To avoid affecting the initialization of the C runtime, we perform our own setup as
late as possible by replacing libc's \texttt{\_\_libc\_start\_main()} function,
responsible for calling the program's \texttt{main()} function.  Because
\textit{libas-safe} is an internal control library
(Section~\ref{sec:libgotcha:control}), doing so is as simple as defining a
non-static function with that name and \textit{libgotcha} automatically considers it
a forced interposition (Section~\ref{sec:libgotcha:interpose}).  Our replacement
wraps the libc implementation, but allocates and switches to a new libset just before
jumping to \texttt{main()}.  It also registers an uninterruptible return callback
(Section~\ref{sec:libgotcha:callbacks}) and checks an environment variable to
determine wither to run in verbose mode and log its actions.

The bulk of \textit{libas-safe}'s code merely wraps the \texttt{sigaction()} function
for installing signal handlers.  We show the replacement for this function in
Listing~\ref{lst:assafe_sigaction}, slightly simplified for
brevity.\footnote{Compared to our actual prototype, the version in these source
listings runs internal \textit{libas-safe} wrapper code with the same signal mask that
the caller requested for \textit{its} signal handler.  This can lead to conflicts
between \textit{libas-safe}'s own handlers for different signals, or even between its
handler for one signal and that handler itself if the caller installs the handler
with the \texttt{SA\_NODEFER} flag.  Unrelatedly, the depicted version of the
\texttt{sigaction()} wrapper does not roll back its changes if the underlying
implementation reports an error.}  If used to set a signal's disposition to default
(\texttt{SIG\_DFL}) or ignored (\texttt{SIG\_IGN}) or query the configuration of a
signal without a custom handler, neither of the conditionals is taken and it defers
to the underlying \texttt{sigaction()} (in this case, \textit{libgotcha}'s own
wrapper).  Otherwise, if the caller is installing a custom handler, it saves a
pointer to the provided handler into the persistent \texttt{handlers} array and
installs its own \texttt{stub()} function as the handler instead; this function
expects three arguments rather than System V's traditional one, so it sets the
\texttt{SA\_SIGINFO} flag~\cite{sigaction-manpage}.  If the caller is querying the
configuration of a custom handler, it looks up the handler the program had requested
and furnishes that instead of a pointer to \texttt{stub()}.

\begin{figure}
\begin{lstlisting}[label=lst:assafe_sigaction,caption=\textit{libas-safe}'s \texttt{sigaction()} replacement]
typedef void (*handler_t)(int, siginfo_t *, void *);

static struct handler_t *handlers;
static bool verbose;

int sigaction(int signum, const struct sigaction *act, struct sigaction *oldact) {
	handler_t oldact = handlers[signum];
	struct sigaction sa;
	if(act && act->sa_sigaction != SIG_DFL && act->sa_sigaction != SIG_IGN) {
		// We have been asked to install a custom signal handler.
		if(verbose) fprintf(
			stderr,
			"LIBAS-SAFE: sigaction() installing signal %d handler\n",
			signum);

		memcpy(&sa, act, sizeof sa);
		handlers[signum] = sa.sa_sigaction;
		sa.sa_sigaction = stub;
		sa.sa_flags |= SA_SIGINFO;
		act = &sa;
	}

	// Call the real sigaction().
	int status = libgotcha_sigaction(signum, act, oldact);
	if(oldact && oldact->sa_sigaction == stub)
		// We have been asked to provide the previous configuration.
		// Fib that we installed the provided handler, not our wrapper.
		oldact->sa_sigaction = oldact;
	return status;
}
\end{lstlisting}
\end{figure}

The \texttt{stub()} function serves as a wrapper for each installed signal handler,
and is shown in Listing~\ref{lst:assafe_helpers}.  If the program is interruptible
(that is, the next libset is not equal to the starting one), neither of the
conditionals is taken.  In this case, the wrapper simply switches to the starting
libset, calls the real handler for the arriving signal, then resets the libset.

Things are more complicated if the signal arrives while uninterruptible code is
running, in which case \textit{libas-safe} defers invoking its handler until the end
of the uninterruptible section.  In this case, \texttt{stub()} takes its
\texttt{else if} branch and stores the \texttt{siginfo\_t} structure describing the
cause of the signal into the thread-local \texttt{pending} variable.\footnote{Our
prototype does not support deferring more than one distinct signal at a time, and
always forwards the \texttt{siginfo\_t} corresponding to the first instance thereof
to arrive.  It follows the semantics of non-realtime Unix signals and only delivers
a deferred signal once, regardless of how many times it occurred while blocked.}  It
then changes the signal mask of the \textit{calling} code to block the signal from
arriving and returns without invoking the handler.  Whenever the program becomes
interruptible again, \textit{libgotcha} will invoke the \texttt{restorer()} callback,
also shown in Listing~\ref{lst:assafe_helpers}.

If it finds a deferred signal to deliver, the callback sends the current thread that
signal if it is not already pending (i.e., if it has not arrived again since the
instance that prompted us to defer it).  It then uses \texttt{sigsuspend()} to
temporarily unblock the signal and atomically wait for its handler to run.  This
jumps back to \texttt{stub()}, which now enters its \texttt{if} branch, sets the
signal as no longer deferred, configures it to be unblocked upon return from the
handler, and substitutes the saved \texttt{siginfo\_t} for the real one (in case
\texttt{restorer()} had to signal the thread).  Finally, it calls the real
handler and leaves the starting libset.\footnote{If curious why restoring the libset
in this way works, see the sister footnote in Section~\ref{sec:libgotcha:callbacks}.}

\begin{figure}
\begin{lstlisting}[label=lst:assafe_helpers,caption=\textit{libas-safe}'s signal handler wrapper and control library callback,morekeywords=thread_local]
static thread_local siginfo_t pending;

static void stub(int no, siginfo_t *si, ucontext_t *uc) {
	libset_t libset = libset_thread_get_next();
	if(pending.si_signo) {
		// It is time to deliver a signal we had previously deferred.
		assert(pending.si_signo == no);
		pending.si_signo = 0;
		sigdelset(&uc->uc_sigmask, no);
		si = &pending;
	} else if(libset == LIBGOTCHA_LIBSET_STARTING) {
		// The program is uninterruptible; we need to defer delivery.
		if(verbose) fprintf(
			stderr,
			"LIBAS-SAFE: stub() deferring handling of signal %d\n",
			no);

		memcpy(&pending, si, sizeof pending);
		libgotcha_sigaddset(&uc->uc_sigmask, no);

		// Do not call the handler at this time.
		return;
	}

	libset_thread_set_next(LIBGOTCHA_LIBSET_STARTING);
	handlers[no](no, si, co);
	libset_thread_set_next(libset);
}

static void restorer(void) {
	if(pending.si_signo) {
		// There is a deferred signal to deliver.
		sigset_t ready;
		sigpending(&ready);
		if(!sigismember(&ready, pending.si_signo))
			libgotcha_pthread_kill(pthread_self(), pending.si_signo);

		sigset_t full;
		libgotcha_sigfillset(&full);
		sigdelset(&full, pending.si_signo);
		sigsuspend(&full);
	}
}
\end{lstlisting}
\end{figure}

While the technique employed by \textit{libas-safe} makes it easier to write signal
handlers in C, this is not its most exciting application.  Async-signal safety is a
very specific case that otherwise sound systems programming type systems have long
struggled to handle.  Rust is no exception, and writing signal handlers requires
\texttt{unsafe} code.  Yet this style of runtime assistance may offer a route to
lifting this requirement in situations where the load time and runtime costs are
acceptable.


\subsection{Automatically repaired example program}

\solb{Find a nice example, ideally one that works for both copying and deferral}

\solb{Mention that there's actually no need to save and restore errno now!}


\section{Establishing async-cancel safety:\@ \textit{libac-safe}}

To artificially establish async-cancel safety, we apply selective relinking in a
slightly different manner.  Instead of establishing partial memory isolation between
signal handlers and the rest of the program, we establish it between each thread and
every other.  Whenever the program spawns a new thread, we allocate and install a
private libset for it, then enable POSIX asynchronous cancelability.  In this way,
every kernel thread in the application except for the main one becomes cancelable at
almost any point in its execution.

Our demo of this approach is called \textit{libac-safe}, and consists of 119 lines of
C that makes threads asynchronously cancelable.  As with \textit{libas-safe}, it can
be either preloaded or linked against like any other library.  And as before, it only
fixes problems arising in library functions due to async-cancel safety, not program
logic errors.  In addition to observing traditional concurrency control practices,
the developer of an application using asynchronously cancelable threads must exercise
extreme caution when interacting with threads that might be canceled in this way or
accessing any data structures that such threads modify.  Our prototype is
experimental and omits important features such as automatic cleanup of resources
allocated by canceled threads (a topic we will revisit in
Chapters~\ref{chap:libinger} and \ref{chap:ingerc}).

As an internal control library, \textit{libac-safe} works by defining a forced
interposition function (Section~\ref{sec:libgotcha:interpose}) that wraps
\texttt{pthread\_create()}.  As shown in Listing~\ref{lst:acsafe_gotchas}, the first
time the application creates a thread, the library bootstraps itself by registering
uninterruptible call and return callbacks with \textit{libgotcha}
(Section~\ref{sec:libgotcha:callbacks}).  The former callback is invoked any time the
thread calls an uninterruptible function (Section~\ref{sec:libgotcha:unint}) and
automatically transitions the thread back to the default cooperative cancellation
mode, wherein certain calls into the C standard library implicitly check whether
there is an outstanding request for the thread to be cancelled.  The latter callback
happens at the end of the uninterruptible region, and transitions the thread into
preemptive cancellation mode once more~\cite{pthreadsetcanceltype-manpage}.  Once the
library has been bootstrapped or on subsequent calls, the wrapper simply packages up
the supplied pointer to the thread's main function and arguments along with a
newly-assigned libset.  It then calls into the real \texttt{pthread\_create()},
substituting the \textit{libac-safe} main function \texttt{wrapper()} and arranging
for it to be passed all of these items.

\begin{figure}
\begin{lstlisting}[label=lst:acsafe_gotchas,caption=\textit{libac-safe}'s \texttt{pthread\_create()} replacement and control library callbacks,morekeywords=thread_local]
static thread_local libset_t reuse;

struct wrapper {
	void *(*fun)(void *);
	void *arg;
	bool finished;
	libset_t libset;
	struct reuse *parent;
};

int pthread_create(pthread_t *thread,
		     const pthread_attr_t *attr,
		     void *(*fun)(void *),
		     void *arg) {
	static bool bootstrapped;
	if(!bootstrapped) {
		libset_register_callback(cooperative);
		libset_register_returnback(preemptive);
		bootstrapped = true;
	}

	struct wrapper *args = malloc(sizeof *args);
	args->fun = fun;
	args->arg = arg;
	args->finished = false;
	args->libset = alloc_libset();
	args->parent = &reuse;
	return pthread_create(thread, attr, wrapper, args);
}

static void cooperative(void) {
	pthread_setcancelstate(PTHREAD_CANCEL_DISABLE, NULL);
}

static void preemptive(void) {
	pthread_setcancelstate(PTHREAD_CANCEL_ENABLE, NULL);
}
\end{lstlisting}
\end{figure}

Listing~\ref{lst:acsafe_helpers} shows the additional initialization and teardown we
perform on each thread.  Once libpthread has created the new kernel thread, it runs
our \texttt{wrapper()} on it, which configures libpthread to run the
\texttt{release()} handler if the thread ever gets canceled.  It then sets the next
libset to the one allocated for this thread and initializes the locale selections of
its copy of libc.\footnote{Without this thread-specific initialization, calls to
\texttt{ctype.h} functions would invoke \texttt{NULL} pointers inside glibc.  The
reason we have to invoke it manually is that libpthread does so from within its
\texttt{pthread\_create()} implementation, but that (intentionally) happens before we
switch away from the starting libset.  Third-party libraries do not receive such
special glibc initialization treatment; instead, their ELF constructors handle setup.
Technically, a direct call into \texttt{\_\_ctype\_init()} as shown in the listing
will not work because \textit{libac-safe}'s status as an internal control library
means that its calls only ever resolve back into the starting libset
(Section~\ref{sec:libgotcha:control}).  Instead of using a simple function call, it
uses an additional \textit{libgotcha} API to look up the specific function pointer
for the allocated libset in the shadow GOTs, then makes an indirect call.}  Then it
calls into the thread's main function.  Assuming the thread does not get canceled,
this will eventually return back, at which point \texttt{wrapper()} will return to
the starting libset, set a flag to indicate that the thread ran to completion, and
call the \texttt{release()} handler.  This will skip the first conditional in the
latter function, save the libset identifier for future zero-cost reuse by the parent
thread (if it does not already have one saved), and deallocate our packaged thread
information.

If instead the thread gets prematurely cancelled, \texttt{release()} will be called
and find that the thread is not tagged as \texttt{finished}, at which point it will
leave the thread's private libset then forcefully reinitialize it.  Thereafter, it
proceeds as before.  Recall that cancellation will have happened cooperatively if the
thread was executing uninterruptible code and preemptively otherwise.  This respects
the selective relinking safety model; that is, assuming the allowlist is configured
correctly and glibc's implementation of cooperative cancelation, is sound,
cancelation will neither corrupt the starting libset nor create dependencies between
it and any thread's private libset (Section~\ref{sec:libgotcha:unint}).

Notice that, regardless of what caused the thread to terminate, it attempted to pass
its now-unused libset's identifier back to its parent for reuse on the next spawn.
The decision to reuse happens in \texttt{pthread\_create()}'s call to the
\texttt{alloc\_libset()} helper function, also shown in
Listing~\ref{lst:acsafe_helpers}.  This implementation takes two related
implementation shortcuts, neither of which is central to our approach's design:
(1)~Terminating child threads assume their parent still exists and directly read and
write its thread-local \texttt{reuse} record.  (2)~Each parent thread only remembers
up to one reusable libsets,\footnote{Because child threads do not atomically update
the parent's variable, the implementation does not guarantee \textit{which} libset
will be reused in case of a tie.  This is still safe because all contending libsets
are valid for reuse.} so if two or more of its child threads exit in between
any two consecutive times that it spawns, it will leak a libset that the process will
never reuse.  The correct solution is to use a pool allocator for libsets, and we
will see an example of this in Chapter~\ref{chap:libinger}.

In summary, \textit{libac-safe} makes it possible for Unix applications to use
asynchronous thread cancelability.  Thus, developers can now leverage a feature that,
as we saw in Chapter~\ref{chap:functions}, was practically useless out of the box.
The library also makes a good case study of a \textit{libgotcha} control library for
which it is correct to adopt the default behavior of tying TCBs to kernel threads
(Section~\ref{sec:libgotcha:tls}), since the kernel thread corresponds to the unit of
cancellation, and therefore memory isolation as well.  Because of this,
\textit{libgotcha} does not need to reinitialize TLS areas, as they are specific to
each thread and not reused with the libset.

\begin{figure}
\begin{lstlisting}[label=lst:acsafe_helpers,caption=\textit{libac-safe}'s thread initializer and cleanup handler]
static void *wrapper(wrapper *wrapper) {
	pthread_cleanup_push(release, wrapper);

	libset_thread_set_next(wrapper->libset);
	__ctype_init();

	void *res = wrapper->fun(wrapper->arg);

	libset_thread_set_next(LIBGOTCHA_LIBSET_STARTING);

	wrapper->finished = true;
	pthread_cleanup_pop(true);
	return res;
}

static void release(void *wrapper) {
	if(!wrapper->finished) {
		libset_thread_set_next(LIBGOTCHA_LIBSET_STARTING);
		libset_reinit(wrapper->libset);
	}
	if(*wrapper->parent != LIBGOTCHA_LIBSET_STARTING)
		// Our parent thread does not already have a libset to reuse.
		*wrapper->parent = wrapper->libset;
	free(wrapper);
}

static libset_t alloc_libset(void) {
	if(reuse != LIBGOTCHA_LIBSET_STARTING)
		// This thread has a leftover libset to reuse.
		return reuse;
	else
		// We have to use a fresh one.
		return libset_new();
}
\end{lstlisting}
\end{figure}


\subsection{Program repaired with the help of our system}

Listing~\ref{lst:acsafe_example} shows an only slightly contrived example program
that tries to use asynchronous thread cancelation.  It attempts to simulate
performing a large number of cancelable DNS lookups:  The main thread starts by
performing a reverse lookup on a link-local IPv4 address (in order to populate a
socket address structure).  It then enters an infinite loop, each iteration of which
updates a circular spinner to indicate that the program is making
progress,\footnote{Specifically, it calls \lstinline'printf("\%c\\b", spinner)', then
\lstinline'fflush(stdout)', and finally rotates \texttt{spinner} to the next of the
characters \texttt{|}, \texttt{/}, \texttt{-}, and \texttt{\textbackslash}.} spawns
a thread to resolve the hostname back to an address, cancels the thread, and joins on
it before spawning the next.  Each time a thread terminates prematurely, the loop it
increments a success counter; each time one runs to completion, it prints out the
current counter and the hostname the thread was passed.  Upon creation, each threads
immediately sets itself as asynchronously cancelable, performs a forward DNS lookup,
and exits.

\begin{figure}
\begin{lstlisting}[label=lst:acsafe_example,caption=Example program using asynchronous thread cancellation]
#define NAMESZ 10

struct args {
	socklen_t addrlen;
	const struct sockaddr *addr;
	char name[NAMESZ];
};

static void *thread(struct args *args) {
	pthread_setcanceltype(PTHREAD_CANCEL_ASYNCHRONOUS, NULL);
	getnameinfo(args->addr, args->addrlen, args->name, NAMESZ, NULL, 0, 0);
	return NULL;
}

int main(void) {
	struct addrinfo *ai;
	getaddrinfo("127.0.0.1", NULL, NULL, &ai);

	struct sockaddr_storage sa;
	socklen_t salen = ai->ai_addrlen;
	memcpy(&sa, ai->ai_addr, salen);
	freeaddrinfo(ai);
	ai = NULL;

	struct args args = {
		.addrlen = salen,
		.addr = (struct sockaddr *) &sa,
	};
	char spinner = '|';
	unsigned canceled = 0;
	while(true) {
		print_and_advance_spinner(&spinner);

		pthread_t tid;
		void *res;
		pthread_create(&tid, NULL, thread, &args);
		pthread_cancel(tid);
		pthread_join(tid, &res);
		if(res != PTHREAD_CANCELED)
			printf("%d %s\n", canceled, args.name);
		else
			++canceled;
	}
	return 0;
}
\end{lstlisting}
\end{figure}

The program does not work.  Asynchronous cancelation both fails to interrupt some
of the threads (likely due to the proximity between the call and a library
function that performs system calls) and eventually deadlocks the program, as
indicated by the spinner becoming frozen.  Here is the output of a representative
run, which lasted under three seconds before hanging:
\begin{lstlisting}
  69437 localhost
  75670 localhost
  128996 localhost
  -
\end{lstlisting}

Running the program with the environment variable \texttt{LD\_PRELOAD} defined to
\texttt{./libac-safe.so} almost fixes the problem with no programmer intervention.
All of the threads are canceled before they can complete, and the only problem is
that each of the calls to \texttt{getnameinfo()} allocates a new file descriptor that
is leaked upon the thread's death.  Eventually, this results in the program crashing
because it has too many open file descriptors.

After adding a cleanup handler, the program runs apparently forever without
deadlocking, allowing any of the threads to finish, or leaking file descriptors.  The
cleanup handler is as follows, and is registered from \texttt{thread()} using the
\texttt{pthread\_cleanup\_push()} function as in Listing~\ref{lst:acsafe_helpers}:
\begin{lstlisting}
  static void handler(void *ignored) {
  {
  	close(STDERR_FILENO + 1);
  }
\end{lstlisting}
