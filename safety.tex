\chapter{Rethinking POSIX safety: \\ \textit{libas-safe} and \textit{libac-safe}}
\label{chap:safety}

\ifdefined\chapquotes
\vspace{-1in}
\begin{chapquote}[1.75in]{Douglas Adams, \textit{The Hitchhiker's Guide to the Galaxy}}
`Ah!  This is obviously some strange use of the word \textit{safe} \\
that I wasn't previously aware of.'
\end{chapquote}
\fi

\solb{Explain what POSIX means by the term asynchronous (and audit how we've used it before now)}

The reader will probably not be surprised to learn that the POSIX specification
guarantees that most C library functions are thread safe; that is, assuming they are
not explicitly instructed to use the same resources, it is safe to call them from
concurrent kernel threads within the same process.  However, POSIX also defines two
less familiar types of safety that become relevant when an individual thread does
something other than execute a single task to synchronous completion or failure:

\paragraph{Async-signal safety.}
Async-signal-safe functions are those that can be safely called from a signal handler
that has interrupted a non-async-signal-safe function (or conversely, can be safely
interrupted by a signal handler that calls non-async-signal-safe functions).

\begin{sloppypar}
\paragraph{Async-cancel safety.}
Async-cancel-safe functions are those that can be called by
asynchronously-cancellable threads, which we first saw in
Chapter~\ref{chap:functions}.  One can ostensibly cancel such threads at any point in
their execution; however, POSIX marks almost no functions as async-cancel safe, so in
practice the feature is only useful for threads executing a compute-bound loop with
no I/O or other reliance on the OS.
\\
\end{sloppypar}

Along with thread safety, these two classes of safety exist because of nonreentrant
interfaces; that is, functions whose signatures do not expose all the shared state
they use.  To understand the need for these two classes of safety, it is helpful to
consider how one might implement a thread-safe function.  For the sake of this
discussion, consider the nonreentrant pseudorandom-number generator \texttt{rand()},
which takes no arguments but returns a random number.  Clearly, the function needs
some kind of entropy pool to produce such a number, and since the caller doesn't
provide it with any information, the \texttt{rand()} function must manage the entropy
pool itself.  This implies the function has internal state that is implicitly shared
among all its callers.  The simplest way to implement an entropy pool is to feed some
seed value to a one-way function, and use the resulting pseudorandom number as the
new seed value.  Thus, the entropy pool needs only to store the current seed value.

It's easy to see that a trivial implementation of \texttt{rand()} using the
aforementioned approach is subject to data races when used by multiple threads:\@
the function must update the entropy pool on each invocation, and concurrent accesses
may interleave.  Establishing thread-safety is as easy as using a thread-local
variable to maintain a separate entropy pool for each thread of execution, thereby
eliminating the shared state.  Unfortunately, this mitigation is neither applicable
to async-signal safety nor to async-cancellation safety:  In the former case, there
is no analogue of thread-local variables capable of retargeting data accesses
depending on whether a signal handler is running.  In the latter case, a function
that mutates state that must be shared \textit{between} threads is likely to corrupt
such state if cancelled in the middle of writing to it, even if the function employs
concurrency control to prevent data races.  (One such example is the
\texttt{malloc()} family of dynamic memory allocation functions, which carve their
allocations out of a fixed heap.  Although they take a lock on a portion of the heap
while reserving each allocation, cancelling them during this critical section will
result in the lock never being released and the affected portion of the heap becoming
unusable.)

In this chapter, we explore how \textit{libgotcha} fills a usability gap in Unix's
userland and allows one to establish either kind of async safety in places where it
would not otherwise exist.  In doing so, we present ordinarily unsound sample
programs that can be automatically fixed through the use of \textit{libgotcha} and
discuss practical applications to real-world systems.  And through its presentation
of two simple control libraries, this chapter also serves as a straightforward
example of how to use \textit{libgotcha}.  The insight from this chapter, both on our
runtime and on POSIX safety itself, will be useful when we later explain the workings
of \textit{libinger}, a more complicated control library.

\begin{promotesubsections}
\input[functions]{gotcha_assafe}
\end{promotesubsections}
