\documentclass[12pt,letterpaper,openright]{report}

\usepackage{amssymb}
\usepackage{booktabs}
\usepackage[T1]{fontenc}
\usepackage[margin=1in]{geometry}
\usepackage{graphicx}
\usepackage{hyperref}
\usepackage[mono=false]{libertine}
\usepackage{listings}
\usepackage{multirow}
\usepackage{subcaption}
\usepackage{xcolor}

\lstset{captionpos=b,
	basicstyle=\ttfamily,
	keywordstyle=\color{blue},
	commentstyle=\color{magenta},
	columns=flexible,
	language=C++
}

\newcommand{\solb}[1]{}

\newcommand{\ms}[1]{#1 ms}
\newcommand{\us}[1]{#1 $\mu$s}

\makeatletter
\let\includegraphics@\includegraphics
\renewcommand{\includegraphics}[2][]{\includegraphics@[#1]{\includegraphicsdir#2}}
\newcommand{\includegraphicsdir}{}

\let\input@\input
\renewcommand{\input}[2][]{
	\renewcommand{\includegraphicsdir}{#1/}
	\input@{#1/#2}
	\renewcommand{\includegraphicsdir}{}
}
\makeatother

\renewenvironment{abstract}{\itshape}

\begin{document}

\begin{titlepage}
\begin{center}
	\vspace*{\fill}

	\textbf{\Large Lightweight Preemptible Functions} \\
	A thesis proposal \\
	\hfill \\
	{\large Sol Boucher} \\
	\today \\

	\vspace{1in}

	\textbf{Thesis committee:} \\
	David G.\@ Andersen, \textit{chair} \\
	Adam Belay \\
	Michael Kaminsky \\
	Brandon Lucia \\

	\vspace{\fill}

	\textit{Submitted in partial fulfillment of the requirements \\
	for the degree of Doctor of Philosophy} \\
	\hfill \\
	Computer Science Department \\
	School of Computer Science \\
	Carnegie Mellon University \\
	Pittsburgh, PA 15213 \\
\end{center}
\end{titlepage}

\pagenumbering{roman}
\tableofcontents
\newpage
\pagenumbering{arabic}

\chapter{Introduction}

\textit{Contemporary general-purpose operating systems provide task preemption, which
they treat as a resource sharing mechanism:\@ when the total number of application
threads exceeds the number of processors, the kernel scheduler preempts long-running
or low-priority threads to allow others to run.  Preemption is also useful to
application programmers in its own right, and the operating system's choice of
preemption abstraction imposes a prescribed structure and architecture on programs
seeking to use this primitive.  Specifically, exposing preemption via the scheduler
encourages programmers to treat the thread as the unit of preemption, meaning that
a program making internal use of preemption is also parallel; however, because
preemption fundamentally entails concurrency but not parallelism, such an application
may be incurring unnecessary complexity and performance penalties.  We provide an
abstraction for preemption at the granularity of a synchronous function call, and
demonstrate that this represents a more efficient and composable interface that
enables new functionality for latency-critical applications, yet is expressive enough
to encode classic asynchrony primitives.
}

\input[functions]{intro}
\input[functions]{concl}


\chapter{Lightweight preemptible functions}

\input[functions]{abstract}
\input[functions]{inger}
\input[functions]{related}


\chapter{Nonreentrancy and selective relinking}

\input[functions]{gotcha}


\chapter{Preemptive userland threading}

\input[functions]{turquoise}
\input[functions]{eval}


\chapter{Microsecond-scale microservices}

\input[microservices]{abstract}
\input[microservices]{intro}
\input[microservices]{motivation}
\input[microservices]{isolation}
\input[microservices]{eval}
\input[microservices]{concl}


\chapter{Proposed work}

\section{Remaining work}
\section{Timeline}


\end{document}
