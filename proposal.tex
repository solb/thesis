\documentclass[12pt,letterpaper,openright]{report}
\usepackage[T1]{fontenc}
\usepackage[margin=1in]{geometry}
\usepackage{hyperref}
\usepackage{libertine}
\begin{document}

\begin{titlepage}
\begin{center}
	\vspace*{\fill}

	\textbf{\Large Lightweight Preemptible Functions} \\
	A thesis proposal \\
	\hfill \\
	{\large Sol Boucher} \\
	\today \\

	\vspace{1in}

	\textbf{Thesis committee:} \\
	David G.\@ Andersen, \textit{chair} \\
	Adam Belay \\
	Michael Kaminsky \\
	Brandon Lucia \\

	\vspace{\fill}

	\textit{Submitted in partial fulfillment of the requirements \\
	for the degree of Doctor of Philosophy} \\
	\hfill \\
	Computer Science Department \\
	School of Computer Science \\
	Carnegie Mellon University \\
	Pittsburgh, PA 15213 \\
\end{center}
\end{titlepage}

\pagenumbering{roman}
\tableofcontents
\newpage
\pagenumbering{arabic}

\chapter{Introduction}

\textit{Contemporary general-purpose operating systems provide task preemption, which
they treat as a resource sharing mechanism:\@ when the total number of application
threads exceeds the number of processors, the kernel scheduler preempts long-running
or low-priority threads to allow others to run.  Preemption is also useful to
application programmers in its own right, and the operating system's choice of
preemption abstraction imposes a prescribed structure and architecture on programs
seeking to use this primitive.  Specifically, exposing preemption via the scheduler
encourages programmers to treat the thread as the unit of preemption, meaning that
a program making internal use of preemption is also parallel; however, because
preemption fundamentally entails concurrency but not parallelism, such an application
may be incurring unnecessary complexity and performance penalties.  We provide an
abstraction for preemption at the granularity of a synchronous function call, and
demonstrate that this represents a more efficient and composable interface that
enables new functionality for latency-critical applications, yet is expressive enough
to encode classic asynchrony primitives.
}

\end{document}
