\documentclass[12pt,letterpaper,openright]{report}

\usepackage{amssymb}
\usepackage{booktabs}
\usepackage[T1]{fontenc}
\usepackage[margin=1in]{geometry}
\usepackage{graphicx}
\usepackage{hyperref}
\usepackage[mono=false]{libertine}
\usepackage{listings}
\usepackage{multirow}
\usepackage[numbers]{natbib}
\usepackage{subcaption}
\usepackage[svgnames]{xcolor}

\ifdefined\thisisathesis
	\newcommand{\chapquotes}{}
	\newcommand{\solb}[1]{{\color{magenta} TODO #1}}
\else
	\newcommand{\solb}[1]{}
\fi

\renewcommand{\contentsname}{Table of Contents}
\renewcommand{\lstlistlistingname}{List of Listings}

\lstset{captionpos=b,
	basicstyle=\ttfamily,
	keywordstyle=\color{Blue},
	commentstyle=\color{Green},
	columns=flexible,
	language=C++
}

\bibliographystyle{abbrvnat}

\newcommand{\ms}[1]{#1 ms}
\newcommand{\us}[1]{#1 $\mu$s}

\newcommand{\attribchapquote}{}
\newcommand{\widthchapquote}{}
\newenvironment{chapquote}[2][3in]{
	\renewcommand{\attribchapquote}{#2}
	\renewcommand{\widthchapquote}{#1}
	\vspace{-2in}
	\begin{flushright}
	{\Large ``}
}{
	{\Large ''} \\
	--- \attribchapquote \\
	\rule{\widthchapquote}{1pt}
	\end{flushright}
	\vspace{1in}
}

\makeatletter
\let\includegraphics@\includegraphics
\renewcommand{\includegraphics}[2][]{\includegraphics@[#1]{\includegraphicsdir#2}}
\newcommand{\includegraphicsdir}{}

\let\input@\input
\renewcommand{\input}[2][.]{
	\renewcommand{\includegraphicsdir}{#1/}
	\input@{#1/#2}
	\renewcommand{\includegraphicsdir}{}
}

\let\section@\section
\newenvironment{swallowsections}{
	\renewcommand{\section}[1]{}
}{
	\let\section\section@
}
\makeatother

\renewenvironment{abstract}{}{}

\begin{document}

\begin{titlepage}
\begin{center}
	\vspace*{\fill}

	\textbf{\Large Lightweight Preemptible Functions} \\
	A thesis proposal \\
	\hfill \\
	{\large Sol Boucher} \\
	\today \\

	\vspace{1in}

	\textbf{Thesis committee:} \\
	David G.\@ Andersen, \textit{chair} \\
	Adam Belay \\
	Michael Kaminsky \\
	Brandon Lucia \\

	\vspace{\fill}

	\textit{Submitted in partial fulfillment of the requirements \\
	for the degree of Doctor of Philosophy} \\
	\hfill \\
	Computer Science Department \\
	School of Computer Science \\
	Carnegie Mellon University \\
	Pittsburgh, PA 15213 \\
\end{center}
\end{titlepage}

\pagenumbering{roman}
\tableofcontents
\listoffigures
\listoftables
\lstlistoflistings
\newpage
\pagenumbering{arabic}

\ifdefined\foreword
\chapter*{Foreword}

Perhaps it is inevitable that when a nonfiction work reaches a certain length, it
begins to serve multiple purposes; if so, this document is no exception.  Yes, it is
a record of the ideas I have explored over the past years of my life.  But like any
thesis, it is also a lesson:\@ in summarizing my findings from these explorations, it
endeavors to save you from spending years of your own on the same topic.  And like
any good lesson, this one begins with an exercise...

With your permission, we will conduct a brief mindfulness activity.  At this moment,
and for however much longer you focus on this document, you mind will be occupied by
ideas.  Many of these ideas I will have put there.  Shortly, they will be ideas about
computer systems, but first let us consider:  How are the ideas getting to you?  You
are reading, but what does that mean?  Perhaps you are holding a printout or a bound
copy of this manuscript, or perhaps you have loaded it onto your personal computer,
tablet, phone, or hand terminal.  In any case, you have opened it to a particular
page, exposing your eyes to a sea of shapes arranged into nested clusters.  Your eyes
have gravitated to a cluster of particularly large shapes known as a ``chapter
title,'' then they have scanned across the page and sent a compressed representation
of each smaller ``word'' cluster of shapes to your brain, which has matched the
clusters of shapes to entries in your mental lexicon, then parsed them according to a
set of linguistic rules to infer their logical connections.  Then you have moved on
to the large ``paragraph'' clusters and processed each in turn, starting with the
first of its ``sentence'' clusters, and in so doing, learning what the next sentences
will be about.  Occasionally, something will go wrong at one of these steps and you
will backtrack and notice a missed word, or more closely examine a misidentified
word, or try a different parsing of the sentence, or reexamine the logical flow of
the paragraph.  You will usually not realize you are doing any of this, preferring
to think simply that you are \textbf{reading}.

The document you are reading is about computer systems, and like your brain, such
systems have many complexities.  If we as computer users had to describe the full
process for doing everything, we would never accomplish anything, so instead we
build \textbf{abstractions} for performing common tasks without examining the
underlying details.  Some would say that any computer systems research is
fundamentally about abstractions.  This particular work centers around an abstraction
for use by application programmers, who in turn work on top of other abstractions,
the most notable of which is software called the \textbf{operating system}.

In computing, as in life, one's fundamental goal is to accomplish some task using a
set of shared resources.  Someone must decide how to allocate these shared resources,
a role usually filled for a particular resource by a piece of software called its
\textbf{scheduler}.  One major responsibility of the operating system is to share
hardware resources such as the processor, the short- and long-term storage devices,
devices for user interaction, and network interfaces.  Among these, the one most
relevant to our discussion is the CPU scheduler, which manages the processor.

Despite itself being an abstraction that hides enormous complexity, a processor is
conceptually quite simple:\@ it receives a stream of simple \textbf{instructions}
telling it what to do, executing them in the order received and occasionally jumping
to a particular point elsewhere in the stream when so instructed.  The simplicity of
this model belies the infinite expressive power of programs constructed from such
instructions.  Indeed, programming at the instruction level is difficult not only
because the simplicity of the instructions make it verbose, but also because ad-hoc
jumps can be deceptively complicated to reason about.  Modern programmers usually
write software in programming languages that provide so-called ``structured control''
abstractions for performing common, formulaic sequences of jumps.

The most fundamental abstraction composing a structured program is the
\textbf{function}:\@ a section of code that expects zero or more input data, performs
some computation, and generates zero or more output data.  One function can call
another, which automatically transfers the input data and jumps to the start of that
function's code.  Later, when the end of its code is reached, the function
automatically jumps back to the program point just after it was called and transfers
its output data back.  Notice that a function call is \textbf{synchronous}; that is,
the function runs to completion before the calling function continues to run.
Because such sequential execution matches the processor's inherent behavior, sharing
the processor between functions is trivial and requires no scheduler.

Of course, an important feature of modern computers is the ability to work on
multiple tasks alongside each other, such as reading a document and composing an
outline or notes.  Operating systems manage such situations by providing an
abstraction called a \textbf{process}, or independent task.  Each process is isolated
from the others on the system and cannot access their data.  Furthermore, processes
exhibit a property known as \textbf{concurrency} wherein their executions can
interleave such that one process executes some of its code ``in the middle of''
another process's work.  (Think of momentarily putting your notetaking on hold to
scroll down in the document you're reading.)

Because isolation prevents processes from calling each other's functions directly,
switching between processes requires a scheduler to transfer control of the
processor.  Specifically, the processor must stop executing the running process and
start running the operating system's CPU scheduler code, which then performs an
action called a \textbf{context switch}:\@ it saves a checkpoint of that process and
restores the other process, resuming it from the state in which it last left off.
The conceptually simpler way for this transition to happen is \textbf{cooperative}
multitasking, in which the former process voluntarily gives up control of the
processor by explicitly telling the operating system to give someone else a turn.
Unfortunately, it is not safe to assume a process will eventually cede its processor,
as it may never decide to do so, through either misbehavior or malice.  Such a
scenario would render the rest of the programs unusable.

Fortunately, processors have a low-level mechanism for spontaneously changing which
instruction they are executing known as a \textbf{timer interrupt}.  Every so often,
the processor jumps into the OS scheduler from whatever code it is currently
executing.  Since it is now has the use of the processor, the scheduler can decide
whether to jump back to that same program or context switch to a different one, a
decision that is usually made based on how long the former program had been running
since the last context switch.  This style of process scheduling is known as
\textbf{preemptive} multitasking because the operating system initiates it by
actively pausing the running process.

Recent decades have seen the introduction of multicore computers that have more than
one processor, creating the opportunity for the operating system to schedule a
different process on each.  Such processes exhibit \textbf{parallelism}:\@ they
actually run at the same time.  Parallelism is also an attractive feature for
application programmers because by carefully restructuring their programs, they can
route some of their work to each processor, thereby speeding up portions of their
program's run.  Unfortunately, fitting such programs into operating systems' existing
process model was cumbersome.

To better accommodate parallel programs, operating systems introduced a hybrid
abstraction called a \textbf{thread}.  Like processes, threads can be both concurrent
and parallel.  Unlike processes, though, threads must share data to effectively
work together on a single task, so the threads within a process are not isolated from
one another.  It turns out that the simultaneous presence of concurrency and shared
data introduces fundamental challenges that make it difficult to write correct
programs due to a class of bugs informally known as race conditions.  Although safe
concurrency is a popular area of study, challenges remain particularly in systems
containing components that predate the parallel programming paradigm.  More detailed
coverage of safe concurrency and backwards compatibility as they relate to this
thesis work appears in chapters~\ref{chap:libinger} and \ref{chap:libgotcha},
respectively.

The lack of isolation between threads permits the programmer to spawn a thread in
much the same way they would call a function:\@ in most programming languages, they
place the code they want to execute on the new thread in its own function, but
instead of calling it directly, they pass it to a special wrapper function.  The
wrapper sets up the thread and begins running the programmer's custom thread thereon.
However, in an important break from functions, threads are \textbf{asynchronous} like
processes.  The wrapper function returns almost instantly, even if the thread is
still running in the background.  As with processes, this property means there must
be a scheduler to decide which application code each processor should run.  Note
that for the sake of this discussion, we are assuming this is the operating system's
CPU scheduler; however, this is not always the case and sometimes a custom scheduler
runs as part of the application itself, a configuration that is addressed in detail
in chapter~\ref{chap:libturquoise}.

Introducing additional scheduler dependencies on an application has important
functionality and performance ramifications for two fundamental reasons.  First, the
scheduler's placement behind an abstraction decouples it from the program's logic,
thereby imposing one or more levels of communication barrier that reduce its
understanding of the particular application's needs, often resulting in a brittle
policy ill suited to the workload.  For instance, few preemptive schedulers provide a
way for an application to customize the timer interrupt interval, even when supported
by the hardware.  Second, every scheduler works by running its own code to make
decisions about how to allocate a resource.  Because it does not represent useful
work from the application's perspective, time spent this way is pure overhead, and it
follows that introducing unnecessary scheduling necessarily reduces performance.

Threads represent the standard application for exploiting preemption within an
application.  However, reminiscent of how processes were cumbersome to use for
parallel programming, threads are ill suited to some use cases of preemption.  For
one thing, programmers who do not need parallelism are led to build their synchrony
atop asynchrony, thereby introducing a useless scheduler dependency.  For instance,
when calling a helper function but needing a result by a specific deadline is tempted
to spawn the function on its own thread, then immediately wait for the thread to
finish, a task better accomplished on the same thread.  Furthermore, although they
support pausing code mid-execution, threads make it very difficult to cancel
in-progress work that is no longer needed at all.

Fortunately, the tendency to use threads for application-level preemption is not
because the operating system does not expose hardware features such as timers.
Rather, it is because such features are presented as very low-level abstractions that
perform hardware-style unstructured jumps rather than using language-style structured
control and abstracting away the details of context switching.  We therefore propose
a new abstraction for easy preemption within an application, but show that it can be
implemented on top of the existing operating system.
\fi


\chapter{Introduction}

The abstraction most fundamental to modern programs is the \textbf{function}, a
section of code that expects zero or more data inputs, performs some computation, and
produces zero or more outputs.  It is a structured control flow primitive that obeys
a strict convention:\@ whenever invoked from one of its \textbf{call sites}, a
function runs from beginning to (one possible) end, at which point execution resumes
in the \textbf{caller} just after the call site.  It is also a \textbf{synchronous}
primitive; that is, all these steps happen sequentially and in order.  Because
processors conceptually implement synchronous computation, scheduling a function is
as trivial as instructing the processor to jump from the call site to its starting
address, then jump back to the (saved) address subsequent to the call site.  Thus,
the program continues executing throughout, with no inherent need for intervention by
an external scheduler or other utility software.

% TODO: Citation on decompression bombs?
Note, however, that just because the program has retained control does not mean the
programmer has.  Precisely because functions represent an abstraction, the programmer
who calls one is not necessarily familiar with its specific implementation.  This can
make it hard to predict the function's duration, yet calling it requires the
programmer to trust it to eventually finish and relinquish control.  The programmer
may have made a promise (e.g., a service-level agreement) that their whole program
will complete within a specified timeframe; unfortunately, they cannot certify their
compliance without breaking the abstraction and examining the internals of each
function they call.  Even then, untrusted or unpredictable input may make the
function's performance characteristics unclear:  Perhaps it solves a problem that is
known to be intractable for certain cases that are difficult to identify a priori.
Perhaps it performs format decoding or translation that is susceptible to attacks
such as decompression bombs.  Or perhaps it simply contains bug that opens it to
inefficient corner cases or even an infinite loop.

Faced with such problems, the programmer is often tempted to resort to an
\textbf{asynchronous} invocation strategy, whereby the function runs in the
background while the programmer maintains control of the rest of the program.  Common
abstractions include the operating system's own processes and threads, as well as
the threads, coroutines, and futures (i.e., promises) provided by some libraries and
language runtimes.  Any use of asynchronous computation requires an external
scheduler to allocate work.

Here again, the programmer is sacrificing control.  Upon handing execution control to
a scheduler, dependencies are no longer clear from the program's structure and must
be communicated to the scheduler via synchronization constructs; however, it is
difficult to fully communicate the relevant bits of the application logic across this
abstraction boundary, which can result in unintended resource-sharing effects such as
priority inversion.  Furthermore, each software scheduler is itself a piece of code,
and because this does not represent useful application work, any time it spends
executing is pure overhead.  Therefore, introducing unnecessary scheduling
necessarily reduces per-processor performance.

In many cases, the \textit{only} tool necessary to ensure timely completion of a
program is preemption.  Instead of confronting this directly, current programming
environments incentivize the programmer to rely on a scheduler to fix the problem,
limiting them to whatever coarse timescales (often milliseconds) the OS schedule
operates, or (in the case of userland schedulers) even to cooperative scheduling that
doesn't even address the problem of infinite loops.  The goal of this work is to
design and prototype an interface that extends the programming model with easy
preemption, thereby allowing the use of functions without having to break the
abstraction and examine their implementations.  If the function times out, it is
paused so that the programmer can later resume and/or cancel it at the appropriate
time.  Note that such an interface is still inherently concurrent; indeed, it is the
programmer who expresses the schedule describing when to devote time to the timed
code, and how much.


\section{Thesis statement}

\textit{Modern operating systems provide task preemption
as a resource sharing mechanism:\@ when the total number of
threads exceeds the number of processors, the kernel scheduler preempts long-running
or low-priority threads to allow others to run.  Preemption is also useful to
applications in its own right, and its interface
influences the structure and architecture of such programs.
Providing only an asynchronous interface encourages the programmer to leave even
simple scheduling to the operating system, thereby accepting the scheduler's overhead
and coarse resolution.  We introduce a novel
abstraction for preemption at the granularity of a synchronous function call, and
demonstrate that this represents a more efficient and composable interface that
enables new functionality for latency-critical applications, while being both
compatible with the existing systems stack and expressive enough
to encode classic asynchrony primitives.
}


\section{Structure}

The rest of the chapters of this thesis break down this research as follows:
\begin{itemize}
\item We prototype our abstraction on top the GNU/Linux operating system with an
	unmodified kernel (Chapter~\ref{chap:libinger}).  Although the current
	implementation limited to the x86-64 architecture and relies on POSIX
	features such as signals, timers, and contexts and GNU dynamic linker
	namespaces, we believe it could be ported to other systems and architectures
	with additional engineering effort.  We examine the performance properties of
	preemptible function invocations, pauses, resumes, and cancellations.
\item Supporting existing nonreentrant code requires an approach we call selective
	relinking; our implementation includes a lightweight runtime to do this
	transparently with modest overhead (Chapter~\ref{chap:libgotcha}).  We
	present the technique, along with its interaction with dynamic linking and
	loading, in detail and with examples of semantic implications.
\item Although lightweight preemptible functions are fundamentally synchronous, we
	demonstrate their expressiveness by applying them to the asynchrony problem
	of making an existing cooperative thread library preemptive
	(Chapter~\ref{chap:libturquoise}).  We demonstrate the applicability of this
	artifact to mitigating head-of-line blocking in a modern Web server.
\item We discuss the relevance of preemptible functions to serverless computing,
	where we argue they could be used to accelerate the launch times of
	microservices, enabling customers to better leverage the low latency of
	modern datacenter networks and the steady performance improvements of FaaS
	providers' walled garden services (Chapter~\ref{chap:microservices}).
\end{itemize}


\chapter{Function calls with timeouts}
\label{chap:libinger}

\ifdefined\chapquotes
\begin{chapquote}{David Mitchell, \textit{Cloud Atlas}}
A half-read book is a half-finished love affair.
\end{chapquote}
\fi

\begin{swallowsections}
\input[functions]{intro}
\end{swallowsections}
\input[functions]{related}
\input[functions]{inger}


\section{Evaluation}

\input[functions]{eval_testbed}


\subsection{Microbenchmarks}

\input[functions]{eval_uinger}


\subsection{Image decompression}

Unlike state of the art approaches, lightweight preemptible functions support
cancellation.
\input[functions]{eval_cancel}


\chapter{Nonreentrancy and selective relinking}
\label{chap:libgotcha}

\ifdefined\chapquotes
\begin{chapquote}{Isaac Asimov, \textit{The Gods Themselves}}
`Does everyone just believe what he wants to?' \\
`As long as possible.  Sometimes longer.'
\end{chapquote}
\fi

\begin{swallowsections}
\input[functions]{gotcha}
\end{swallowsections}


\section{Evaluation}

\input[functions]{eval_ugotcha}

\input[functions]{eval_testbed}


\chapter{Preemptive userland threading}
\label{chap:libturquoise}

\ifdefined\chapquotes
\begin{chapquote}{James S.\@ A.\@ Corey, \textit{Nemesis Games}}
`Alien superweapons were used,' Alex said, walking into the room, \\
sleep-sweaty hair standing out from his skull in every direction. \\
`The laws of physics were altered, mistakes were made.'
\end{chapquote}
\fi

\begin{swallowsections}
\input[functions]{turquoise}
\end{swallowsections}


\section{Evaluation}

\input[functions]{eval_turquoise}

\input[functions]{eval_testbed}


\chapter{Microsecond-scale microservices}
\label{chap:microservices}

\ifdefined\chapquotes
\begin{chapquote}{Douglas Adams, \textit{The Hitchhiker's Guide to the Galaxy}}
The ships hung in the sky in much the same way that bricks don't.
\end{chapquote}
\fi

\input[microservices]{abstract}
\input[microservices]{intro}
\input[microservices]{motivation}
\input[microservices]{isolation}
\input[microservices]{design}
\input[microservices]{design_howoften}
\input[microservices]{design_howclean}
\input[microservices]{eval}
\input[microservices]{future}
\input[microservices]{concl}


\chapter{Proposed work}

\begin{swallowsections}
\input[functions]{concl}
\end{swallowsections}


\section{Remaining work}

I propose extending the work already completed in all three of the following ways:

\paragraph{Cancellation resource cleanup for the Rust interface}
Although we currently support asynchronous cancellation of timed-out preemptible
functions via \textit{libinger}'s \texttt{cancel()} facility, we do not yet perform
any automatic cleanup of resources they have already allocated.  This is impossible
in general for C programs because the language lacks a destructor mechanism; however,
I intend to add at least partial support for doing so for Rust programs.  The basic
principle will be to throw an artificial exception (Rust panic) on the preemptible
function's execution stack, thereby unwinding the stack and invoking local variables'
destructors.  This approach alone will not guarantee comprehensive cleanup, because
such variables may not have been fully initialized at the time of preemption; for the
same reason, it may have safety ramifications that will merit additional study.  I
have some ideas about how to augment the technique, such as by extending
\textit{libgotcha} to keep a running cache of the several most recent allocations and
deallocations of common resources (e.g., memory and file descriptors) from each
libset.  While exhaustive cleanup may prove difficult to achieve, I hope to at least
catalogue situations where we guarantee not to leak certain classes of resource, and
perhaps provide a tunable allowing users to request full resouce tracking if needed.

\paragraph{Automatic selection and variation of timer frequency}
For simplicity, the current \textit{libinger} implementation subscribes to timer
signals a globally constant interval apart throughout the entire duration of each
preemptible function.  To improve efficiency while preserving preemption granularity,
I plan to dynamically determine this interval based on the requested timeout.  This
will likely include delaying the first of these signals until shortly before the
timeout would expire.  Maintaining accuracy across multiple CPUs will probably
require building in a calibration routine to infer configuration parameters.

\paragraph{Support OpenSSL and benchmark hyper with HTTPS}
Earlier in \textit{libgotcha}'s development history, it was able to run nginx with
OpenSSL.  Recent attempts at getting hyper to run with OpenSSL have ended in crashes,
yet this configuration is desirable for measurement because it exhibits a far greater
number of dynamic function calls, of which \textit{libgotcha} alters the performance
characteristics.  I aim to support and benchmark the configuration, which will likely
involve a regression test using the nginx setup.

\hfill \\
\noindent
Additionally, I will complete up to two of the following projects, as selected by the
committee:

\paragraph{Optimize libset reinitialization}
The high cost of cancellation is a consequence of our libset reinitialization
approach, which currently involves unloading all libraries from the libset, then
reloading them and running all their constructors in the process.  I believe cheaper
schemes are possible, such as checkpointing only the writeable portions of each
library's address space (e.g., by replacing their page mappings with copy-on-write
mappings backed by their contents immediately after return from the library's
constructors).

\paragraph{Achieve even finer--grained preemption}
The \textit{Shinjuku} authors included IPI microbenchmarks that suggest it should be
possible for us to achieve interrupt latencies and spacing within a small constant
factor of theirs.  I could explore this possibility by reusing some of their
optimizations to POSIX contexts and adding a specialized timer signal delivery
mechanism to the kernel to reduce the number of ISR instructions at the expense of
generality.

\paragraph{Benchmark against \textit{Shinjuku}}
Competing with \textit{Shinjuku} on a network benchmark would require porting
kernel-bypass networking to preemptible functions, and possibly running our system
under Dune or IX.  Sharing hardware under \textit{libgotcha} would require careful
thought because the normal approach of (library copying and selective relinking)
changes program semantics in situations where an uncopyable resource such as a
network adapter underlies the library.

\paragraph{Implement ``\textit{libingerOS}'' container framework}
Generalizing the serverless platform case study, it should be possible to generate a
utility that takes two or more position-independent executables and combines them
into a single process whose thread(s) are timeshared between the two using
preemptible functions.  Obviously, this would only provide memory isolation for
programs implemented in memory-safe languages.  This would be implemented on top of
\textit{libinger} as a sample application.

\paragraph{Implement a caching RPC framework}
The preemptible functions API naturally lends itself to problems of caching partial
computations, and one interesting case study would be a caching RPC framework.  I
imagine this working as follows:  Clients would enclose alongside each request the
timeout they were using when listening.  The server would process each request inside
its own preemptible function.  When the client timed out waiting for a response, the
server would simultaneously time out and cache the continuation.  Subsequent requests
for the same procedure with identical inputs would resume the interrupted computation
from where it left off.


\section{Timeline}

\begin{itemize}
\item 24 April 2020: ATC '20 author notification
\item 4 June 2020: ATC '20 cameraready deadline (if applicable)
\item June 2020: Start cancellation resource cleanup, fix OpenSSL support
\item July 2020: Continue cancellation resource cleanup, fix OpenSSL support
\item August 2020: Finish cancellation resource cleanup
\item September 2020: Speaking skills talk, start automatic preemption intervals
\item October 2020: Finish and evaluate automatic preemption intervals
\item November 2020: Start first additional task
\item December 2020: Finish and evaluate first additional task
\item January 2021: Start second additional task
\item February 2021: Finish and evaluate second additional task
\item March 2021: Start thesis writing, run outstanding experiments
\item April 2021: Continue thesis writing, run any final experiments
\item May 2021: Finish thesis and defend
\end{itemize}


\newpage
\addcontentsline{toc}{chapter}{Bibliography}
\bibliography{ref}

\end{document}
