\documentclass[12pt,letterpaper,openright]{report}

\usepackage{amssymb}
\usepackage{booktabs}
\usepackage[T1]{fontenc}
\usepackage[margin=1in]{geometry}
\usepackage{graphicx}
\usepackage{hyperref}
\usepackage[mono=false]{libertine}
\usepackage{listings}
\usepackage{multirow}
\usepackage{subcaption}
\usepackage[svgnames]{xcolor}

\lstset{captionpos=b,
	basicstyle=\ttfamily,
	keywordstyle=\color{Blue},
	commentstyle=\color{Green},
	columns=flexible,
	language=C++
}

\newcommand{\solb}[1]{{\color{magenta} TODO #1}}

\newcommand{\ms}[1]{#1 ms}
\newcommand{\us}[1]{#1 $\mu$s}

\makeatletter
\let\includegraphics@\includegraphics
\renewcommand{\includegraphics}[2][]{\includegraphics@[#1]{\includegraphicsdir#2}}
\newcommand{\includegraphicsdir}{}

\let\input@\input
\renewcommand{\input}[2][]{
	\renewcommand{\includegraphicsdir}{#1/}
	\input@{#1/#2}
	\renewcommand{\includegraphicsdir}{}
}
\makeatother

\renewenvironment{abstract}{\itshape}

\begin{document}

\begin{titlepage}
\begin{center}
	\vspace*{\fill}

	\textbf{\Large Lightweight Preemptible Functions} \\
	A thesis proposal \\
	\hfill \\
	{\large Sol Boucher} \\
	\today \\

	\vspace{1in}

	\textbf{Thesis committee:} \\
	David G.\@ Andersen, \textit{chair} \\
	Adam Belay \\
	Michael Kaminsky \\
	Brandon Lucia \\

	\vspace{\fill}

	\textit{Submitted in partial fulfillment of the requirements \\
	for the degree of Doctor of Philosophy} \\
	\hfill \\
	Computer Science Department \\
	School of Computer Science \\
	Carnegie Mellon University \\
	Pittsburgh, PA 15213 \\
\end{center}
\end{titlepage}

\pagenumbering{roman}
\tableofcontents
\newpage
\pagenumbering{arabic}

\chapter{Introduction}

\textit{Contemporary general-purpose operating systems provide task preemption, which
they treat as a resource sharing mechanism:\@ when the total number of application
threads exceeds the number of processors, the kernel scheduler preempts long-running
or low-priority threads to allow others to run.  Preemption is also useful to
application programmers in its own right, and the operating system's choice of
preemption abstraction imposes a prescribed structure and architecture on programs
seeking to use this primitive.  Specifically, exposing preemption via the scheduler
encourages programmers to treat the thread as the unit of preemption, meaning that
a program making internal use of preemption is also parallel; however, because
preemption fundamentally entails concurrency but not parallelism, such an application
may be incurring unnecessary complexity and performance penalties.  We provide an
abstraction for preemption at the granularity of a synchronous function call, and
demonstrate that this represents a more efficient and composable interface that
enables new functionality for latency-critical applications, yet is expressive enough
to encode classic asynchrony primitives.
}

The contributions of this thesis are as follows:
\begin{itemize}
\item We prototype our abstraction on top the GNU/Linux operating system with an
	unmodified kernel (Chapter~\ref{chap:libinger}).  Although the current
	implementation limited to the x86-64 architecture and relies on POSIX
	features such as signals, timers, and contexts and GNU dynamic linker
	namespaces, we believe it could be ported to other systems and architectures
	with additional engineering effort.  We examine the performance properties of
	preemptible function invocations, pauses, resumes, and cancellations.
\item Supporting existing nonreentrant code requires an approach we call selective
	relinking; our implementation includes a lightweight runtime to do this
	transparently with modest overhead (Chapter~\ref{chap:libgotcha}).  We
	present the technique, along with its interaction with dynamic linking and
	loading, in detail and with examples of semantic implications.
\item Although lightweight preemptible functions are fundamentally synchronous, we
	demonstrate their expressiveness by applying them to the asynchrony problem
	of making an existing cooperative thread library preemptive
	(Chapter~\ref{chap:libturquoise}).  We demonstrate the applicability of this
	artifact to mitigating head-of-line blocking in a modern Web server.
\item We discuss the relevance of preemptible functions to serverless computing,
	where we argue they could be used to accelerate the launch times of
	microservices, enabling customers to better leverage the low latency of
	modern datacenter networks and the steady performance improvements of FaaS
	providers' walled garden services (Chapter~\ref{chap:microservices}).
\end{itemize}

\input[functions]{intro}
\input[functions]{concl}


\chapter{Lightweight preemptible functions}
\label{chap:libinger}

\input[functions]{abstract}
\input[functions]{inger}
\input[functions]{related}


\chapter{Nonreentrancy and selective relinking}
\label{chap:libgotcha}

\input[functions]{gotcha}


\chapter{Preemptive userland threading}
\label{chap:libturquoise}

\input[functions]{turquoise}
\input[functions]{eval}


\chapter{Microsecond-scale microservices}
\label{chap:microservices}

\input[microservices]{abstract}
\input[microservices]{intro}
\input[microservices]{motivation}
\input[microservices]{isolation}
\input[microservices]{eval}
\input[microservices]{concl}


\chapter{Proposed work}

\section{Remaining work}

Plan on doing:
\begin{itemize}
\item Cancellation resource cleanup for the Rust interface
\item Automatic selection and variation of preemption quanta
\item Get OpenSSL working and benchmark hyper with HTTPS
\end{itemize}

\noindent
Pick 2 from these:
\begin{itemize}
\item Optimize \textit{libgotcha} libset reinitialization?
\item Implement ``\textit{libac-safe}'' and use it to benchmark cancellation against pthreads?
\item Port to Dune and/or benchmark against Shinjuku?
\item Implement request-caching RPC framework?
\item Implement ``\textit{libingerOS}'' container framework?
\end{itemize}


\section{Timeline}

\begin{itemize}
\item 24 April 2020: ATC '20 author notification
\item 4 June 2020: ATC '20 cameraready deadline (if applicable)
\item 15 July 2020: ATC '20 conference (?)
\item August 2020: Start cancellation resource cleanup
\item September 2020: Speaking skills talk?
\item October 2020: Finish cancellation resource cleanup, start automatic preemption quanta
\item November 2020: Finish and evaluate automatic preemption quanata
\item December 2020: Start first additional task
\item January 2021: Finish and evaluate first additional task
\item February 2021: Start second additional task
\item March 2021: Finish and evaluate second additional task
\item April 2021: Thesis writing
\item May 2021: Finish thesis and defend
\end{itemize}


\end{document}
