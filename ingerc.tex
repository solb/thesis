\chapter{Resource cleanup and async unwinding: \\ the \textit{ingerc} compiler}
\label{chap:ingerc}

\ifdefined\chapquotes
\vspace{-1in}
\begin{chapquote}[1.75in]{J. R. R. Tolkien, \textit{The Fellowship of the Ring}}
And with that Gl\'oin embarked on a long account of the doings of the \\
Dwarf-Kingdom.  He was delighted to have found so polite a listener...
\end{chapquote}
\fi

As described so far, one of the facilities that \textit{libinger} enables is
asynchronous function cancellation.  As we saw in Chapters~\ref{chap:functions} and
\ref{chap:safety}, this is a significant achievement that is only possible under the
POSIX safety model thanks to selective relinking.  However, one missing piece of
functionality is automatic cleanup of any resources the cancelled function had
allocated.

The resource leaks associated with cancelling a function are a significant problem:\@
they make cancellation infeasible for long-running applications, which would
experience the cumulative leakage of the resources allocated by all such cancelled
functions.  While a garbage collector would be able to find the leaked resources,
deallocating them might still prove challenging because, without a record of the
interruption point where cancellation occurred, it would not be safe to run object
finalizers.  Of course, our system targets unmanaged languages, so we must accomplish
resource cleanup without a garbage collector.


\section{Languages with unstructured resource management}

In languages such as C, resource lifetimes are completely unstructured, with each
allocation and deallocation performed via an ad-hoc function call.  Some such
functions are well-known because they are prescribed by the C and/or POSIX standards:
\texttt{malloc()}/\texttt{free()}, \texttt{open()}/\texttt{close()}, etc.  However,
applications and libraries can provide their own resource-allocation interfaces, so
it is not possible to identify or track resource management in general.  Worse, there
is no standardization of deallocation functions' interface.  These language
properties mean that automating cleanup would require hand-annotating all custom
allocation and deallocation functions throughout the application and its
dependencies; such annotations would have to provide associations between each
allocator and its corresponding deallocator, as well as information about how to call
the latter.

Were one to build a system to support this, one would need to use an approach like
that of Valgrind's Memcheck~\cite{seward:usenix2005} and LLVM's MemorySanitizer and
instrument the application's allocation and deallocation calls (for which
\textit{libgotcha}'s existing ability to intercept function calls might prove
useful!).  While the memory footprint of this technique is modest compared to that
imposed by \textit{libgotcha}, the runtime slowdown is 3x on memory-intensive
workloads~\cite{stepanov:cgo2015}.  It is probably difficult to dramatically reduce
this overhead because the necessary tracking amounts to adding potentially-expensive
bookkeeping work to each allocation and deallocation, already expensive operations
that can dominate applications' execution.  Worse, the bookkeeping structures need to
be mutable, so care must be taken to avoid designing around data structures with
amortized time complexities, as this would introduce undesirable unpredictable pauses
in preemptible function execution reminiscent of garbage collection\footnote{The
\textit{libgotcha} runtime itself does not suffer from this problem because its
function lookup tables are immutable once process initialization is complete.}.  For
instance, storing allocation records in a hash table would require periodic
rebalancing.

Because of the above limitations, we have not pursued automatic resource cleanup for
preemptible functions written in C.  We advise developers of long-running C
applications to enjoy the other benefits of lightweight preemptible functions, but to
always eventually allow their functions to run to completion.


\section{Languages following the RAII principle}

The situation is more promising in Rust.  Like C++, it adheres to the RAII (Resource
Allocation Is Initialization) idiom that associates each resource's lifetime with
that of some object.  Whenever an object goes out of scope, the program invokes its
destructor and those of its members, freeing the associated resources.  Thus, the
problem of releasing the resources associated with a cancellation can be reduced to
that of invoking the destructors of the objects that are alive at the interruption
point.  Notice that, in contrast to garbage collection, such a model does not divorce
the problem of deallocation from the cancelled function's code; as such, it is not
subject to the safety problems of invoking finalizers, as only the destructors of
objects whose initialization is already complete can be invoked.

Faced with the challenge of safely preempting in the presence of shared state caused
by nonreentrant library interfaces, we found that we could leverage dynamic linking
to solve the problem automatically, and built the \textit{libgotcha} runtime to do
just that.  Here again, we are fortunate to find an existing runtime facility that
can be repurposed to call destructors at an arbitrary position in the program:\@
the Rust language already supports exceptions (which it calls ``panics'').  One
significant advantage to building on top of constructors rather than implementing
separate resource tracking is that exception handling is already designed to add no
overhead to the non-exceptional execution path.  With the exception of adding one
cheap function call to each function that owns objects with destructors, we are able
to provide automatic cleanup with no runtime overhead.


\section{A brief tour of exception handling}

Whenever a program throws an exception, the language runtime must find the point in
the program that handles that exception.  To prevent resource leaks, deadlocks, and
other bugs, it must then invoke the destructors of all objects that are in scope at
the point where the exception was thrown, but out of scope at the point where it is
caught.  This feature of exception handling is perfectly suited to our use case.

It is possible for a function to throw an exception that is then caught by one of its
callers, so the language runtime must be able to ``unwind'' the stack, locating the
stack frame of each function's caller.  Code for the x86 architecture used to
maintain a frame pointer that made it easy to find the bounds of a function's stack
frame, but with the advent of x86-64, this is no longer standard; thus, the runtime
needs some other way to find the next frame.  Debuggers have long faced this very
problem on other architectures, and the common approach is to rely on extra debugging
information stored in the executable or library on the disk.  On Unix operating
systems, most debuggers use the CFI (Call Frame Information) facility of the standard
DWARF debugging format~\cite{eager:spec2012}.

Modern exception runtimes repurpose this debugging information to unwind the stack
once an exception has been thrown.  The compiler produces the requisite information
by generating CFI pseudoinstructions, which the assembler then transcribes into DWARF
format and stores in the \texttt{.eh\_frame} section of the object file.  This
section is present in non-debug builds and stripped object files and gets loaded into
the process's memory image by the ELF loader or dynamic linker, in contrast to the
CFI's more traditional home, the \texttt{.debug\_frame} section.  With the complexity
of this approach comes the advantage that the application no longer has to update
frame pointers during normal execution.

Call Frame Information alone is not a sufficient primitive to implement exception
handling:\@ the runtime must also be able to find the exception handler(s) present in
each call frame and the destructors to invoke based on where in the function the
exception was thrown.  The compiler must supply this information, which it does by
emitting pseudoinstructions that describe a metadata region known as the LSDA
(Language-Specific Data Area); the assembler stores this in the object file's
\texttt{.gcc\_except\_table} section.  For each function, the LSDA contains a table
mapping instruction address ranges to landing pads, code regions within the function
that serve either to catch exceptions or to invoke destructors.  Our discussion will
focus on the latter type, known as cleanup landing pads.


\section{Asynchronous exception handling}

Because exceptions are generated synchronously, they can only occur on calls to
functions that can throw.  Since compilers know which functions can throw, they
generally only output LSDA entries that are accurate for those functions' call sites.
But since \textit{libinger} interrupts functions preemptively, we need to trigger
unwinding and cleanup at whatever arbitrary point the function was paused at before
being cancelled.

Triggering unwinding is a simple matter of the runtime tweaking the stack pointer and
instruction pointer of the preemptible function to be canceled in order to forge a
call to a function that raises an exception using Rust's \texttt{panic!()} macro.
But providing instruction-accurate cleanup information requires us to address the
following challenges:
\begin{enumerate}
\item \textbf{Optimized builds remove some functions' LSDA tables and landing pads.}
We have noticed that enabling optimizations via the Rust compiler's \texttt{-O}
switch causes some functions that have exception-handling support in debug builds to
instead be compiled without it.

\item \textbf{Functions that ``return'' values via pointer parameters lack
exception-handling information.}  We have noticed that such ``sret'' functions tend
to lack any exception information at the LLVM IR level, even if they operate on
objects with destructors.  This is a problem because, although the objects exist in
the caller's stack frame, they must still be treated as owned by the function that is
``returning'' them, so that we will clean them up if cancellation occurs between the
time they are allocated and that function returns.  Such functions are more common
than one might expect and include most constructors:\@ the Rust compiler prefers to
compile functions that return large objects in this manner to avoid moving them to
the caller's stack frame immediately afterward.

\item \textbf{Many LSDA entries associate the landing pad with too few instructions
following a function call site or instructions preceding the call site.}  Injecting
an exception in such execution regions results in leaks or deallocating before
allocation, respectively.  Our investigation revealed that these discrepancies result
from changing instruction boundaries during lowering from LLVM IR to the platform's
assembly language; in particular, the backend does not account for the \texttt{mov}
and \texttt{lea} instructions that perform argument passing before most
\texttt{call}s.

\item \textbf{The runtime does not discriminate between being in the middle of
executing a function and having just retired its \texttt{ret} instruction and jumped
back to the call site.}  In either case, it will not invoke any cleanup landing pads
in the caller.  The two scenarios are indistinguishable under the assumption that no
exception can occur at these points in the function.  However, the fact that we can
inject one there creates an important distinction for our purposes:\@ until the
function returns, it still has ownership of its live variables and its landing pads
are responsible for cleaning them up, whereas after it has returns, it is impossible
for those landing pads to be invoked and cleanup must necessarily be up to the
caller.

\item \textbf{Unwinding on the first instruction of a function fails because it looks
at the LSDA table for the function whose definition precedes the actual one in
memory.}  This issue turns out to have the same cause as the previous one, but the
two situations demand different solutions.

\item \textbf{If attempted on the instruction just after one that repositions the
stack pointer, unwinding miscalculates the frame address.}  While this behavior
appears consistent between the libgcc and libunwind (LLVM) unwind implementations, we
suspect it exists because exceptions ordinarily never occur in the prologue or
epilogue of the function.  GCC has an \texttt{-fasynchronous-unwind-tables} switch
that is intended to make the frame information accurate down to the instruction, but
Clang only includes this switch for command-line compatibility and doesn't actually
implement this feature.  As a likely consequence of this lack of support from the
LLVM project itself, the Rust compiler also makes no attempt to offer it.

\item \textbf{Cleanup landing pads do not work reliably if associated with the
function epilogue.}  This happens because the epilogue adjusts the stack pointer,
in many cases causing any synthetic function call (e.g., to inject an exception) to
clobber the very stack values the landing pad is trying to clean up.  Incidentally,
a Google search for ``LLVM unwind function epilogue'' reveals that the unwind info is
not trustworthy during the epilogue in the general case.  Indeed, there have been
several patchsets attempting to fix this, some of which were merged, but each of
which was subsequently reverted for breaking some other architecture.  So it would
appear not only that this is the primary design issue blocking LLVM support for
asynchronous unwind tables, but also that we must avoid injecting exceptions in
epilogues altogether.

\end{enumerate}

Rather than integrate a fully general resource cleanup solution into
\textit{libinger}, we have prototyped the components to solve these problems and used
these to build a proof of concept implementation of the compiler transformations
necessary to support asynchronous exception handling.  In doing so, we have convinced
ourselves that our approach is feasible, although it would take additional
engineering effort to get the implementation to a point where it worked on nontrivial
applications.

The below numbered sections describe our approach to solving each of the
challenges listed above.


\subsection{Skipping optimization passes that remove exception handling}

Testing with rustc 1.56.0, we have found that the \texttt{prune-eh},
\texttt{function-attrs}, and \texttt{inline} LLVM optimization passes are
responsible for stripping the LSDA tables and landing pads from some functions in
optimized builds.  We have developed a shell script to invoke rustc without these
passes, a task that is unfortunately complicated by the compiler's command-line
interface, which only accepts a list of all the passes to run.

We recognize that disabling the inlining pass is likely to reduce the efficiency of
compiled code, but we leave it to future work to investigate why this pass is
removing exception information from even functions otherwise unaffected by inlining.


\subsection{Adding exception-handling support to functions' LLVM IR}
\label{sec:ingerc:optimization}

The above script does not address functions for which the compiler emits no
exception-handling information even in debug builds.  As before, this problem is
easiest to address in the intermediate representation, where the addition of an
exception-handling personality and a \texttt{landingpad} instruction will cause the
LLVM backend to emit an LSDA table and landing pad for the function.

To reduce implementation complexity, we do not attempt to detect which functions own
objects with destructors and instead introduce exception handling into any functions
that do not already have them.  This saves us from having to query complex properties
of the IR and reduces our task to one of simple text transformations.  We implement
these in a TypeScript script performing regular expression replacements.

The landing pads we insert at this stage are empty skeletons that do not actually
invoke any destructors.  We describe how we identify which destructor(s) to invoke,
if any, and add the calls at the end of Section~\ref{sec:ingerc:codegen}.


\subsection{Adjusting LSDA entries}
\label{sec:ingerc:codegen}

The possibility that we inject an exception during the argument-passing instructions
preceding a call violates a design assumption of LLVM's LSDA generation.  IR
instructions such as \texttt{call} often expand to multiple machine instructions,
most commonly to perform argument passing before the function dispatch.  However, the
backend generates the address ranges for LSDA entries using labels in the IR.  This
means that ordinary optimization and transform passes cannot associate landing pads
with some but not all of a function call sequence.

To get around this problem, we had to implement a plugin that loads a code generation
pass into \texttt{llc}, the LLVM static compiler.  The pass works at the x86-64
machine instruction level to reposition LSDA-related labels to resize the code region
on the normal execution path with which a cleanup landing pad is associated.  To
prevent leaks, if the ending label falls before a destructor call, we move it downward
to just before the machine \texttt{call} instruction; otherwise, we move it downward
to just before the function epilogue.  To prevent issuing destructor calls before
construction, if the starting label falls before the function call that produces the
object to be cleaned up, we move it downward to just after that call.

As part of this pass, we also identify functions with parameters annotated as
\texttt{sret} in the LLVM IR.  These correspond to functions where the script from
Section~\ref{sec:ingerc:optimization} added synthetic landing pads.  We check to see
whether the involved type(s) have destructors; if so, we add destructor calls to the
landing pad.
