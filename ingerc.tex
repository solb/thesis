\chapter{Resource cleanup and async unwinding: \\ the \textit{ingerc} compiler}
\label{chap:ingerc}

\ifdefined\chapquotes
\vspace{-1in}
\begin{chapquote}[1.75in]{J. R. R. Tolkien, \textit{The Fellowship of the Ring}}
And with that Gl\'oin embarked on a long account of the doings of the \\
Dwarf-Kingdom.  He was delighted to have found so polite a listener...
\end{chapquote}
\fi

As described so far, one of the facilities that \textit{libinger} enables is
asynchronous function cancellation.  As we saw in Chapters~\ref{chap:functions} and
\ref{chap:safety}, this is a significant achievement that is only possible under the
POSIX safety model thanks to selective relinking.  However, one missing piece of
functionality is automatic cleanup of any resources the cancelled function had
allocated.

The resource leaks associated with cancelling a function are a significant problem:\@
they make cancellation infeasible for long-running applications, which would
experience the cumulative leakage of the resources allocated by all such cancelled
functions.  While a garbage collector would be able to find the leaked resources,
deallocating them might still prove challenging because, without a record of the
interruption point where cancellation occurred, it would not be safe to run object
finalizers.  Of course, our system targets unmanaged languages, so we must accomplish
resource cleanup without a garbage collector.


\section{Languages with unstructured resource management}

In languages such as C, resource lifetimes are completely unstructured, with each
allocation and deallocation performed via an ad-hoc function call.  Some such
functions are well-known because they are prescribed by the C and/or POSIX standards:
\texttt{malloc()}/\texttt{free()}, \texttt{open()}/\texttt{close()}, etc.  However,
applications and libraries can provide their own resource-allocation interfaces, so
it is not possible to identify or track resource management in general.  Worse, there
is no standardization of deallocation functions' interface.  These language
properties mean that automating cleanup would require hand-annotating all custom
allocation and deallocation functions throughout the application and its
dependencies; such annotations would have to provide associations between each
allocator and its corresponding deallocator, as well as information about how to call
the latter.

Were one to build a system to support this, one would need to use an approach like
that of Valgrind's Memcheck~\cite{seward:usenix2005} or LLVM's
AddressSanitizer~\cite{serebryany:usenix2012}, which
instrument the application's allocation and deallocation calls.  Neither system could
be imported wholesale:  Both assume at a design level that memory is the only
resource whose allocations are being tracked.  Valgrind depends on expensive dynamic
instruction translation that is not suitable for production use.  AddressSanitizer
does not track how each resource was allocated unless paired with the separate
MemorySanitizer system~\cite{stepanov:cgo2015} run in origin-tracking mode; this adds
another 2--7x slowdown on top of AddressSanitizer's own 2x execution time penalty.

For rolling one's own allocation and deallocation tracker, \textit{libgotcha}'s
existing ability to intercept function calls might prove useful.  The bookkeeping
structures would need to be mutable, so care would have to be taken to avoid
designing around data structures with
amortized time complexities, as this would introduce undesirable unpredictable pauses
in preemptible function execution reminiscent of garbage collection.\footnote{The
\textit{libgotcha} runtime itself does not suffer from this problem because its
symbol lookup tables are immutable once process initialization is complete.}  For
instance, storing allocation records in a hash table would require periodic
rebalancing.

Because of the above limitations, we have not pursued automatic resource cleanup for
preemptible functions written in C.  Developers of long-running C applications should
always write a cleanup handler for each preemptible function they might need to
cancel (Section~\ref{sec:libinger:cancellation}).


\section{Languages following the RAII principle}
\label{sec:ingerc:raii}

The situation is more promising in Rust.  Like C++, it adheres to the RAII (Resource
Allocation Is Initialization) idiom that associates each resource's lifetime with
that of some object.  Whenever an object goes out of scope, the program invokes its
destructor and those of its members, freeing the associated resources.  Thus, the
problem of releasing the resources associated with a cancellation can be reduced to
that of invoking the destructors of the objects that are alive at the interruption
point.  Notice that, in contrast to garbage collection, such a model does not divorce
the problem of deallocation from the cancelled function's code; as such, it is not
subject to the safety problems of invoking finalizers, as only the destructors of
objects whose initialization is already complete can be invoked.

Faced with the challenge of safely preempting in the presence of shared state caused
by nonreentrant library interfaces, we found that we could leverage dynamic linking
to solve the problem automatically, and built the \textit{libgotcha} runtime to do
just that.  Here again, we are fortunate to find an existing runtime facility that
can be repurposed to call destructors at an arbitrary position in the program:\@
the Rust language already supports exceptions (which it calls ``panics'').  One
significant advantage to building on top of exceptions rather than implementing
separate resource tracking is that exception handling is already designed to add no
overhead to the non-exceptional execution path.  With the exception of adding one
function call to each function that owns objects with destructors, we believe it is
possible to provide automatic cleanup with no runtime overhead.


\section{A brief tour of exception handling}

Whenever a program throws an exception, the language runtime must find the point in
the program that handles that exception.  To prevent resource leaks, deadlocks, and
other bugs, it must then invoke the destructors of all objects that are in scope at
the point where the exception was thrown, but out of scope at the point where it is
caught.  This feature of exception handling is perfectly suited to our use case.

It is possible for a function to throw an exception that is then caught by one of its
callers, so the language runtime must be able to ``unwind'' the stack, locating the
stack frame of each function's caller.  Code for the x86 architecture used to
maintain a frame pointer that made it easy to find the bounds of a function's stack
frame, but with the advent of x86-64, this is no longer standard; thus, the runtime
needs some other way to find the next frame.  Debuggers have long faced this very
problem on other architectures, and the common approach is to rely on extra debugging
information stored in the executable or library on the disk.  On Unix operating
systems, most debuggers use the CFI (Call Frame Information) facility of the standard
DWARF debugging format~\cite{eager:spec2012}.

Modern exception runtimes repurpose this debugging information to unwind the stack
once an exception has been thrown.  The compiler produces the requisite information
by generating CFI pseudoinstructions, which the assembler then transcribes into DWARF
format and stores in the \texttt{.eh\_frame} section of the object file.  This
section is present in non-debug builds and stripped object files and gets loaded into
the process's memory image by the ELF loader or dynamic linker, in contrast to the
CFI's more traditional home, the \texttt{.debug\_frame} section.  With the complexity
of this approach comes the advantage that the application no longer has to update
frame pointers during normal execution.

Call Frame Information alone is not a sufficient primitive to implement exception
handling:\@ the runtime must also be able to find the exception handler(s) present in
each call frame and the destructors to invoke based on where in the function the
exception was thrown.  The compiler must supply this information, which it does by
emitting pseudoinstructions that describe a metadata region known as the LSDA
(Language-Specific Data Area); the assembler stores this in the object file's
\texttt{.gcc\_except\_table} section.  For each function, the LSDA contains a table
mapping instruction address ranges to landing pads, code regions within the function
that serve either to catch exceptions or to invoke destructors.  Our discussion will
focus on the latter type, known as cleanup landing pads.


\section{Asynchronous exception handling}
\label{sec:ingerc:async}

Because exceptions are generated synchronously, they can only occur on calls to
functions that can throw.  Since compilers know which functions can throw, they
generally only output LSDA entries that are accurate for those functions' call sites.
But since \textit{libinger} interrupts functions preemptively, we need to trigger
unwinding and cleanup at whatever arbitrary point the function was paused at before
being cancelled.

Triggering unwinding is a simple matter of tweaking the stack pointer and
instruction pointer of the preemptible function to be cancelled in order to forge a
call to a function that raises an exception using Rust's \texttt{panic!()} macro.
But providing instruction-accurate cleanup information requires us to address the
following challenges:
\begin{enumerate}
\item \textbf{Optimized builds remove some functions' LSDA tables and landing pads.}
We have noticed that enabling optimizations via the Rust compiler's \texttt{-O}
switch causes some functions that have exception-handling support in debug builds to
instead be compiled without it.  We describe our workaround for this issue in
Section~\ref{sec:ingerc:skip}.

\item \textbf{Functions that ``return'' values via pointer parameters lack
exception-handling information.}  We have noticed that such ``sret'' functions tend
to lack any exception information at the LLVM IR level, even if they operate on
objects with destructors.  This is a problem because, although the objects exist in
the caller's stack frame, they must still be treated as owned by the function that is
``returning'' them, so that we will clean them up if cancellation occurs between the
time they are allocated and that function returns.  Such functions are more common
than one might expect and include most constructors:\@ the Rust compiler prefers to
compile functions that return large objects in this manner to avoid moving them to
the caller's stack frame immediately afterward.  See
Section~\ref{sec:ingerc:optimization}.

\item \textbf{Many LSDA entries associate the landing pad with too few instructions
following or preceding a function call site.}  Injecting
an exception in such execution regions results in leaks or deallocating before
allocation, respectively.  Our investigation revealed that these discrepancies result
from changing instruction boundaries during lowering from LLVM IR to the platform's
assembly language; in particular, the backend does not account for the \texttt{mov}
and \texttt{lea} instructions that perform argument passing before most
\texttt{call}s. See Section~\ref{sec:ingerc:codegen}.

\item \textbf{The runtime does not discriminate between being in the middle of
executing a function and having just retired its \texttt{ret} instruction and jumped
back to the call site.}  In either case, it will not invoke any cleanup landing pads
in the caller.  The two scenarios are indistinguishable under the assumption that no
exception can occur at these points in the function.  However, the fact that we can
inject one there creates an important distinction for our purposes:\@ until the
function returns, it still has ownership of its live variables and its landing pads
are responsible for cleaning them up, whereas after it has returned, it is impossible
for those landing pads to be invoked and cleanup must necessarily be up to the
caller.  See Section~\ref{sec:ingerc:return}.

\item \textbf{Unwinding on the first instruction of a function fails because the
runtime consults the LSDA table for the function whose definition precedes it in
memory.}  This issue turns out to have the same cause as the previous one, but the
two situations demand different solutions.  See Section~\ref{sec:ingerc:start}.

\item \textbf{Performing function calls during the prologue or epilogue of a function
is unsafe.}  The x86-64 ABI (Application Binary Interface) specifies that the stack
must always be 16-byte aligned before calling a function, which is not true until the
function has reserved space for an odd number of 8-byte values (excluding the return
address) in its stack frame.  See Section~\ref{sec:ingerc:realign}.

\item \textbf{If attempted on the instruction just after one that repositions the
stack pointer, unwinding miscalculates the frame address.}  While this behavior
appears consistent between the libgcc and libunwind (LLVM) unwind implementations, we
suspect it exists because exceptions ordinarily never occur in the prologue or
epilogue of the function.  GCC has an \texttt{-fasynchronous-unwind-tables} switch
that is intended to make the frame information accurate down to the instruction, but
Clang only includes this switch for command-line compatibility and doesn't actually
implement this feature.  As a likely consequence of this lack of support from the
LLVM project itself, the Rust compiler also makes no attempt to offer it.  See
Section~\ref{sec:ingerc:boundary}.

\item \textbf{Cleanup landing pads do not work reliably if associated with the
function epilogue.}  This happens because the epilogue adjusts the stack pointer,
in many cases causing any synthetic function call (e.g., to inject an exception) to
clobber the very stack values the landing pad is trying to clean up.  Incidentally,
a Web search for ``LLVM unwind function epilogue'' reveals that the unwind info is
not trustworthy during the epilogue in the general case.  Indeed, there have been
several patchsets attempting to fix this, some of which were merged, but each of
which was subsequently reverted for breaking some other architecture.  So it would
appear not only that this is the primary design issue blocking LLVM support for
asynchronous unwind tables, but also that we must avoid injecting exceptions in
epilogues altogether.  See Section~\ref{sec:ingerc:epilogue}.

\end{enumerate}

Rather than integrate a fully general resource cleanup solution into
\textit{libinger}, we have prototyped the components to solve these problems and used
these to build a proof of concept implementation of the compiler transformations
necessary to support asynchronous exception handling.  This prototype represents
preliminary evidence that our approach is feasible, although it would take additional
engineering effort to achieve compatibility with nontrivial applications.

The below numbered sections describe our approach to solving each of the
challenges listed above.  The product of the work described in this section is a
shell script, \textit{ingerc}, that wraps rustc and applies all the described
transformations to produce an output program or library ready for runtime-assisted
cancellation cleanup.


\subsection{Skipping optimization passes that remove exception handling}
\label{sec:ingerc:skip}

Testing with rustc 1.56.0, we have found that the \texttt{prune-eh},
\texttt{function-attrs}, and \texttt{inline} LLVM optimization passes are
responsible for stripping the LSDA tables and landing pads from some functions in
optimized builds.  We have developed a shell script to invoke rustc without these
passes, a task that is unfortunately complicated by the compiler's command-line
interface, which only accepts a list of all the passes to run.

We recognize that disabling the inlining pass is likely to reduce the efficiency of
compiled code, but we leave it to future work to investigate why this pass is
removing exception information from even functions otherwise unaffected by inlining.


\subsection{Adding exception-handling support to functions' LLVM IR}
\label{sec:ingerc:optimization}

The above script does not address functions for which the compiler emits no
exception-handling information even in debug builds.  As before, this problem is
easiest to address in the intermediate representation, where the addition of an
exception-handling personality and a \texttt{landingpad} instruction will cause the
LLVM backend to emit an LSDA table and landing pad for the function.

To reduce implementation complexity, we do not attempt to detect which functions own
objects with destructors, and instead introduce exception handling into any functions
that do not already have it.  This saves us from having to query complex properties
of the IR and reduces our task to one of simple text transformations.  We implement
these in a TypeScript script performing regular expression replacements.

The landing pads we insert at this stage are empty skeletons that do not actually
invoke any destructors.  We describe how we identify which destructor(s) to invoke
(if any) and add the calls at the end of Section~\ref{sec:ingerc:codegen}.


\subsection{Adjusting LSDA entries}
\label{sec:ingerc:codegen}

The possibility that we inject an exception during the argument-passing instructions
preceding a call violates a design assumption of LLVM's LSDA generation.  IR
instructions such as \texttt{call} often expand to multiple machine instructions,
most commonly to perform argument passing before the function dispatch.  However, the
backend generates the address ranges for LSDA entries using labels in the IR.  This
means that ordinary optimization and transform passes cannot associate landing pads
with some but not all of the instructions comprising a function call sequence.

To get around this problem, we had to implement a plugin that loads a code generation
pass into \texttt{llc}, the LLVM static compiler.  The pass works at the x86-64
machine instruction level to reposition LSDA-related labels and resize the code
region
on the normal execution path with which a cleanup landing pad is associated.  To
prevent leaks, if the ending label falls before a destructor call, we move it downward
to just before the machine \texttt{call} instruction; otherwise, we move it downward
to just before the function epilogue.  To prevent issuing destructor calls before
construction, if the starting label falls before the function call that produces the
object to be cleaned up, we move it downward to just after that call.

As part of this pass, we also identify functions with parameters annotated as
\texttt{sret} in the LLVM IR.  These correspond to functions where the script from
Section~\ref{sec:ingerc:optimization} added synthetic landing pads.  We check to see
whether the involved type(s) have destructors; if so, we add destructor calls to the
landing pad.


\subsection{Detecting whether a function has just returned}
\label{sec:ingerc:return}

Regardless of whether a function has returned, LLVM's libunwind treats the caller
frame as sitting within the call instruction, rather than on the subsequent
instruction located at the return address.  Here is the offending libunwind code:
\begin{lstlisting}
  // If the last line of a function is a "throw" the compiler sometimes
  // emits no instructions after the call to __cxa_throw.  This means
  // the return address is actually the start of the next function.
  // To disambiguate this, back up the pc when we know it is a return
  // address.
  if (isReturnAddress)
  	--pc;
\end{lstlisting}

Since we inject the exception by forging a function call, libunwind always assumes
the frame where we did this is a return address and performs the decrement.  The
obvious workaround would be to remove this code from libunwind, at the cost of
potentially breaking unwinding through C++ code that might be present in the program.
Indeed, a glance through the disassembly of the Rust standard library shows that
rustc emits \texttt{ud2} (invalid) instructions following the call sites in the
scenario described in the comment, so Rust code is unaffected by the problem.

However, it turns out that the above code has another important effect beyond that
documented in the comment:\@ it avoids running the cleanup landing pad associated
with the code region following the call site if the called function was still
executing at the time the exception was thrown.  This is essential because in this
case, the called function still has ownership over any objects requiring cleanup, and
their state is undefined from the perspective of the caller.  The safe and correct
thing for the runtime to do is to invoke the called function's landing pad but not
the caller's.

For this reason, we need to override this libunwind behavior only at the instruction
where we injected the exception, and only if that instruction happens to be the
instruction following a call (so that libunwind would confuse the situation with one
where the called function was still executing).  We can accomplish that by applying
a heuristic-based tweak just before the \textit{libinger} cancellation code injects
the exception call:\@ if the instruction pointer is equal to the 8-byte value offset
--8 bytes from the stack pointer, the function return just completed and we should
add one to the instruction pointer to counteract the described libunwind behavior for
this stack frame only.


\subsection{Unwinding from the first instruction of a function}
\label{sec:ingerc:start}

Another consequence of the libunwind implementation detail described in
Section~\ref{sec:ingerc:return} is that unwinding with the instruction pointer
positioned on the first instruction of a function results in an address associated
with the preceding function in memory.  Therefore, the runtime does not find the
correct LSDA for the function (if it finds one at all).

It is hard to detect this problem without consulting the LSDA, so we implement a fix
by patching libunwind, which already decodes this information.  We insert a check
whether the current instruction pointer falls at the very beginning of its function;
if so, we set the \texttt{isReturnAddress} flag to skip the instruction pointer
adjustment.


\subsection{Calling functions when the stack is misaligned}
\label{sec:ingerc:realign}

Our standard cancellation response of forging a call to a function that panics
causes crashes when the stack is misaligned, as during a function
prologue.  Fortunately, we can easily solve this by instead selectively calling a
function that does not allocate any space in its own stack frame.  This has the
effect of restoring the stack alignment (because of the return address pushed by the
\texttt{call} instruction) before calling any complex code that relies on alignment.
Crucially, it does so without introducing any invalid stack frames that would break
unwinding.  The following function suffices:
\begin{lstlisting}[language={[x86masm]Assembler},morekeywords=ud2]
  	.globl realign
  realign:
  	.cfi_startproc
  	call panic
  	ud2
  	.cfi_endproc
\end{lstlisting}

\noindent
(where \texttt{panic} is the function that would ordinarily inject the exception).


\subsection{Unwinding after an instruction that moves the stack pointer}
\label{sec:ingerc:boundary}

To compute the offset from the stack pointer to the return address, libunwind
contains a function called \texttt{parseFDEInstructions()}.  It loops through the CFI
instructions in the \texttt{.eh\_frame} section, continuing as long as
\texttt{codeOffset < pcoffset} to process the stack pointer adjustments for the
instructions that have executed so far.  Unfortunately, this appears to fall one CFI
instruction short when the instruction that has just retired repositioned the stack
pointer.  Changing the \texttt{<} to a \texttt{<=} fixes the problem.

Section 6.4.3 of the DWARF specification~\cite{www-dwarf-spec} seems to agree with
this sign change.  We hypothesize that libunwind inherited this off-by-one error from
libgcc in its effort to replicate the older library's behavior.  The libunwind test
suite continues to pass after making the change, suggesting that a noncomprehensive
test suite has
allowed the mistake to avoid detection.  Furthermore, Clang's lack of support for
asynchronous unwind tables has probably prevented the community from encountering the
unwind failures we have.


\subsection{Unwinding in the epilogue of a function}
\label{sec:ingerc:epilogue}

As discussed in Section~\ref{sec:ingerc:async}, function epilogues are perilous for
exception injection, and even unwinding in them is currently unreliable.  To work
around these limitations, we have developed our own compiler-assisted runtime
component.

That epilogues pop values off the stack might suggest that it is no longer possible
to clean up a function's resources once its epilogue has started executing;
fortunately, they have an important property that refutes this intuition.  Although
the epilogue removes elements such as saved register values from the stack, it only
moves the stack pointer and does not overwrite the contents of the stack frame.
Thus, if we could undo the epilogue's effects, we could then inject an exception and
it would be handled as if the epilogue had never executed at all.  This is precisely
our approach.

To support time traveling backward to just before a function's epilogue, we need the
program to record its instruction pointer just before it enters the epilogue.  We
accomplish this by having the script introduced in
Section~\ref{sec:ingerc:optimization} and our LLVM pass cooperate to insert a call to
a custom function, \texttt{ingerc\_epilogue\_start()}, before each function's
epilogue(s).

In addition to saving the instruction pointer, \texttt{ingerc\_epilogue\_start()}
informs \textit{libinger} that an epilogue is currently running, so that cancelling a
preemptible function will not inject an exception in the usual way.  This means that
we must also be able to inform \textit{libinger} once the epilogue is finished, so
\texttt{ingerc\_epilogue\_start()} overwrites the function's return address with the
location of another function, \texttt{ingerc\_epilogue\_end()}, that performs this
notification before returning to the real return address.

\begin{sloppypar}
There are a few other values we need to save before starting the epilogue:  (1)~When
\texttt{ingerc\_epilogue\_end()} runs, it will need to know the function's original
return address.  (2)~If either of the functions we introduce at the beginning and end
of the epilogue is running when the function is cancelled, the stack pointer will be
different than at the point we intend to travel to, so we always store the original
stack pointer as well.  (3)~To restore the return address in
\texttt{ingerc\_epilogue\_end()} or when time traveling, we need the frame pointer.
The frame address is not normally accessible at runtime, so
we have our LLVM plugin insert code to pass it to \texttt{ingerc\_epilogue\_begin()}.
\end{sloppypar}

The functions we introduce run just before the function returns, so they must not
overwrite any return registers.  Because of this, we have hand coded them in
assembly.  We give their implementations in Listing~\ref{lst:epilogue}.  The
\texttt{ingerc\_epilogue\_begin()} function saves all values into globals so they are
accessible by the runtime.  This allows us to use the stored return address
(\texttt{ingerc\_epilogue\_ra}) to notify the runtime that the epilogue is currently
executing, so we are careful to set that last and reset it to null in
\texttt{ingerc\_epilogue\_end()}.

\begin{figure}
\begin{lstlisting}[label=lst:epilogue,language={[x86masm]Assembler},caption=Code to support time travel out of the epilogue. The \texttt{@gotpcrel} relocations are position-independent GOT lookups of the globals' addresses.]
	.globl	ingerc_epilogue_start
ingerc_epilogue_start:
	# Save our return address as the destination instruction pointer.
	mov	ingerc_epilogue_ip@gotpcrel(%rip), %rsi
	mov	(%rsp), %rcx
	mov	%rcx, (%rsi)
	# Save the stack pointer as it was before we were called.
	mov	ingerc_epilogue_sp@gotpcrel(%rip), %rsi
	lea	8(%rsp), %rcx
	mov	%rcx, (%rsi)
	# Save the frame pointer, which we received as an argument.
	mov	ingerc_epilogue_fp@gotpcrel(%rip), %rsi
	mov	%rdi, %rcx
	mov	%rcx, (%rsi)
	# Save the return address of our caller.
	mov	ingerc_epilogue_ra@gotpcrel(%rip), %rsi
	mov	(%rdi), %rcx
	mov	%rcx, (%rsi)
	# Make our caller return to ingerc_epilogue_end().
	mov	ingerc_epilogue_end@gotpcrel(%rip), %rsi
	mov	%rsi, (%rdi)
	# Return.
	ret

	.globl	ingerc_epilogue_end
ingerc_epilogue_end:
	# Put the original return address on the stack.
	mov	ingerc_epilogue_ra@gotpcrel(%rip), %rsi
	mov	(%rsi), %rdi
	push	%rdi
	# Clear the saved return address.
	xor	%rcx, %rcx
	mov	%rcx, (%rsi)
	# Return to the original caller.
	ret
\end{lstlisting}
\end{figure}

\begin{sloppypar}
There is one other thing that these functions have to be careful about:\@
if cancellation occurs while they are running but outside the region where
\texttt{ingerc\_epilogue\_ra} is set, unwinding must be safe and invoke the correct
cleanup landing pads.  This is why \texttt{ingerc\_epilogue\_end()} returns to the
real caller using a \texttt{push} and a \texttt{ret} instead of an unconditional
branch.  The \texttt{ingerc\_epilogue\_start()} implementation is compatible with
ordinary unwinding, but its invocation can be troublesome because it is
intentionally the last instruction before the epilogue.  This would mean that cleanup
in the caller frame would invoke the epilogue's landing pad, but it never has one.
To prevent a leak in this situation, our LLVM plugin inserts a \texttt{nop}
instruction after the call and before the epilogue label, in order to associate the
function's return address with the landing pad for the
preceding basic block.
\end{sloppypar}


\section{Preemptible function cancellation}
\label{sec:ingerc:cancellation}

While we have not integrated resource cleanup support into \textit{libinger}, our
work on asynchronous exception handling suggests a design.  We have prototyped the
approach in isolation using a set of scripts that use GDB to interrupt execution
after any arbitrary number of instructions have retired and inject an exception at
that point.  In this section, we give the algorithm and how it would integrate with
the existing \textit{libinger} codebase.  We conclude by reasoning about its
correctness based on where the preemptible function is in its execution at the time
it is cancelled and discussing performance considerations.

\begin{sloppypar}
Section~\ref{sec:ingerc:async} introduced our fundamental approach to asynchronous
cleanup:\@ the runtime should inject a synthetic exception at an arbitrary point in
the preemptible function.  It also presented a compiler wrapper script,
\textit{ingerc}, that applies a series of transformations to the code to make this
safe and correct.  In this section, we assume that the preemptible functions being
cancelled are written in Rust, and that the application and all its Rust dependencies
have been compiled with \textit{ingerc} instead of rustc.  We believe the latter
requirement is reasonable because the Cargo build system already expects to have the
source of all dependencies available.  Indeed, new languages such as Rust and Go
follow a growing trend of having unstable ABIs that preclude linking against
precompiled build artifacts generated by a different compiler version.
\end{sloppypar}

Whereas the \textit{libinger} C bindings implement the \texttt{cancel()} operation as
a standalone function, the Rust interface performs cancellation in the destructor.
Whenever a paused preemptible function goes out of scope, the destructor notices that
it has not run to completion and reinitializes its libset to prepare it for reuse.
Listing~\ref{lst:cleanup} gives pseudocode for a function that the destructor would
call right before this reinitialization to clean up the preemptible function's own
resources.  This works because, for safety, \texttt{resume()} catches all panics
before they can cross the FFI (Foreign Function Interface) boundary
(Section~\ref{sec:libinger:jumps}).  To prevent the
Rust runtime from outputting a diagnostic message when the panic occurs, it is
advisable to first disable the Rust standard library's panic handler in the
preemptible function's libset.  The standard library exposes the
\texttt{panic::set\_hook()} function for doing this.

\begin{figure}
\begin{lstlisting}[label=lst:cleanup,escapeinside={(*}{*)},caption=Resource cleanup for cancelled preemptible functions (pseudocode)]
function cleanup(linger_t func):
	ucontext_t snapshot = func.continuation;

	// Check whether some callee just returned (section (*{\color{Green} \ref{sec:ingerc:return}}*))
	uint64_t retaddr = 8 bytes preceding snapshot.uc_mcontext[REG_RSP]
	if snapshot.uc_mcontext[REG_RIP] == retaddr:
		increment snapshot.uc_mcontext[REG_RIP]

	// If in epilogue, time travel to before (section (*{\color{Green} \ref{sec:ingerc:epilogue}}*))
	if ingerc_epilogue_ra != NULL:
		snapshot.uc_mcontext[REG_RIP] = ingerc_epilogue_ip
		snapshot.uc_mcontext[REG_RSP] = ingerc_epilogue_sp
		location ingerc_epilogue_fp = ingerc_epilogue_ra
		ingerc_epilogue_ra = NULL

	if 16 divides snapshot.uc_mcontext[REG_RSP]:
		// Inject an exception using panic!() (section (*{\color{Green} \ref{sec:ingerc:async}}*))
		snapshot.uc_mcontext[REG_RIP] = panic
	else:
		// Realign the stack and inject exception (section (*{\color{Green} \ref{sec:ingerc:realign}}*))
		snapshot.uc_mcontext[REG_RIP] = realign

	// Throw the exception and let the cleanup landing pads run
	resume(func, UNLIMITED_TIME)
\end{lstlisting}
\end{figure}

Table~\ref{tab:cleanup} summarizes our method of resource cleanup, showing the
actions taken by our proposed runtime at each possible point the preemptible function
(or any of its callees) might be paused when cancellation occurs.  It can be seen
that we have handled all possible points within the body of an ordinary function.
The case we have scoped out of our investigation is cancelling a preemptible function
while it is running a destructor; instead of attempting this, we suggest implementing
a mechanism to detect this case (e.g., unwinding the stack or hooking into the Rust
standard library) and using a return address trick similar to that from
Section~\ref{sec:ingerc:epilogue} to delay resource cleanup until the destructor has
finished.

\begin{table}
\begin{center}
\begin{tabular}{c | p{0.175\textwidth} p{0.15\textwidth} c c}
Position within && Possible to & \multicolumn{2}{c}{Handling} \\
running function & Indicator & unwind here? & Normal execution & When cancelling \\
\hline
First instruction & Stack alignment & No (stack misalignment) & -- & \textsection~\ref{sec:ingerc:start}, \ref{sec:ingerc:realign} \\
Function prologue & Stack alignment & No (stack misalignment) & -- & \textsection~\ref{sec:ingerc:realign} \\
Function body & -- & Yes & -- & Raise exception \\
After call site & Return address & Yes (but leaks) & -- & \textsection~\ref{sec:ingerc:return} \\
Before epilogue & -- & Yes & \textsection~\ref{sec:ingerc:epilogue} & Raise exception \\
Function epilogue & Saved return address & No (calls could clobber stack frame) & -- & \textsection~\ref{sec:ingerc:epilogue} \\
After return & -- & -- & \textsection~\ref{sec:ingerc:epilogue} & Raise exception \\
\end{tabular}
\end{center}
\caption{Cancelled function resource cleanup by position within the running function}
\label{tab:cleanup}
\end{table}

While our approach mostly avoids adding operations to the common execution path, the
exception is epilogue handling.  To support that case, we add one function call and
six global variable accesses.  We saw in Chapter~\ref{chap:libgotcha} that in a
normal application, each of these operations has a negligible cost of just a few
cycles.  However, we also found that \textit{libgotcha} slows down accesses to
dynamically-linked global variables significantly.
Fortunately, the names of the symbols we introduce are well known, so
when integrating the new runtime components, we could introduce quirks to prevent
\textit{libgotcha} from intercepting their uses.  This would make the symbols
local to each libset and thereby obviate the need to pay the expensive reference
costs.
