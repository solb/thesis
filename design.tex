\section{Design}

Our interface (shown in Listing~\ref{lst:ingerapi}) for working with
preemptible functions consists of two primary functions, \texttt{launch()} and
\texttt{resume()}, and a structured \texttt{linger} type for storing return values
and execution state.  Client code creates a preemptible function by passing an
ordinary function (or closure) to the \texttt{launch()} function along with an
execution time cap.  If the function completes on time,
\texttt{launch()} returns its result to the caller; otherwise, it returns an opaque
continuation object that the caller may later pass to \texttt{resume()} to continue
executing the function from wherever it was preempted.

\begin{figure}
\begin{lstlisting}[label=lst:ingerapi,caption=Preemptible functions C-language interface]
struct linger {
  bool is_completion;
  union {
    void *completion;
    opaque_t continuation;
  };
};

typedef void *(*function_t)(void *);

struct linger launch(function_t func, uint64_t time_us, void *args);
void resume(struct linger *timed_func, uint64_t time_us);
\end{lstlisting}
\end{figure}

Listing~\ref{lst:usage} shows a usage example where the caller must
perform some work (say, call a watchdog) by a certain deadline, but first
calls into some helper code with weak time bounds.  With preemptible functions,
the caller does not have to trust the helper's timing properties to know that the
watchdog will be called
in time.

\begin{figure}
\begin{lstlisting}[label=lst:usage,caption=Preemptible function usage example]
res = launch(helper, TIMEOUT, no_args);
call_watchdog(); // Will not be late
if(res.is_completion) {
  // We are already done
  return res.completion;
} else {
  // Give helper() more time
  resume(&res, TIMEOUT);
}
\end{lstlisting}
\end{figure}

The intentional simplicity of our API masks the implementation effort required to
hide three categories of challenges:

\paragraph{Performance on contemporary systems stacks}
It is traditional to conceptualize concurrency as threads that must be scheduled onto
a CPU by some combination of the language runtime and kernel scheduler.  While this
has the obvious advantage of allowing the launched function to run on a different
core than its caller, it comes at the cost of involving the schedulers even when
parallelism is not desired.  Furthermore, recall from Section~\ref{sec:intro} that
conventional thread-based software stacks only support cancellation at process
granularity.  To efficiently support the common case of running and cancelling
preemptible functions on the same thread as the caller, our interface is synchronous
(but still compatible with threaded execution).  We achieve preemption on top of the
existing systems stack by subscribing to a periodic timer signal.

\paragraph{Automatic handling of shared state}
One of our goals is to allow the use of third-party libraries
without modifying or recompiling them.
Because our interface executes the preemptible function on the same thread, the
function might be interrupted before it completes, so the code following a call to
\texttt{launch()} or \texttt{resume()} shares certain attributes with a signal
handler.  As
such, it ought to be limited to calling async-signal-safe functions, as
discussed in Section~\ref{sec:related}.  Such a requirement is unacceptable, as
it would make preemptible functions almost impossible to use correctly.  Our work
lifts this requirement in the common case by automatically creating copies of
libraries to isolate their shared state.
Unlike prior work, we do this without relying solely on nonpreemptible regions
(which would severely weaken our ability to enforce timeouts):\@ in many cases, we
can instead use the runtime dynamic linker to maintain compatibility
with third-party code.

\paragraph{Language independence}
Our interface is inspired by that of Scheme engines~\cite{haynes:iucs1984}; however,
since we seek language independence, there are some fundamental differences.  Most
notably, our preemptible functions may contain state\footnote{In addition to lifting
a restriction on timed code, this difference led us to make the \texttt{linger} type
stateful as well, and allow the \texttt{resume()} function to mutate it in place
rather than constructing a copy.  In another move to improve portability, we chose to
make \texttt{linger} a structured type instead of a function, thereby avoiding
dependence on languages supporting closures.  In languages with operator
overloading, the preemptible function bindings may define a function-call operator
that calls \texttt{resume()} to achieve Scheme-style surface syntax.}, something not
handled by Scheme due to the functional nature of the language.  We simplify
their API by returning the preemptible function's return value in the \texttt{linger}
union rather than passing it to a success callback.  Our goal was to create a core
API amenable to language-specific integration
integration; for instance, in addition to the barebones C interface, our current
implementation exposes a Rust interface that uses a first-class tagged union (sum
type) to ensure at compile time that the caller has checked which variant a
\texttt{linger} contains before using it.

\vspace{\parsep}
Many programming languages now include a futures (promises) interface for expressing
the dependencies between asynchronous computations.  When a program needs to use an
asynchronously-computed result, it polls on the future to block until the
corresponding computation completes.  Most languages allow the poller to specify a
timeout, but this does not mean the computation is executed preemptively:\@ even
after its future times out, a computation containing an infinite loop will continue
to run, tying up a worker thread for the rest of the process lifecycle.  By wrapping
a preemptible function in a future, truly preemptive polling could be achieved.
