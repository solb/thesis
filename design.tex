\section{Design}

Our interface (shown in Listing~\ref{lst:ingerapi}) for working with
preemptible functions consists of two primary functions, \texttt{launch()} and
\texttt{resume()}, and a structured \texttt{linger} type for storing return values
and execution state.  Client code creates a preemptible function by passing an
ordinary function (or closure) to the \texttt{launch()} function along with an
execution time cap.  If the function completes on time,
\texttt{launch()} returns its result to the caller; otherwise, it returns an opaque
continuation object that the caller may later pass to \texttt{resume()} to continue
executing the function from wherever it was preempted.

\begin{figure}
\begin{lstlisting}[label=lst:ingerapi,caption=Preemptible functions C-language interface]
struct linger {
	bool is_completion;
	union {
		void *completion;
		opaque_t continuation;
	};
};

typedef void *(*function_t)(void *);

struct linger launch(function_t func, uint64_t time_us, void *args);
void resume(struct linger *timed_func, uint64_t time_us);
\end{lstlisting}
\end{figure}

Listing~\ref{lst:usage} shows a usage example where the caller must
perform some work (say, call a watchdog) by a certain deadline, but first
calls into some helper code with weak time bounds.  With preemptible functions,
the caller does not have to trust the helper's timing properties to know that the
watchdog will be called
in time.

\begin{figure}
\begin{lstlisting}[label=lst:usage,caption=Preemptible function usage example]
res = launch(helper, TIMEOUT, no_args);
call_watchdog(); // Will not be late
if(res.is_completion)
	// We are already done
	return res.completion;
else
	// Give helper() more time
	resume(&res, TIMEOUT);
\end{lstlisting}
\end{figure}

The intentional simplicity of our API masks the implementation effort required to
hide three categories of challenges:

\paragraph{Decent performance on contemporary systems stack}
Most prior work models concurrency in terms of threads that must be scheduled onto
a CPU by some combination of the language runtime and kernel scheduler.  While this
has the obvious advantage of allowing the launched function to run on a different
core than its caller, it comes at the cost of involving the schedulers even when
parallelism is not desired.  Furthermore, recall from Section~\ref{sec:intro} that
conventional thread-based software stacks only support cancellation at process
granularity.  To efficiently support the common case of running and cancelling
preemptible functions on the same thread as the caller,
our interface is synchronous.  We achieve preemption by subscribing to
a periodic timer signal; however, unlike RT and Shinjuku, we receive these signals
directly on the thread executing the preemptible function.  As we will soon see, this
does not render preemptible functions incompatible with threading; indeed, we will
demonstrate the straightforward application of our abstraction to implement
preemptive scheduling in a userland thread library.

\paragraph{Automatic handling of shared state}
Recall that one of our goals is to allow the use of most third-party libraries
without modifying or recompiling them.
Because our interface executes the preemptible function on the same thread, and said
function might be interrupted before it completes, the code following a call to
\texttt{launch()} or \texttt{resume()} shares certain attributes with a signal
handler.  As
such, we would expect it to be limited to calling async-signal-safe functions, as
discussed in Section~\ref{sec:related}.  We find such a requirement unacceptable, as
it would make preemptible functions almost impossible to use correctly.  Our work
lifts this requirement in the common case by automatically creating copies of
libraries to isolate their shared state.
Unlike prior work, we do this without relying solely on nonpreemptible regions
(which would severely weaken our ability to enforce timeouts):\@ in many cases, we
are able to instead leverage the runtime dynamic linker to maintain compatibility
with third-party code.

\paragraph{Language independence}
Our interface is inpired by that of Scheme engines~\cite{haynes:iucs1984}; however,
since we seek language independence, our design differs substantially from theirs.
Most notably, the Scheme language's functional nature meant that engines did not have
to handle shared state.  Our API differences from Scheme engines are as follows:
First, because the preemptible function itself may contain state, we make the
\texttt{linger} type stateful as well, allowing \texttt{resume()} to mutate it in
place.  Second, to support languages without closures, we
return a structured type instead of a function\footnote{Languages with operator
overloading may achieve the other syntax by defining the \texttt{linger} type's
function-call operator to call the \texttt{resume()} function.}
Finally, instead of passing the
preemptible function's return value to a separate callback function, we return it
directly from \texttt{launch()}.  This core API is amenable to good language-specific
integration; for instance, in addition to the barebones C interface, our current
implementation exposes a Rust interface that uses a first-class tagged union (sum
type) to ensure at compile time that the caller has checked which variant a
\texttt{linger} contains before using it.

\vspace{\parsep}
Our discussion may remind some readers of the futures (promises) interface now
present in many programming languages.  The futures programming model's main
contribution is the ability to express a parallel computation as a dataflow graph.
Contemporary futures runtimes do not support asynchronous preemption, but it is
possible to encapsulate preemptible functions inside futures to achieve this.
Although each language's futures differ slightly,
language bindings can be constructed in the following trivial manner:
To create a new preemptible future, the bindings should call \texttt{launch()} with a
budget of 0 $\mu$s.  Each attempt to poll the future for a value should resolve to a
call to \texttt{resume()} with the timeout passed to poll (if allowed by the
language's API), or else previously associated with the future by other means.
