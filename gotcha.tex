\chapter{Nonreentrancy and selective relinking: \\ the \textit{libgotcha} runtime}
\label{chap:libgotcha}

\ifdefined\chapquotes
\vspace{-1in}
\begin{chapquote}[1.5in]{James S.\@ A.\@ Corey, \textit{Nemesis Games}}
`Alien superweapons were used,' Alex said, walking into the room, \\
sleep-sweaty hair standing out from his skull in every direction. \\
`The laws of physics were altered, mistakes were made.'
\end{chapquote}
\fi

In Section~\ref{sec:libinger:reentrancy}, we saw that it is not safe in general for a
preemptible function to call into stateful code that was written without the
preemptible function abstraction in mind.  However, such code is prolific in the
modern systems stack, and in order to support interoperability with it, we need to
automatically transform the program to fix the safety hole.  This chapter covers a
novel software system designed to do just that, dubbed \textit{libgotcha}.

\begin{figure}
\begin{center}
\includegraphics[width=0.7\columnwidth]{figs/procimg_perobj}
\end{center}
\caption{Layout of a typical module within the process image.  \textbf{Bold} sections
contain program data; \textit{italicized} ones contain metadata for the runtime.}
\label{fig:procimgobj}
\end{figure}

\begin{promotesubsections}
\begin{swallowsections}
\input[functions]{gotcha_gotcha}
\end{swallowsections}
\end{promotesubsections}


\section{A brief tour of linking}

We begin with background about linking, a two-stage process that ultimately produces
an in-memory \textbf{process image} containing a program's code, all the data it
needs to execute, and the code and data of all its dependencies.  Linking operates on
\textbf{object files} that can take the form of either an \textbf{executable} or a
\textbf{shared library}.  Once a program is running, its process image contains a
region corresponding to each loaded object file.  We will refer to each such region as
a \textbf{module}, regardless of whether it corresponds to an executable or a shared
library.  Each module is divided into logical \textbf{sections}, each containing a
particular type of information.  Figure~\ref{fig:procimgobj} shows a typical module's
layout; notice that it contains both data corresponding to the source code and
generated metadata for runtime consumption.

The linking process occurs in two parts.  Static linking occurs at compile time and
forms the last step of the traditional build process.  Dynamic linking occurs at a
phase of runtime we will refer to as \textbf{load time}, because it starts before the
program has been loaded from disk or the language runtime initialized.

\solb{Diagram of unaltered lookup path through GOTs and PLTs}

\begin{promotesubsections}
\begin{swallowsections}
\input[functions]{gotcha_namespaces}


\input[functions]{gotcha_libsets}

\solb{Above reference is to the wrong listing!}

\solb{Define control library in the first paragraph}

\solb{Expand interface listing and add comments with section references}


\input[functions]{gotcha_init}
\hspace{-1.5em}
\input[functions]{gotcha_reinit}
\input[functions]{gotcha_linker}

\solb{Reinitialization diagram (non-TLS part)}


\input[functions]{gotcha_goot}

\solb{\textbf{Subsections?}}

\input[functions]{gotcha_plot}
\input[functions]{gotcha_globals}

\solb{\textbf{Subsection on thread-local storage}}

\solb{Diagram of thread-local data layout}

\solb{Reinitialization diagram (TLS part)}


\input[functions]{gotcha_uncopyable}

\solb{Add the effects in the above TODO to the UML diagram?}

\solb{Stipulate that libgotcha itself is always uncopyable}

\solb{Say that the hook function runs in the interrupted module's namespace}

\solb{Mention pre-call hooks}

\solb{Give the limitations of each type of hook}

\solb{\textbf{Section More on control libraries}}

\solb{Types of control libraries from the frontmatter}

\solb{Address monomorphization}


\input[functions]{gotcha_tls}

\solb{Drop TLS stuff, incorporating whatever is salvageable earlier}

\input[functions]{gotcha_relocations}

\solb{Move paragraph from Managing libsets here}

\solb{Recompiling glibc from the frontmatter}

\end{swallowsections}
\end{promotesubsections}


\section{Evaluation}

\input[functions]{eval_ugotcha}

\input[functions]{eval_testbed}

\solb{Thread creation, TLS allocation, and libtlsblock}
