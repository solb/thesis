Functions' addresses are replaced by the address of a special function,
\texttt{procedure\_linkage\_override()}.  Whenever the program tries to call one of
the affected functions, this intermediary checks the thread's next libset, looks up
the address of the appropriate definition in the shadow GOT, and jumps to it.
Because \texttt{procedure\_linkage\_override()} runs between the caller's
\texttt{call} instruction and the real function, it is written in assembly to avoid
clobbering registers.  Instead of being linked to their symbol definitions at load
time, some function calls resolve lazily the first time they are called:\@ their GOT
entries initially point to a special lookup function in the dynamic linker that
rewrites the GOT entry when invoked.  Such memoization would remove our intermediary,
so we alter the ELF relocation entries of affected symbols to trick the dynamic
linker into updating our shadow GOT instead.

\thesis{Cover trick for defeating lazy PLT relocations' memoization.}
