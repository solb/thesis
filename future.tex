\section{Future Work}
\label{sec:future}

As noted above, our exploration is preliminary; this section outlines several
open questions.  These questions fall into two categories: shortcomings in our
current implementation and defense-in-depth safeguards against unexpected
failures (e.g., compiler bug or the operator allowing use of a buggy or malicious library).

% This being preliminary work, there remain security problems in need of addressing.
% While a couple of them represent unresolved limitations in our implementation, most
% of these are simply defense in depth to ensure that an unexpected failure (e.g., due
% to a compiler bug or the operator allowing use of a buggy or malicious library) is
% contained.  We endeavor to catalog many of these here, along with proposed solutions
% that we intend to implement going forward.

\paragraph{Non-reentrancy.}
Our use of Rust panics to unwind the stack during preemption can
corrupt the internal state of non-reentrant functions (e.g., Rust's dynamic
allocator).  Possible fixes include blacklisting these functions and delaying
preemption until they are finished or replacing the problematic function with a
safe one (e.g., a custom memory allocator).
%allocator with a custom one that manages a separate heap for each microservice.  The
%latter approach would also open the possiblity of one-shot memory deallocation by
%simply ``deallocating'' an entire heap.

\paragraph{Host process.}
Our current implementation does not provide isolation between the host
and worker processes.  We plan to apply standard OS techniques to reduce
the chance of corruption or interference by a misbehaving worker.  Examples
include auditing interactions with the shared memory region to ensure invalid
or inconsistent data originating from the host cannot create an unrecoverable
dispatch error; handling the \texttt{SIGCHLD} signal to detect a worker that has
somehow crashed; and keeping a recovery log in the host process so that any
user jobs lost to a failed worker process can be reassigned to operational
workers.  

\paragraph{Further defense in depth with ERIM.}
ERIM outlines a set of techniques and binary rewriting tools useful for using
Intel's Memory Protection Keys to restrict memory access by 
threads within a process~\cite{vahldiek2018erim}.  While preliminary and without
source yet available, this appears to be an attractive approach for
defense-in-depth both within worker processes and between the worker and hosts.

%We believe it would be unwise to run the system in production without isolating the
%dispatch process from interference or corruption by a misbehaving worker.  In
%practice, this should include at least:\


\paragraph{Library functions.}
As with system calls, there may exist library functions in Rust (and certainly
in libc, which we deny by default) that are unsafe to allow microservices 
access to.  Because the Rust standard library by necessity uses unsafe
code,  defense-in-depth suggests that a whitelisting-based approach should be
employed for access to its functions.  Certainly library functions must
obviously be masked---for example, our use of Rust's panic handler for
preemption means that we must deny microservice code the ability to catch the
panic and return to execution.  While we mitigate this possibility by
detecting and blacklisting microservices that fail to yield under a SIGALRM, it
would be desirable to block such behavior entirely.
Possible
options include using the dynamic linker to interpose stub implementations or
linking against a custom build of the library, or using more in-depth static analysis.



% Similar steps may be needed in the Rust standard library as well; for
% instance, we currently have no way to stop a user from catching the panic we trigger
% and preventing control from returning to the worker loop, although this could be
% accomplished using static analysis or even a simple search when we compiled submitted
% code.
% simple search when we compiled submitted code.

\paragraph{Resource leaks.}
Safe Rust code provides memory safety, but it cannot prevent memory
leaks~\cite{www-rustlang-nu}.  For example, destructor invocation is not
guaranteed using Rust's default reference counting-based reclamation; therefore,
unwinding the stack during preemption is not guaranteed to free all of a
microservice's memory or other resources.  Potential solutions are interposing
on the dynamic allocator to record tracking information (likely proving
expensive) or using per-microservice heaps that main worker process can simply
deallocate when terminating a microservice.  The worker can also deallocate
other resources, such as unclosed file descriptors.  If these checks end up
being too expensive, the worker could execute its cleanup after a certain number
of microservices have run or when the load is sufficiently low.
%, and/or the system could track which microservices were (likely)
%responsible for resource leaks and impose warnings or bans.

\paragraph{Side channels.}
Our current approach is vulnerable to side-channel
attacks~\cite{Lipp2018,Kocher2018}.  For example, microservices have access to
the memory addresses and timings of dynamic memory allocations, as well as the
numbers of opened file descriptors.  Although side-channels exist in many
systems, the short duration of microservice functions may
make mounting such attacks more challenging; nevertheless, standard
preventative practices found in the literature should apply.

% As this year's Meltdown~\cite{Lipp2018} and Spectre~\cite{Kocher2018} CPU
% vulnerabilities reminded our community, the peril of timing attacks and
% other sidechannels can lurk at any abstraction layer and deserves careful
% consideration.  We are aware of at least two significant pieces of state
% that are observable in our approach: microservices have access to the memory
% addresses and timings of dynamic memory allocations, as well as the numbers of
% opened file descriptors.  We would not be surprised if an enterprising attacker
% could use this information to reconstruct private details of recently-executed
% microservices.  However, we believe that some of the steps proposed for
% addressing resource leaks could help block these side channels as well.  In the
% worst case, however, we may need to modify the Rust standard library (e.g., to
% prevent access to pointer addresses via debug formatting and/or disallow safe
% access to file descriptor numbers).
%
% We freely admit that the pervasive design changes proposed herein require more
% rigorous verification to catalog and mitigate side-channels; however, we believe this
% shortcoming is not limited to this project, but that many such severe attacks will be
% discovered in existing systems, both in the months to come and after the introduction
% of much-needed tooling to support the looming pervasive vulnerability search.

% The reader may be disheartened by the number of attack vectors we've considered, or
% even that we've failed to consider.  However, there is a small piece of good news:\
% because that the worker processes should be well-isolated from the rest of the
% system, solutions can be inexact, with the dispatcher restarting a failed worker and
% automatically banning suspect microservices in the worst case.

~\\
Despite the security challenges of running microservice as functions, worker
processes are still well-isolated from the rest of the system.
Worst case, the central host process can restart a failed worker and
automatically ban suspect microservices.
