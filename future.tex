\section{Future work}

Like \textit{Shinjuku} and \textit{RT} before us, we have presented a preemptive
thread library; however, we have done so by building parallelism on top of
preemption,
not the other way around.  Our approach offers a more flexible abstraction that
we intend to apply to other problems.  This also goes for our \textit{libgotcha}
tool, whose potential usese we explored in Section~\ref{sec:assafe}.

\solb{Mention other use cases, such as the continuation-memoizing RPC server?}

We also diverge from \textit{Shinjuku} and \textit{RT}
in another way:\@ unlike these systems, we make each thread responsible for
its own preemption rather than serving preemption signals from a shared watchdog
thread.  We informed this decision by back-of-the-envelope calculations based on
Shinjuku's measurements of bare-metal interprocessor
interrupts (IPIs) and Linux signals.

Although Shinjuku reports that IPIs take an average of only 1,993 cycles, compared to
4,950 for signals (roughly 1:2.5), their sender/receiver breakdown of the latter
number suggests significant latency savings by avoiding cross-core signaling:
First, 343 of those cycles (6.9\%) are spent propagating the signal between the two
cores; we expect this delay to be nearly absent for a timer interrupt originating at
its destination core's own local interrupt controller.  Second, 2,084 cycles (42\%)
are incurred by the sending core; assuming the interrupt controller supports
periodic timer interrupts at the necessary frequency, this cost does not need to be
paid between each interrupt, suggesting measurable savings here too.  Although
our prototype is not yet optimized to this extent, we expect it is possible for a
system built on intra-thread Linux signals to achieve an average preemption latency
within 2x that of Shinjuku's custom operating system.  It is our hope that further
optimization of our system, including through the adoption of some of Shinjuku's
continuation optimizations, will allow us to meet this performance goal.

Of course, in our design, increasing the granularity of preemption is a tradeoff:\@ more
frequent timer signals mean a tighter bound on timeout detection, but also a lower
throughput of useful work.  The correct balance depends at least on the
timeout value and size of the function, and also merits further study.

\thesis{Might want to accelerate kernel signal handling?}

\solb{Might want to support nested preemptible functions?}
