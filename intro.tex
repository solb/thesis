\chapter{Introduction}

The abstraction most fundamental to modern programs is the \textbf{function}, a
section of code that expects zero or more data inputs, performs some computation, and
produces zero or more outputs.  It is a structured control flow primitive that obeys
a strict convention:\@ whenever invoked from one of its \textbf{call sites}, a
function runs from beginning to (one possible) end, at which point execution resumes
in the \textbf{caller} just after the call site.  It is also a \textbf{synchronous}
primitive; that is, these steps happen sequentially and in order.  Because
processors conceptually implement synchronous computation, scheduling a function is
as trivial as instructing the processor to jump from the call site to its starting
address, then later jump back to the (saved) address of whatever follows the call
site.  Thus,
the program continues executing throughout, with no inherent need for intervention by
an external scheduler or other utility software.

Note, however, that just because the program has retained control does not mean the
programmer has.  Precisely because functions represent an abstraction, the programmer
who calls one is not necessarily familiar with its specific implementation.  This can
make it hard to predict the function's duration, yet calling it requires the
programmer to trust it to eventually finish and relinquish control.  The programmer
may have made a promise (a ``service-level agreement'') that their whole program
will complete within a specified timeframe; unfortunately, they cannot certify their
compliance without breaking the abstraction and examining the internals of each
function they call.  Even then, untrusted or unpredictable input may make a
function's performance characteristics unclear:  Perhaps the function solves a
problem that is
known to be intractable for certain cases, but such inputs are difficult to identify
\textit{a priori}.  Perhaps it performs format decoding or translation that is
susceptible to attacks such as decompression bombs.  Or perhaps it simply contains
bugs that open it to inefficient corner cases or even an infinite loop.

Faced with such problems, the programmer is often tempted to resort to an
\textbf{asynchronous} invocation strategy, whereby the function runs in the
background while the programmer maintains control of the rest of the program.  Common
abstractions following this model include the operating system's own processes and
threads, as well as the threads, coroutines, and futures (i.e., promises) provided by
some libraries and language runtimes.  Any use of asynchronous computation requires
an external scheduler to allocate work.

Here again, the programmer is sacrificing control.  Upon handing execution control to
a scheduler, dependencies are no longer clear from the program's structure and must
be passed to the scheduler by encoding them in synchronization constructs.  (For
instance, ``Do not execute past this line until no one else is accessing so-and-so
resource.'')  Sadly,
it is difficult to fully communicate the relevant bits of the application logic
across this abstraction boundary, which can result in unintended resource-sharing
effects such as priority inversion, where the scheduler chooses to run a different
task than the system designer anticipated.  Furthermore, each software scheduler is
itself a
piece of code, and because its job does not represent useful application work, any
time it spends executing is pure overhead.  Therefore, introducing unnecessary
scheduling necessarily reduces per-processor performance.

In many cases, the \textit{only} tool necessary to ensure timely completion of a
program is preemption, the ability to externally interrupt execution.  Instead of
confronting this directly, current programming
environments incentivize the programmer to rely on a scheduler to fix the problem,
limiting them to whatever coarse timescales (often milliseconds) the OS scheduler
operates at, or (in the case of userland schedulers) to cooperative scheduling
that doesn't even address the problem of infinite loops.  The goal of this work is to
design and prototype an interface that extends the programming model with simple
preemption, thereby allowing the use of functions without having to break the
abstraction and examine their implementations.  If a function times out, it is
paused so that the programmer can later resume and/or cancel it at the appropriate
time.  Note that such an interface is still inherently \textbf{concurrent}; that is,
the application has to manage multiple tasks at the same time.  Indeed, it is now the
programmer who expresses the schedule describing when to devote time to the timed
code, and how much.

It bears mentioning that sometimes a system designer does need asynchronous
invocation and a dedicated scheduler.  Most notably, both are necessary to support
\textbf{parallel} applications that actually \textit{execute} multiple tasks
simultaneously.  Preemptive function calls are equally applicable in such situations
because they compose with existing concurrency abstractions.  In fact, we find that
they make it surprisingly easy to extend existing cooperative schedulers with
preemption, without adding a dependency on nonstandard OS kernel features.


\section{Thesis statement}

We introduce novel programming abstractions for isolation of both time and memory.
They operate at finer granularity than traditional primitives, supporting preemption
at sub-millisecond timescales and tasks defined at the level of a function call.
This resolution enables new functionality for application programmers, including
users of unmanaged systems programming languages, all without requiring changes to
the existing systems stack.  Despite being concurrency abstractions, they employ
synchronous invocation to allow application programmers to make their own scheduling
decisions.  However, we find that they compose naturally with existing concurrency
abstractions centered around asynchronous background work, such as threads and
futures.  We demonstrate how such composition can enable asynchronous cancellation of
threads and the implementation of preemptive thread libraries in userland, both
regarded for decades as challenging problems.



\section{Structure and contributions}

This \thesis motivates and refines a new programming abstraction for calling a
function with a
timeout (Chapters~\ref{chap:functions} and \ref{chap:libinger}), and also introduces
a separate supporting primitive for providing memory isolation within a process
(Chapter~\ref{chap:libgotcha}).  It presents possible applications for memory
isolation (Chapter~\ref{chap:safety}) and timed function calls
(Chapters~\ref{chap:libturquoise}, \ref{chap:strobelight}, and
\ref{chap:microservices}).  It also includes an analysis of the barriers to and a
path toward achieving automatic resource cleanup upon cancellation of unfinished work
(Chapter~\ref{chap:ingerc}).

The remainder of the \thesis proceeds as follows:

\paragraph{Function calls with timeouts (Chapter~\ref{chap:functions})}
We examine prior approaches to running timed code from the literature, triaging the
state of the art's shortcomings.  In the process, we rediscover a nigh-forgotten
interface
for making timed function calls, Scheme engines.  The interface is elegant, but only
capable of handling purely functional code.  Drawing inspiration from it, we
devise a novel interface for calling impure \textit{preemptible functions}.
We set the goal of language agnosticism, aiming to demonstrate support
for unmanaged systems programming languages (because they provide few abstractions
that might be unavailable in other settings).  We also observe that using preemptible
functions in an application introduces concurrency that creates unsoundness arising
from shared state, specifically nonreentrancy.  The application developer is unable
to address the problem, so we conclude that doing so is a prerequisite for
implementing preemptible functions.

\paragraph{Nonreentrancy and selective relinking (Chapter~\ref{chap:libgotcha})}
We confront nonreentrancy, a program property that permeates the contemporary systems
stack from the operating system up, and which poses a fundamental safety challenge to
preempting impure code.  To overcome this hazard, we introduce a new form of memory
isolation called \textit{selective relinking}.  Crucially, this new primitive
operates at a granularity finer than a kernel thread; in fact, it can be applied at
the level of function calls, as is needed to support preemptible functions.

\paragraph{Rethinking POSIX safety (Chapter~\ref{chap:safety})}
We examine the POSIX safety concepts, the rules that govern what certain parts of a
Unix program may---or may not---do.  In particular, signal handlers are ordinarily
only allowed to call a restricted subset of the available standard functions, and
cancelable threads are conventionally barred from using operating system facilities
altogether.  We show how selective relinking can be applied to lift either of these
restrictions and enable \textit{safe signal handlers} or \textit{asynchronously
cancelable threads}.  While our implementations are only intended as a
demonstration, we find that selective relinking makes both of these seemingly
Herculean tasks simple enough to make for short instructive examples.

\paragraph{Function calls with timeouts, revisited (Chapter~\ref{chap:libinger})}
We return to the topic of preemptible functions and illustrate \textit{how to
implement them}
atop the existing systems stack.  In the process, we specialize what we had designed
as a C interface for the more modern Rust programming language.  The result is a
platform that exhibits both seamless interoperability with legacy code and harmonious
integration with the memory and concurrency safety features of Rust's type system.
The discussion serves both as a demonstration of the kind of considerations that
would go into integrating preemptible functions into other language ecosystems and as
an example of a more advanced use of selective relinking.

\paragraph{Resource cleanup and async unwinding (Chapter~\ref{chap:ingerc})}
We discuss the problem of resource leaks that can occur when asynchronously
cancelling code.  We develop an approach and proof-of-concept system for ensuring the
cleanup of cancelled code's own resources with assistance from the compiler.  We do
this by repurposing exception handling, which requires us to repair
\textit{asynchronous stack unwinding}.  Compilers have struggled to support this
feature, but we devise runtime workarounds to handle the troublesome cases,
introducing only minimal additional overhead on the normal execution path.

\paragraph{Preemptive userland threading (Chapter~\ref{chap:libturquoise})}
We show how preemptible functions compose with other concurrency abstractions, namely
threads and futures.  We create a \textit{thread library that implements preemptive
scheduling in userland} and still supports unmanaged code.  To accomplish this, we
construct a preemptible future type, which serves as a language-specific adapter of
the preemptible futures interface.  We use this to add preemption to the thread pool
of an existing futures executor.

\paragraph{Preemptive remote procedure calls (Chapter~\ref{chap:strobelight})}
We observe a pain point common to RPC systems that support impure code:\@ it is
universally incumbent on the developer of each server-side function to manually, and
periodically, check whether it has exceeded its service-level agreement.  We describe
how a first-year undergraduate student used our preemptible functions abstraction to
build a \textit{preemptive RPC system} without
this limitation in two months.  Its server uses a single-process architecture but
isolates each request within a preemptible function.  It is also capable of memoizing
both fully- and partially-computed requests; in the latter case, a repeat request
resumes the paused computation from wherever it left off.

\paragraph{Microsecond-scale microservices (Chapter~\ref{chap:microservices})}
We discuss how preemptible functions could be applied to address the problem of
\textit{invocation latency in serverless computing}.  Whereas contemporary systems
typically
place each tenant in its own separate container comprising one or more processes and
a virtual filesystem, we propose consolidating numerous tenants into a single worker
process.  Preemptible functions would provide compute time isolation, whereas memory
isolation would be achieved by requiring tenants to submit only code that could be
proven safe and restricting them to a vetted set of dependencies.

\paragraph{Conclusions and continuations (Chapter~\ref{chap:thatsawrap})}
We summarize our work, including noteworthy technical challenges we faced and
selected lessons for future systems builders.  We propose possible future research
directions.
