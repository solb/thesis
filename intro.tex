\section{Introduction}
\label{sec:intro}

After years of struggling to gain adoption, the coroutine has finally become a
mainstream abstraction for cooperatively scheduling function invocations.  Languages
as diverse as C\#, JavaScript, Kotlin, Python, and Rust now support ``async
functions,'' each of which expresses its dependencies by ``awaiting'' a
\textbf{future} (or promise); rather than polling, the language translates this to a
yield if the result is not yet available.

Key to the popularity of this concurrency abstraction is the ease and seamlessness of
parallelizing it.  Underlying most futures runtimes is some form of green threading
library, typically consisting of a scheduler that distributes work to a pool of
OS-managed worker threads.  Without uncommon kernel
support (e.g., scheduler activations~\cite{anderson:sosp1991}), however, this logical
threading model renders the operating system unaware of individual tasks, meaning
context switches are purely cooperative.  This limitation is common among userland
thread libraries, and illustrates the need for a mechanism for \textit{preemptive}
scheduling at finer granularity than the kernel thread.

In this paper, we propose an abstraction for calling a function with a timeout:  Once
invoked, the function runs on the same thread as the caller.  Should the function
time out, it is preempted and its execution state is returned as a continuation in
case the caller later wishes to resume it.  The abstraction is exposed via a wrapper
function reminiscent of a thread spawn interface such as \texttt{pthread\_create()}
(except \textit{synchronous}).  Despite their synchronous nature, \textbf{preemptible
functions} are useful to programs that are parallel or rely on asynchronous I/O;
indeed, we later demonstrate how our abstraction composes with futures and threads.

Introducing preemption into the contemporary programming model is fundamentally
challenging because the existing systems stack contains pervasive nonreentrancy,
support for which requires special handling to maintain compatibility with existing
code.  Those programmers who have written POSIX signal handlers have already grazed
the issue:\@ because such functions interrupt the rest of the program, they can
safely call only async-signal-safe (roughly, reentrant)
functions~\cite{signal-safety-manpage}.  This restriction makes writing nontrivial
signal handlers difficult, but it would be crippling if we had to extend it to
preemptible functions, which interrupt the rest of the program.  In any na\"ive
implementation, therefore, \textit{the rest of the program} would have to be written
like a signal handler.

The most obvious approach to implementing preemptible functions is to map them to
OS threads, where the function would run on a new thread that
could be cancelled upon timeout.  Unfortunately, the existence of nonreentrant
functions makes cancelling a thread hard.  UNIX's pthreads provide
asynchronous cancelability, but according to the Linux documentation, it ``is rarely
useful.  Since the thread could be cancelled at \textit{any} time, it cannot safely
reserve resources (e.g.  allocating memory with \texttt{malloc()}), acquire mutexes,
semaphores, or locks, and so on... some internal data structures (e.g., the linked
list of free blocks managed by the \texttt{malloc()} family of functions) may be left
in an inconsistent state if cancellation occurs in the middle of the function
call''~\cite{pthreadsetcanceltype-manpage}.  The same is true on Windows, whose API
documentation warns that asynchronously terminating a thread ``can result in the
following problems: If the target thread owns a critical section, the critical
section will not be released.  If the target thread is allocating memory from the
heap, the heap lock will not be released...'' and goes on from
there~\cite{www-microsoft-terminatethread}.

One might instead seek to implement preemptible functions via the UNIX
\texttt{fork()} call.  Even ignoring the high overhead, this would require
careful configuration of shared memory
to ensure that objects created outside the function were accessible inside, and worse
yet, that allocations performed inside remained available after it exited.
And all this is
without addressing the difficulty of even calling \texttt{fork()} in a multithreaded
program:\@ because doing so effectively cancels all threads in the child process
except the calling one, that process can experience the same problems as with the
thread approach~\cite{baumann:hotos2019}.

These na\"ive designs share another shortcoming:\@ in reducing
preemptible functions to a problem of parallelism, they hurt performance by placing
thread creation on the critical path and limit composability by increasing the
abstraction's complexity.  We observe that, when calling a function with
a timeout, it is concurrency alone---and not parallelism---that is fundamental.
Leveraging this key insight, we present a design that \textit{separates interruption
from asynchrony} in order to provide \textit{preemption at granularities in the tens
of microseconds}, orders of magnitude finer than contemporary OS schedulers'
millisecond scale.  Our research prototype\footnote{Our system is open source; the
code is available from
\href{https://efficient.github.io/\#lpf}{\texttt{efficient.github.io/\#lpf}}.} is
implemented entirely in userland, and
requires neither custom compiler or runtime support nor managed runtime features such
as garbage collection.

This paper makes three primary contributions:  (1) It proposes function calls that
return a continuation upon preemption, a novel primitive for unmanaged languages.
(2) It introduces selective relinking, a compiler-agnostic approach to automatically
lifting safety restrictions related to nonreentrancy.  (3) It demonstrates how to
support asynchronous function cancellation, a feature missing from state-of-the-art
approaches to preemption, even those that operate at the coarser granularity of a
kernel thread.
