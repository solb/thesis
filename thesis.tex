\documentclass[12pt,letterpaper]{book}

\usepackage{amssymb}
\usepackage{booktabs}
\usepackage[T1]{fontenc}
\usepackage[margin=1in]{geometry}
\usepackage{graphicx}
\usepackage[hidelinks]{hyperref}
\usepackage[mono=false]{libertine}
\usepackage{listings}
\usepackage{multirow}
\usepackage[numbers]{natbib}
\usepackage[libertine]{newtxmath}
\usepackage{subcaption}
\usepackage[svgnames]{xcolor}
\usepackage{xspace}

\newcommand{\chapquotes}{}
\newcommand{\Chap}{Chapter}
\newcommand{\chap}{chapter\xspace}
\newcommand{\paper}{thesis\xspace}
\newcommand{\Unix}{Unix\xspace}

\newcommand{\blacklist}[1][\xspace]{blocklist#1}
\newcommand{\whitelist}[1][\xspace]{allowlist#1}

\newcommand{\solb}[1]{{\color{magenta} TODO #1}}
\newcommand{\thesis}[1]{\solb{THESIS #1}}

\renewcommand{\contentsname}{Table of Contents}
\renewcommand{\lstlistlistingname}{List of Listings}

\lstset{captionpos=b,
	basicstyle=\ttfamily,
	keywordstyle=\color{Blue},
	commentstyle=\color{Green},
	columns=flexible,
	language=C++
}

\bibliographystyle{abbrvnat}

\newcommand{\ms}[1]{#1 ms}
\newcommand{\us}[1]{#1 $\mu$s}

\newcommand{\attribchapquote}{}
\newcommand{\gapchapquote}{}
\newcommand{\widthchapquote}{3in}
\newenvironment{chapquote}[2][1in]{
	\renewcommand{\attribchapquote}{#2}
	\renewcommand{\gapchapquote}{#1}
	\vspace{-2in}
	\begin{flushright}
	{\Large ``}
}{
	{\Large ''} \\
	--- \attribchapquote \\
	\rule{\widthchapquote}{1pt}
	\end{flushright}
	\vspace{\gapchapquote}
}

\makeatletter
\let\includegraphics@\includegraphics
\renewcommand{\includegraphics}[2][]{\includegraphics@[#1]{\includegraphicsdir#2}}
\newcommand{\includegraphicsdir}{}

\let\input@\input
\renewcommand{\input}[2][.]{
	\renewcommand{\includegraphicsdir}{#1/}
	\input@{#1/#2}
	\renewcommand{\includegraphicsdir}{}
}

\let\figure@\figure
\let\endfigure@\endfigure
\let\includegraphixs@\includegraphics
\let\caption@\caption
\let\label@\label
\newenvironment{swallowfigures}{
	\renewenvironment{figure}{
		\renewcommand{\includegraphics}[2][]{}
		\renewcommand{\caption}[1]{}
		\renewcommand{\label}[1]{}
	}{
		\let\includegraphics\includegraphixs@
		\let\caption\caption@
		\let\label\label@
	}
}{
	\let\figure\figure@
	\let\endfigure\endfigure@
}

\let\section@\section
\newenvironment{swallowsections}{
	\renewcommand{\section}[1]{}
}{
	\let\section\section@
}

\let\subsection@\subsection
\newenvironment{swallowsubsections}{
	\renewcommand{\subsection}[1]{}
}{
	\let\subsection\subsection@
}

\newenvironment{promotesubsections}{
	\renewcommand{\subsection}[1]{\section@{##1}}
}{
	\let\subsection\subsection@
}
\makeatother

\newenvironment{abstract}{}{}

\newcommand{\mytableistoobig}{}

\begin{document}

\makeatletter
\let\label@\label
\let\ref@\ref
\newenvironment{namespacereferences}[1]{
	\renewcommand{\label}[1]{\label@{#1##1}}
	\renewcommand{\ref}[1]{\ref@{#1##1}}
}{
	\let\label\label@
	\let\ref\ref@
}
\makeatother

\frontmatter

\begin{titlepage}
\begin{center}
	\vspace*{\fill}

	\textbf{\Large Lightweight Preemptible Functions} \\
	A thesis \\
	\hfill \\
	{\large Sol Boucher} \\
	\today \\

	\vspace{1in}

	\textbf{Thesis committee:} \\
	David G.\@ Andersen, \textit{chair} \\
	Adam Belay \\
	Michael Kaminsky \\
	Brandon Lucia \\

	\vspace{\fill}

	\textit{Submitted in partial fulfillment of the requirements \\
	for the degree of Doctor of Philosophy} \\
	\hfill \\
	Computer Science Department \\
	School of Computer Science \\
	Carnegie Mellon University \\
	Pittsburgh, PA 15213 \\
\end{center}
\end{titlepage}

\cleardoublepage
\addcontentsline{toc}{chapter}{\contentsname}
\tableofcontents
\listoffigures
\addcontentsline{toc}{chapter}{\listfigurename}
\listoftables
\addcontentsline{toc}{chapter}{\listtablename}
\lstlistoflistings
\addcontentsline{toc}{chapter}{\lstlistlistingname}

\chapter{TODOs THESIS}

\section{POSIX contexts self-signaling trick (897fe6b)}

The handler's logic is as follows:  It checks whether
the preemptible function has exceeded its timeout; if so, it swaps the contents
of the signal handler's continuation (accessible via the final argument to the
function~\cite{sigaction-manpage}) with a checkpoint continuation saved by
\texttt{launch()}.  This causes the subsequent return from the signal handler
to jump back to \texttt{launch()}, which then returns a \texttt{linger}
structure containing the signal handler's original context.  A subsequent
\texttt{resume()} call on this packaged continuation proceeds in much the same
way as \texttt{launch()}, but resumes the original computation by sending
itself a special signal with \texttt{pthread\_kill()}, then swapping the saved
context with the contents of that handler's context\footnote{This is necessary
because POSIX left the semantics of calling \texttt{setcontext()} on the
continuation saved by a signal handler invocation unspecified, leading
implementations such as GNU not to handle this
case~\cite{getcontext-manpage}.}.

Signal pool trick:\@ see 17b86d2.


\section{Program automatically repaired by \textit{libas-safe} (ee2602a)}

The main use of \textit{libgotcha} is to make library function calls async-signal safe when
they would not otherwise be.  To more clearly illustrate the library's usefulness, we
pause to give a minimal example of such usage; specifically, we will leverage
\textit{libgotcha} to automatically fix the buggy program in Listing~\ref{lst:handlerbug},
which calls the async-signal-unsafe function \texttt{printf()} from its signal
handler.  Unfortunately, this function takes a lock on the \texttt{stdout} stream's
associated file descriptor, and the signal handler eventually interrupts the program
within \texttt{fflush()} while it is holding this same lock, resulting in deadlock.

\begin{figure}
\begin{lstlisting}[label=lst:handlerbug,caption=C program with a buggy signal handler]
static void handler(int ignored) {
  printf("In signal handler\n");
}

int main(void) {
  struct sigaction sa = {
    .sa_handler = handler,
  };
  sigaction(SIGALRM, &sa, NULL);

  struct timeval tv = {
    .tv_sec = 1,
  };
  struct itimerval it = {
    .it_interval = tv,
    .it_value = tv,
  };
  setitimer(ITIMER_REAL, &it, NULL);

  while(true)
    fflush(stdout);
}
\end{lstlisting}
\end{figure}


\section{Footnote introducing the concept of \textit{libac-safe} (d3336e4)}

While \textit{libas-safe} reestablishes async-signal safety, it would also be
possible to write a \textit{libgotcha} control library that did the same for what POSIX calls
async-cancellation safety.  The obvious application of this would be supporting
asynchronous thread cancellation (in the common case where \whitelist{ed} code was not
currently executing):\@ as we saw in Section~\ref{sec:intro}, today's POSIX and
Windows interfaces for this are broken to the point of practical uselessness.


\section{Preemptible functions rebuttal}

\input{functions/rebuttal/atc20-reviews-198.txt}


\section{A note on terminology (inger 7d5b2e2)}

We use the term lightweight preemptible function (LPF) to refer to the timed \textit{version} of a
function, as invoked via the \texttt{launch()} wrapper function in this library.  It's not quite right to
say that \textit{libinger} "provides preemptible functions"; rather, it provides a transformation from
an ordinary function into a preemptible one.

To provide the memory isolation necessary to introduce preemption and asynchronous cancellation at
sub-thread granularity without breaking existing program dependencies, the \textit{libgotcha} runtime
allocates a separate copy of all the program's loaded dynamic libraries for each preemptible
function.  While the thesis refers to this isolation unit as a libset, that term was unfortunately
coined late in development; as such, the source code and configuration variables refer to it as a
"group" instead.


\section{A note on design (inger 7d5b2e2)}

The
\texttt{pause()} primitive allows a preemptible function to "yield" back to its caller by immediately
"timing out."  One can imagine building higher-level synchronization constructs atop this; for
example, a custom mutex that paused instead of blocking would allow two or more preemptible
functions to share state, even when some of them executed from the same kernel thread.

\textbf{We provide wrapper functions rather than a language extension.}  Many
	have argued that it is a mistake to extend a language's library with
	threading APIs without also endowing the language itself with an
	understanding of concurrency.  The number of preventable concurrency bugs
	that ISO C and POSIX threads have enabled seems to support this view.
	However, the decision has been made, concurrency is part of the C language,
	and the resulting bugs have been released upon the unsuspecting world.
	Fortunately, newer languages such as Rust have demonstrated that sound
	function-level concurrency is possible.  We provide wrapper functions
	reminiscent of \texttt{pthread\_create()} and leave it up to the language
	bindings to provide as much type safety as possible.


\section{A note on implementation (inger 7d5b2e2)}

The timer signal handler in \textit{libinger} refuses to pause while the next libset is set to 0 (the
starting libset).  Because \textit{libinger} is statically linked with \textit{libgotcha}, the latter enforces a
transparent switch to this libset whenever a dynamic function call transfers control into the module
in the process image that corresponds to \texttt{libinger.so}.  This means that preemption is deferred on a
given kernel thread while \textit{libinger}'s own code is executing on that thread.

Of course, things are not quite that simple.  There are noteworthy exceptions to the rule:
\begin{itemize}
\item The generic
	functions are therefore implemented such that they package everything that differs by type, then
	call into non-specialized functions such as \texttt{setup\_stack()} and \texttt{switch\_stack()} to do the scary
	stuff.
\item The \texttt{resume\_preemption()} function is installed as a \textit{libgotcha} callback hook, and is implicitly
	invoked at the end of each deferred-preemption library call made by a preemptible function.
	\textit{This happens in the preemptible function's libset rather than the starting one}; this is
	essential because the callback's main task is to force the timer signal handler to run
	\textit{immediately} and check for a timeout, and we don't want the libset to inhibit preemption!
\end{itemize}


\section{Lessons for system builders (slides)}

Resuming is a useful feature that is cheap, but introduces concurrency.
Cannot get CPU time isolation without memory isolation.
Design abstractions modularly and with an eye to simple use cases (e.g., \textit{libgotcha} as a separate
runtime with a very small control API, things listed in the note on design).
Treat debuggability as a first-order concern (e.g., permit disabling features that interfere at
runtime, test and maintain support for running under debugging and diagnostic tools).


\chapter*{Foreword}

Perhaps it is inevitable that when a nonfiction work reaches a certain length, it
begins to serve multiple purposes; if so, this document is no exception.  Yes, it is
a record of the ideas I have explored over the past years of my life.  But like any
thesis, it is also a lesson:\@ in summarizing my findings from these explorations, it
endeavors to save you from spending years of your own on the same topic.  And like
any good lesson, this one begins with an exercise...

With your permission, we will conduct a brief mindfulness activity.  At this moment,
and for however much longer you focus on this document, you mind will be occupied by
ideas.  Many of these ideas I will have put there.  Shortly, they will be ideas about
computer systems, but first let us consider:  How are the ideas getting to you?  You
are reading, but what does that mean?  Perhaps you are holding a printout or a bound
copy of this manuscript, or perhaps you have loaded it onto your personal computer,
tablet, phone, or hand terminal.  In any case, you have opened it to a particular
page, exposing your eyes to a sea of shapes arranged into nested clusters.  Your eyes
have gravitated to a cluster of particularly large shapes known as a ``chapter
title,'' then they have scanned across the page and sent a compressed representation
of each smaller ``word'' cluster of shapes to your brain, which has matched the
clusters of shapes to entries in your mental lexicon, then parsed them according to a
set of linguistic rules to infer their logical connections.  Then you have moved on
to the large ``paragraph'' clusters and processed each in turn, starting with the
first of its ``sentence'' clusters, and in so doing, learning what the next sentences
will be about.  Occasionally, something will go wrong at one of these steps and you
will backtrack and notice a missed word, or more closely examine a misidentified
word, or try a different parsing of the sentence, or reexamine the logical flow of
the paragraph.  You will usually not realize you are doing any of this, preferring
to think simply that you are \textbf{reading}.

The document you are reading is about computer systems, and like your brain, such
systems have many complexities.  If we as computer users had to describe the full
process for doing everything, we would never accomplish anything, so instead we
build \textbf{abstractions} for performing common tasks without examining the
underlying details.  Some would say that any computer systems research is
fundamentally about abstractions.  This particular work centers around an abstraction
for use by application programmers, who in turn work on top of other abstractions,
the most notable of which is software called the \textbf{operating system}.

In computing, as in life, one's fundamental goal is to accomplish some task using a
set of shared resources.  Someone must decide how to allocate these shared resources,
a role usually filled for a particular resource by a piece of software called its
\textbf{scheduler}.  One major responsibility of the operating system is to share
hardware resources such as the processor, the short- and long-term storage devices,
devices for user interaction, and network interfaces.  Among these, the one most
relevant to our discussion is the CPU scheduler, which manages the processor.

Despite itself being an abstraction that hides enormous complexity, a processor is
conceptually quite simple:\@ it receives a stream of simple \textbf{instructions}
telling it what to do, executing them in the order received and occasionally jumping
to a particular point elsewhere in the stream when so instructed.  The simplicity of
this model belies the infinite expressive power of programs constructed from such
instructions.  Indeed, programming at the instruction level is difficult not only
because the simplicity of the instructions make it verbose, but also because ad-hoc
jumps can be deceptively complicated to reason about.  Modern programmers usually
write software in programming languages that provide so-called ``structured control''
abstractions for performing common, formulaic sequences of jumps.

The most fundamental abstraction composing a structured program is the
\textbf{function}:\@ a section of code that expects zero or more input data, performs
some computation, and generates zero or more output data.  One function can call
another, which automatically transfers the input data and jumps to the start of that
function's code.  Later, when the end of its code is reached, the function
automatically jumps back to the program point just after it was called and transfers
its output data back.  Notice that a function call is \textbf{synchronous}; that is,
the function runs to completion before the calling function continues to run.
Because such sequential execution matches the processor's inherent behavior, sharing
the processor between functions is trivial and requires no scheduler.

Of course, an important feature of modern computers is the ability to work on
multiple tasks alongside each other, such as reading a document and composing an
outline or notes.  Operating systems manage such situations by providing an
abstraction called a \textbf{process}, or independent task.  Each process is isolated
from the others on the system and cannot access their data.  Furthermore, processes
exhibit a property known as \textbf{concurrency} wherein their executions can
interleave such that one process executes some of its code ``in the middle of''
another process's work.  (Think of momentarily putting your notetaking on hold to
scroll down in the document you're reading.)

Because isolation prevents processes from calling each other's functions directly,
switching between processes requires a scheduler to transfer control of the
processor.  Specifically, the processor must stop executing the running process and
start running the operating system's CPU scheduler code, which then performs an
action called a \textbf{context switch}:\@ it saves a checkpoint of that process and
restores the other process, resuming it from the state in which it last left off.
The conceptually simpler way for this transition to happen is \textbf{cooperative}
multitasking, in which the former process voluntarily gives up control of the
processor by explicitly telling the operating system to give someone else a turn.
Unfortunately, it is not safe to assume a process will eventually cede its processor,
as it may never decide to do so, through either misbehavior or malice.  Such a
scenario would render the rest of the programs unusable.

Fortunately, processors have a low-level mechanism for spontaneously changing which
instruction they are executing known as a \textbf{timer interrupt}.  Every so often,
the processor jumps into the OS scheduler from whatever code it is currently
executing.  Since it is now has the use of the processor, the scheduler can decide
whether to jump back to that same program or context switch to a different one, a
decision that is usually made based on how long the former program had been running
since the last context switch.  This style of process scheduling is known as
\textbf{preemptive} multitasking because the operating system initiates it by
actively pausing the running process.

Recent decades have seen the introduction of multicore computers that have more than
one processor, creating the opportunity for the operating system to schedule a
different process on each.  Such processes exhibit \textbf{parallelism}:\@ they
actually run at the same time.  Parallelism is also an attractive feature for
application programmers because by carefully restructuring their programs, they can
route some of their work to each processor, thereby speeding up portions of their
program's run.  Unfortunately, fitting such programs into operating systems' existing
process model was cumbersome.

To better accommodate parallel programs, operating systems introduced a hybrid
abstraction called a \textbf{thread}.  Like processes, threads can be both concurrent
and parallel.  Unlike processes, though, threads must share data to effectively
work together on a single task, so the threads within a process are not isolated from
one another.  It turns out that the simultaneous presence of concurrency and shared
data introduces fundamental challenges that make it difficult to write correct
programs due to a class of bugs informally known as race conditions.  Although safe
concurrency is a popular area of study, challenges remain particularly in systems
containing components that predate the parallel programming paradigm.  More detailed
coverage of safe concurrency and backwards compatibility as they relate to this
thesis work appears in chapters~\ref{chap:libinger} and \ref{chap:libgotcha},
respectively.

The lack of isolation between threads permits the programmer to spawn a thread in
much the same way they would call a function:\@ in most programming languages, they
place the code they want to execute on the new thread in its own function, but
instead of calling it directly, they pass it to a special wrapper function.  The
wrapper sets up the thread and begins running the programmer's custom thread thereon.
However, in an important break from functions, threads are \textbf{asynchronous} like
processes.  The wrapper function returns almost instantly, even if the thread is
still running in the background.  As with processes, this property means there must
be a scheduler to decide which application code each processor should run.  Note
that for the sake of this discussion, we are assuming this is the operating system's
CPU scheduler; however, this is not always the case and sometimes a custom scheduler
runs as part of the application itself, a configuration that is addressed in detail
in chapter~\ref{chap:libturquoise}.

Introducing additional scheduler dependencies on an application has important
functionality and performance ramifications for two fundamental reasons.  First, the
scheduler's placement behind an abstraction decouples it from the program's logic,
thereby imposing one or more levels of communication barrier that reduce its
understanding of the particular application's needs, often resulting in a brittle
policy ill suited to the workload.  For instance, few preemptive schedulers provide a
way for an application to customize the timer interrupt interval, even when supported
by the hardware.  Second, every scheduler works by running its own code to make
decisions about how to allocate a resource.  Because it does not represent useful
work from the application's perspective, time spent this way is pure overhead, and it
follows that introducing unnecessary scheduling necessarily reduces performance.

Threads represent the standard application for exploiting preemption within an
application.  However, reminiscent of how processes were cumbersome to use for
parallel programming, threads are ill suited to some use cases of preemption.  For
one thing, programmers who do not need parallelism are led to build their synchrony
atop asynchrony, thereby introducing a useless scheduler dependency.  For instance,
when calling a helper function but needing a result by a specific deadline is tempted
to spawn the function on its own thread, then immediately wait for the thread to
finish, a task better accomplished on the same thread.  Furthermore, although they
support pausing code mid-execution, threads make it very difficult to cancel
in-progress work that is no longer needed at all.

Fortunately, the tendency to use threads for application-level preemption is not
because the operating system does not expose hardware features such as timers.
Rather, it is because such features are presented as very low-level abstractions that
perform hardware-style unstructured jumps rather than using language-style structured
control and abstracting away the details of context switching.  We therefore propose
a new abstraction for easy preemption within an application, but show that it can be
implemented on top of the existing operating system.



\mainmatter

\section{Introduction}
\label{sec:intro}

As the scope and scale of Internet services continues to grow, system designers
have sought platforms that simplify scaling and deployment.
Services that outgrew self-hosted servers moved to datacenter racks, then
eventually to virtualized cloud hosting environments.
However, this model only partially delivered two related benefits:
\begin{enumerate}
\item Pay for only what you use at very fine granularity.
\item Scale up rapidly on demand.
\end{enumerate}

The VM approach suffered from relatively coarse granularity, creating entire
machines at a time, and with minutes to months of minimum billing
time.
%\footnote{Amazon recently followed Google's lead in moving down from
%  hour-granularity, but both services remain more expensive for always-on VMs
%  than providers who rent in month-long increments.} \ak{I don't think we
%need this footnote.}
Relatively long startup
times required system designers to keep some spare capacity online to handle load
spikes.

These shortcomings led cloud providers to introduce a new model, known as
serverless computing, in which the customer provides \textit{only} their code,
without having to configure its environment.   Such ``Function as a Service''
(FaaS) platforms are now available as AWS Lambda~\cite{www-amazon-lambda}, Google
Cloud Functions~\cite{www-google-cf}, Azure Functions~\cite{www-microsoft-af}, and
Apache OpenWhisk~\cite{www-apache-openwhisk}.  These platforms provide a model in
which: (1)  User code is invoked whenever some \textbf{event} occurs (e.g., an HTTP
request to some API), runs to completion, and nominally stops running (and being
billed) after it completes; and (2)  There is no state preserved between
separate invocations of the user code.  Property (2) enables easy auto-scaling
of the function as load changes.

\solb{We used to address the 5- to 9-minute capped runtime quota here; add it back in
if we end up referring back to it.}

Because these services execute within a cloud provider's existing
infrastructure, they reap the benefits of low-latency access to other cloud
services.  In fact, acting as an access-control proxy is a recurring microservice
pattern:\ receive an API request from a user, validate it, then access
a backend storage service (e.g., S3) using the service's credentials.

In this paper, we explore a design intended to reduce the tension between two of
the desiderata for cloud functions:\ low latency invocation and low cost.  We first
emphasize that serverless platforms' invocation latencies are bimodal, with the cusps
corresponding to warm and cold starts.  The latter occur when the cloud provider has
discarded a microservice's container after several minutes of inactivity, and
currently range from hundreds of milliseconds to seconds.  Although our approach can
reduce these warm start times by at least an order of magnitude, \solb{We don't
currently elaborate on this, having streamlined the next section to focus on warm
starts.  Is it a claim we want to include at all?  Should I add a brief paragraph in
it in the motivation?} we focus our
argument on warm starts because we expect to be able to keep many more of our lighter
microservices warm.

Even in the case of warm starts, we find that traditional invocation techniques
exhibit a significant latency tail.  Unfortunately, such a latency profile is
unsuitable for many modern distributed systems (e.g., those that handle search
indexing or assembling multi-component views for website front pages), which involve
high-fanout communication, sometimes performing tens or thousands of
lookups to handle each user request.  Because their user-visible response time often
depends on the tail latency of the slowest chain of dependent
responses~\cite{Dean:cacm2013}, much recent work seeks to reduce
such systems' response
times~\cite{Jalaparti:sigcomm2013,Xu:nsdi2013,Li:socc2014,Jeon:asplos2016}.

Thus we seek to reduce both the invocation latency and its predictability, a
goal supported by the impressively low network latencies available in
modern datacenters. It
now takes $<20\mu{}s$ to perform an RPC between two machines in Microsoft's Azure
cloud~\cite{www-firestone-azure-latency}.  We believe, however, that fully leveraging
this improving network performance will require reducing microservices' invocation
latencies to the point where the network is once again the bottleneck.

We further hypothesize---admittedly without much proof for this chicken-and-egg
scenario---that substantially reducing both the latency and cost of running
infrequently-used services will enable new classes and scales of applications
for cloud functions, and in the remainder of this paper, present a design that
achieves this.  As Lampson noted, there is power in making systems 
``fast rather than general or powerful''~\cite{Lampson1983}, because fast
building blocks can be used more widely.

In this paper, we propose a restructuring of the serverless model centered around
low-latency: \textit{lightweight microservices} run in \textit{shared processes}
and are isolated primarily using \textit{compile-time guarantees} and
\textit{fine-grained preemption}.

%% \subsection{Not yet rewritten}

%% At the time of writing, AWS Lambda, Azure Functions, and Google Cloud Functions
%% cap instance execution time to 5--10 minutes.

%% \solb{AK: I see you touched this recently.  Do you still want to see us say it (and
%% where)?  I currently mention penultimate section of Motivation and related work that
%% providers presently impose a limit, without actually saying what it is.}
%% \mk{This subsection seems out of place.}

\chapter{Function calls with timeouts}
\label{chap:functions}

\ifdefined\chapquotes
\begin{chapquote}{Isaac Asimov, \textit{The Gods Themselves}}
`Does everyone just believe what he wants to?' \\
`As long as possible.  Sometimes longer.'
\end{chapquote}
\fi

In this chapter, we introduce the design of lightweight preemptible functions, our
abstraction for making ordinary function calls with a timeout.  We will cover the
implementation in Chapter~\ref{chap:libinger}, after the intervening chapters have
introduced a supporting abstraction for memory isolation and shown how to use it.

One thing that distinguishes preemptible functions is that their invocation is
synchronous;
that is, the program does not continue executing the code following the call until
the callee has made some progress (though not necessarily run to completion).  This
stands in contrast to abstractions with asynchronous invocation.  The thread and
callback-based future abstractions are like preemptible functions in that structuring
code for them involves writing a function describing the task (a thread main function
or a callback, respectively).  But these traditional abstractions differ in that this
function does not necessarily
begin executing until later:\@ a scheduler or event loop in the runtime or operating
system manages all of the program's tasks and decides when to invoke each one.

The other distinguishing feature of preemptible functions is that they are
preemptive, meaning that interruption is external and can occur at (almost) any point
in their execution.  Abstractions such as futures and user threads are generally
cooperative.  As such, interruption is internal and control transfers to another task
only when the executing one explicitly yields control (or calls a library function
that does so on its behalf).  Table~\ref{tab:invokeinterrupt} classifies concurrency
abstractions based on their type of invocation and interruption.

\begin{table}
\begin{center}
\begin{tabular}{r r | c c}
&& \multicolumn{2}{c}{task interruption} \\
&& cooperative & preemptive \\
\hline
& \\
\multirow{3}{*}{\rotatebox{90}{task invocation}} & \multirow{2}{*}{asynchronous} & user threads & \multirow{2}{*}{kernel threads} \\
&& callback-based futures \\
& \\
& synchronous & ``async'' futures & \textbf{preemptible functions} \\
&& (\texttt{async}/\texttt{await}) \\
& \\
\end{tabular}
\end{center}
\caption{Concurrency abstractions classified by type of invocation and interruption}
\label{tab:invokeinterrupt}
\end{table}

From the table, it is apparent that the term \textbf{asynchronous} is overloaded.
In the context of invocation, it is synonymous with ``background.''  Confusingly, in
the context of futures, async functions are those that can use the \texttt{await}
keyword to insert a yield point that takes a \textit{synchronous} function call off
the critical path.  POSIX has a third meaning for the term:  In the context of
signals and cancellation, it means ``preemptive.''  Examples of this usage include
the phrases ``asynchronous cancellation'' and ``async safety.''


\section{Motivation}
\label{sec:functions:motivation}

\begin{figure}
\begin{center}
\includegraphics[width=0.65\columnwidth]{functions/figs/progsupport}
\end{center}
\caption[Taxonomy of support for library code]{
Taxonomy of support for library code.  It is difficult to determine whether
a library is fully reentrant, so in practice we always
apply one of the two mitigations.  Library copying is used by default, but deferred
preemption is needed to preserve the semantics of \texttt{malloc()} and users of
uncopyable resources such as file descriptors or network adapters.}
\label{fig:progsupport}
\end{figure}

\begin{swallowsections}
\begin{swallowfigures}
\input[functions]{intro}
\end{swallowfigures}
\end{swallowsections}


\input[functions]{related}

Some recent languages offer explicit userland threading, which can be used to
support timed code.  One example is the Go language's~\cite{www-golang}
\textit{goroutines}, which originally relied on a cooperative scheduler that
conditionally yielded at function call sites.  This caused real-world problems for
tight loops, requiring affected programmers to manually add calls to the
\texttt{runtime.Gosched()} yield function~\cite{www-golang-tightloop}.  To address
this, the language eventually migrated to a preemptive goroutine
scheduler~\cite{www-golang-rel14notes}.

\input[functions]{related_gofigure}


\section{Preemptible functions: \textit{libinger}}

We observe that today's concurrency abstractions offer either synchronous invocation
or preemptive scheduling, but not both.  On one hand we have futures, which are now
synchronous\footnote{Code interfacing with futures via the ``async/await''
continuation passing style is now prolific.} but purely cooperative.  On the other
are kernel threads, which are preemptive but asynchronous.  We bridge this gap by
introducing a novel abstraction that provides synchrony \textit{and} preemption for
unmanaged languages.

Doing so requires confronting the very nonreentrancy problems that have long doomed
attempts to support asynchronous cancellation outside of purely functional contexts.
This turns out to be a slightly harder problem than safely supporting concurrency, so
in addition to cancellation, we get the ability to externally pause for free.

\begin{swallowsections}
\input[functions]{inger_inger}
\end{swallowsections}


\subsection{Design principles}

Although we are introducing a new concurrency abstraction, we have striven to keep
the interface simple and understandable.  The following design principles have guided
this effort:
\begin{itemize}
\item \textbf{We do not assume that users need asynchrony.}  Hence, preemptible
	functions \textit{run on the same kernel thread as their caller}.  This is
	good for performance (especially invocation latency), but it is also
	important to be aware of; for instance, it means that a preemptible function
	will deadlock if it attempts to acquire a lock held by its caller, or vice
	versa.  Of course, some users may need asynchrony.  The preemptible function
	abstraction composes naturally with both threads and futures
	(Chapter~\ref{chap:libturquoise}), so there is no need to reinvent the wheel.
\item \textbf{We assume that simply calling a preemptible function is the common use
	case.}  As such, the \texttt{launch()} wrapper both constructs and invokes
	the preemptible function rather than asking the user to first employ a
	separate constructor.  Users wishing to separate the construction and
	invocation operations can pass the sentinel \texttt{0} as the
	timeout, then later use \texttt{resume()} to start execution.
\item \textbf{We favor a simple, language-agnostic interface.}  The fact that our
	interface centers on a higher-order function in the style of the
	\texttt{pthread\_create()} and \texttt{spawn()} wrapper functions means that
	using preemptible functions looks similar regardless of the programming
	language.  Currently, \textit{libinger} provides bindings for both C and
	Rust.  If and when we add bindings for other languages, we expect them to
	have the same feel; in the meantime, other languages can use preemptible
	functions (unsafely) through their C
	foreign-function interfaces.  We considered adhering to the futures interface
	instead, but decided against it because each language has its own
	incompatible variant thereof.  The relative ease of building a futures
	adapter type (Chapter~\ref{chap:libturquoise}) affirms our decision.
\item \textbf{We keep argument and return value passing simple yet extensible.}
	Because Rust supports closures, the Rust version of \texttt{launch()} accepts
	only nullary functions:\@ those seeking to pass arguments should just capture
	them from the environment.  C supports neither closures nor generics, so the
	C version of \texttt{launch()} accepts a single \texttt{void *} argument that
	can serve as an in/out parameter.  It occupies the last position in the
	parameter list to permit (possible) eventual support for variable argument
	lists.
\item \textbf{We choose defaults to favor flexibility and performance.}  When a
	preemptible function times out, \textit{libinger} assumes the caller might
	later want to resume it from where it left off.  As such, both
	\texttt{launch()} and \texttt{resume()} pause in this situation; this incurs
	some memory and time overhead to provide a separate execution stack and
	package the continuation object, but exhibits lower latency than
	asynchronous cancellation.  If the program does require cancellation, we
	provide ways to explicitly request it (Chapter~\ref{chap:libinger}).
\item \textbf{In addition to preemption, we offer the option to yield.} This feature
	enables the construction of higher-level synchronization constructs tailored
	to preemptible functions (Chapter~\ref{chap:libinger}).  It also allows
	preemptible functions to coexist with cooperatively-scheduled tasks
	such as futures (Chapter~\ref{chap:libturquoise}).
\end{itemize}


\section{The preemptible functions ecosystem}
\label{sec:libinger:concurrency}

In divorcing preemption from asynchronous invocation, preemptible functions
disentangle interruption from parallelism.  Indeed, \textit{libinger} does not
provide a task scheduler because the only decision it makes is whether to pause the
currently executing code.  Whenever it opts to do so, it unconditionally returns
control to the preemptible function's caller.

This design allows client code to pick and choose the level of runtime support it
needs.  If it only invokes preemptible functions synchronously and makes all
scheduling decisions itself, it can link directly against \textit{libinger} and use
the interface presented in Section~\ref{sec:libinger}.  If it prefers to delegate
scheduling to a runtime, we also provide \textit{libturquoise}, a preemptive futures
executor providing an event loop and optional thread pool
(Chapter~\ref{chap:libturquoise}).

Figure~\ref{fig:architecture} shows the dependency relationship between
\textit{libinger} and \textit{libturquoise}, in the context of the other software
components developed for this thesis.  Notably, both libraries support nonreentrancy
by depending on another library called \textit{libgotcha}, which provides a novel
abstraction of its own for enforcing isolation boundaries.  The \textit{libinger}
library is implemented in approximately 2,500 lines of Rust; \textit{libgotcha}
comprises another 3,000 lines of C, Rust, and x86-64 assembly.

\begin{figure}
\begin{center}
\includegraphics[width=\columnwidth]{functions/figs/architecture}
\end{center}
\caption[Preemptible functions software stack]{
Preemptible functions software stack.  Hexagonal boxes show
the required runtime environment.  Rectangular boxes represent components
implementing the preemptible functions abstraction.  Ovals represent components built
on top of these.  A preemptible function's body (i.e., \texttt{func}) may be defined
directly in
your program, or in some other loaded library.}
\label{fig:architecture}
\end{figure}


\subsection{Automatic handling of shared state: \textit{libgotcha}}

\begin{swallowsubsections}
\input[functions]{inger_gotcha}
\end{swallowsubsections}

Note that preemptible functions are still a concurrency abstraction, and our
automatic handling of shared state internal to dependencies does not exempt the
author of a preemptible function from writing safe concurrent code.

\chapter{Nonreentrancy and selective relinking: \\ the \textit{libgotcha} runtime}
\label{chap:libgotcha}

\ifdefined\chapquotes
\vspace{-1in}
\begin{chapquote}[1.5in]{James S.\@ A.\@ Corey, \textit{Nemesis Games}}
`Alien superweapons were used,' Alex said, walking into the room, \\
sleep-sweaty hair standing out from his skull in every direction. \\
`The laws of physics were altered, mistakes were made.'
\end{chapquote}
\fi

In Section~\ref{sec:libinger:reentrancy}, we saw that it is not safe in general for a
preemptible function to call into stateful code that was written without the
preemptible function abstraction in mind.  However, such code is prolific in the
modern systems stack, and in order to support interoperability with it, we need to
automatically transform the program to fix the safety hole.  This chapter covers a
novel software system designed to do just that, dubbed \textit{libgotcha}.

\begin{figure}
\begin{center}
\includegraphics[width=0.7\columnwidth]{figs/procimg_perobj}
\end{center}
\caption{Layout of a typical module within the process image.  \textbf{Bold} sections
contain program data; \textit{italicized} ones contain metadata for the runtime.}
\label{fig:procimgobj}
\end{figure}

\begin{promotesubsections}
\begin{swallowsections}
\input[functions]{gotcha_gotcha}
\end{swallowsections}
\end{promotesubsections}
\hspace{-2.5em}
Our discussion in this chapter uses \textit{libinger} as a motivating example of a
\textit{libgotcha} user, as this configuration was the inspiration for the runtime's
creation.  However, we have found that the described techniques to be general and
equally relevant to applications besides timed functions.  As such,
\textit{libgotcha} exposes a general API that allows any \textbf{control library} to
configure its behavior for the process.  We give more details later in the chapter,
and study other examples of control libraries in Chapter~\ref{chap:safety}.


\section{A brief tour of linking}
\label{sec:libgotcha:link}

We begin with background about linking, a two-stage process that ultimately produces
an in-memory \textbf{process image} containing a program's code, all the data it
needs to execute, and the code and data of all its dependencies.  Linking operates on
\textbf{object files} that can take the form of either an \textbf{executable} or a
\textbf{shared library}.  Once a program is running, its process image contains a
region corresponding to each loaded object file.  We will refer to each such region as
a \textbf{module}, regardless of whether it corresponds to an executable or a shared
library.  Each module is divided into logical \textbf{sections}, each containing a
particular type of information.  Figure~\ref{fig:procimgobj} shows a typical module's
layout; notice that it contains both data corresponding to the source code and
generated metadata for runtime consumption.

The linking process occurs in two parts.  Static linking occurs at compile time and
forms the last step of the traditional build process.  Dynamic linking occurs at a
phase of runtime we will refer to as \textbf{load time}, because it starts before the
program has been loaded from disk or the language runtime initialized.


\subsection{Static linking}

Invoking the \texttt{cc} compiler driver does more than just compile C code:\@ it
runs the C preprocessor \texttt{cpp}, the C compiler (\texttt{cc1} in GCC's case),
then the static linker \texttt{ld}.

The output of the second step is a relocatable object file containing code and data
with referenced addresses identified by named \textbf{symbols}.  In a relocatable
object file, symbol \textbf{references} such as instructions making function calls
or accessing global variables are encoded with a null address as a placeholder.  Each
object file contains a \textbf{relocation table} in a separate section that
associates each placeholder with a symbol name, which may or may not be located in
the same file.  Each object file also contains a \textbf{symbol table} to identify
the symbols it defines and associate them with the file offset of their definition.
Note that only non-\texttt{static} C symbols generate global symbol table entries
that can be referenced from other object files; this keyword is confusingly named and
does not refer to static linking.  The compiler's ultimate output is one relocatable
object file for each source file.

The static linker is responsible for combining one or more relocatable object files
into a single executable or shared library, where either type of output file is
ready for loading into memory for execution.  This process consists of verifying that
there is a definition corresponding to each symbol reference, unifying the sections
across object files and choosing a final address (or relative address) for each
symbol, encoding the chosen addresses at the location recorded in each relocation
table entry, and writing the resulting file to disk.  This output file does not
preserve the relocation table because the linker has already fixed the null pointers
it described.  The file does contain a symbol table because it can be useful for
debugging (e.g., to generate stack traces), but this can be removed using the
\texttt{strip} utility without affecting the program semantics.

With the exception of macOS, most modern Unix systems use ELF (Executable and
Linkable Format) object files.  One advantage of this format is that executables and
shared libraries are themselves ELF object files.


\subsection{Dynamic linking}
\label{sec:libgotcha:dylink}

Static linking allows programs to reuse ``libraries'' of precompiled object files,
but each program must be built with its own copy of all its libraries within the
executable.  This means that every time an application is loaded, its libraries' code
and data must be read back from disk, even if another running program uses the same
libraries; it also means that updating a library requires recompiling all dependent
programs installed on the system.  Dynamic linking solves both problems by separating
libraries into separate files that are not read until the executable runs.\footnote{
Specifically, this separation obviates the need to read the files from disk multiple
times because the runtime maps them into the process image using the \texttt{mmap()}
family of system calls.  The kernel tracks regions that are already mapped and serves
recurring requests from memory instead of disk, mapping to the same physical memory
if the pages are read-only or creating copy-on-write page mappings otherwise.}

By splitting libraries into their own files, dynamic linking introduces a build-time
challenge:\@ the relative position and offset of modules cannot be known until
runtime.  As such, rather than performing the relocations for inter-module symbol
references, the static linker leaves the placeholder addresses and adds a separate
dynamic relocation table and dynamic symbol table into the output object file.
Unlike the tables used for static linking, these are needed to launch the program, so
tools such as \texttt{strip} leave them in place.  For executables, the linker also
writes the path to an ``interpreter'' program into the ELF program header.

When asked to load a program that declares an interpreter, the kernel loads and jumps
to the interpreter instead of the executed program.  Usually, this interpreter is the
system \textbf{dynamic linker}, traditionally named \texttt{ld.so}.  Before jumping
into the program code, the dynamic linker loads all the modules and processes the
entries in each of their dynamic relocation tables.  The relocations are not
restricted to modifying writeable memory:\@ they can update constant global data and
even executable instructions.  Even if they leave the code unchanged, its position
relative to the rest of the module matters.  These points are critical to our use
case, as they mean that in order to duplicate modules' data, we must also duplicate
their code.

Another consequence of relocations being able to alter read-only memory is that the
dynamic linker must change the page protections of these regions after it is finished
processing relocations.  To support this, the compiler splits up module components
into fine-grained sections by purpose.  Non-\texttt{const} global variables are
placed in the \texttt{.data} and \texttt{.bss} sections, which must remain writeable
at runtime and therefore require no special action.  In contrast, \texttt{const}
globals are split between the \texttt{.rodata} and \texttt{.data.rel.ro} sections
based on whether they require relocation; in the latter case, the dynamic linker
marks the pages read-only before passing control to the program.

If relocations routinely modified scattered locations throughout the executable
\texttt{.text} section, the dynamic linker would have to change page protections on
most or all of each module's code pages.  This would require a lot of system calls,
but it would also require copy-on-write code mappings, preventing instruction cache
hits between processes using the same library.  To avoid these problems, the compiler
indirects references to dynamic symbols via a structure called the GOT (Global Offset
Table).

\begin{figure*}
\begin{minipage}{\textwidth}
	\includegraphics[width=\textwidth]{figs/gotables-crop}
	\subcaption{Reading a library's global variable: \texttt{size\_t tmp = data;}}
	\label{fig:dytabs:got}
	\end{minipage}

	\begin{minipage}{\textwidth}
	\includegraphics[width=\textwidth]{figs/pltables-crop}
	\subcaption{Calling an eagerly-resolved library function: \texttt{fun()}}
	\label{fig:dytabs:plt}
	\end{minipage}

	\begin{minipage}{\textwidth}
	\includegraphics[width=\textwidth]{figs/jstables-crop}
	\subcaption{Calling a lazily-resolved library function.  In step \textcircled{5},
	the dynamic linker memoizes the resolved address into the GOT; subsequent calls
	proceed as above.}
	\label{fig:dytabs:lazy}
	\end{minipage}
\caption{Table references required to reference global symbols in dynamically-linked
programs}
\label{fig:dytabs}
\end{figure*}

The GOT is a table of relocated pointers to symbol definitions, whether those
definitions are within the same module or in a different one.  To avoid generating
code pages that require relocations, the compiler compiles each reference to or
dereference of non-\texttt{static} global data into a position-independent load of
the corresponding pointer from the GOT.  Figure~\ref{fig:dytabs:got} shows an example
of the two instructions and one table reference needed for a dereference.  A
reference would generate only the first \texttt{mov} instruction, as would taking a
pointer to a function.

Calling a global function works differently and relies on another indirection
structure called the PLT (Procedure Linkage Table), which contains code instead of
pointers.  For each call to a non-\texttt{static} function, the compiler generates a
position-independent call to a PLT entry corresponding to the function being called.
It generates a PLT entry, which is a short sequence of instructions that loads the
pointer to the real definition from the GOT, then executes and indirect jump to that
location.  Figure~\ref{fig:dytabs:plt} shows an example function call.  As with GOT
entries, there are PLT entries for functions defined both within and outside the
referencing module.

Not all function calls are this simple.  To save the dynamic linker some work at load
time, many function calls resolve lazily on their first execution.  Such resolution
involves a series of jumps designed to memoize the address so that subsequent calls
to the function from the same module do not repeat the expensive lookup.
Figure~\ref{fig:dytabs:lazy} shows the effect of the first call to such a
function:\footnote{This representation is slightly simplified for brevity.  In
practice, it is undesirable to hardcode the address of a dynamic linker function into
each module.  Therefore, instead of jumping directly to the symbol resolver, the slow
lookup path jumps to a dedicated PLT stub that loads its address from another GOT
entry.  Technically, there are separate identifiers for the symbol and the module,
each pushed to the stack by one of these two involved PLT stubs.}  \textcircled{1}
The program calls the PLT stub, just as it would for an eagerly-resolved function.
\textcircled{2} The PLT stub is longer (three instructions instead of one), but still
begins with an indirect jump to the pointer found in the corresponding GOT entry.
\textcircled{3} The GOT entry initially contains the address of the PLT stub's second
instruction, so the indirect jump is a no-op and merely advances the instruction
pointer.  \textcircled{4} The rest of the PLT stub pushes a constant identifying the
module and symbol onto the stack, then jumps to a symbol-lookup function in the
dynamic linker.  \textcircled{5} After looking up the address of the symbol's
definition, the dynamic linker uses the identifier from the stack to find and update
the GOT entry in the calling module.  \textcircled{6} The dynamic linker jumps to the
symbol in the defining module.  Because the GOT entry has been updated, future calls
proceed exactly like eagerly-resolved ones and jump directly to the symbol definition
from the first instruction of the PLT stub.  Of course, the GOT entries associated
with lazy PLT stubs must be writeable at runtime; this is why
Figure~\ref{fig:procimgobj} shows the GOT as split between two sections.

The dynamic linker performs all relocations and other standard module setup
automatically at load time, but the initialization process is pluggable.  In
particular, modules can include \textbf{constructor} functions to be invoked before
control is transferred to the runtime and ultimately the program's main function.  As
we will see, our work leverages this feature to override certain relocations at the
conclusion of load time.


\begin{promotesubsections}
\begin{swallowsections}
\input[functions]{gotcha_namespaces}


\input[functions]{gotcha_libsets}

Thus, selective relinking is selective in two ways, only affecting execution when the
next libset differs from the current libset and the program references a dynamic
symbol defined in a module that is not currently executing on that thread.

\solb{Expand interface listing and add comments with section references}
\end{swallowsections}
\end{promotesubsections}


\subsection{Detecting cross-module symbol references}

Identifying which GOT entries correspond to cross-module symbol references is a
multi-step process:
First, we traverse the relocation table for each loaded module, cross referencing
each of its relocation entries against the local module's symbol table.  If the
symbol table does not contain a definition matching the relocation entry's target, we
conclude that the relocation must correspond to a cross-module call.  Otherwise, we
check the address in the GOT entry corresponding to the relocation:  If this address
is outside the memory bounds of the current module, it is a cross-library call.
Otherwise, if this address matches the one from the symbol table entry, it is not a
cross-library call, and should be skipped.  The trickiest case is when the GOT entry
does not match but does point somewhere within the current module, since this means
it probably still refers to the PLT stub (because the symbol reference is lazy and
has not yet been resolved, as covered at the end of
Section~\ref{sec:libgotcha:dylink}).  In this case, we resolve the symbol early,
update the GOT entry, and recheck whether it resolved to the local definition to
determine whether it is a cross-module call.


\begin{promotesubsections}
\begin{swallowsections}
\input[functions]{gotcha_init}

If a preemptible function is canceled rather than being allowed to return, execution
might be interrupted within a call to a library function.  For this reason,
\textit{libgotcha} must treat the libset's shared state as corrupted; it provides the
\texttt{libset\_reinit()} function shown in Listing~\ref{lst:gotchaapi} to allow
control libraries to inform it of such a situation so it can \textbf{reinitialize}
the libset before returning it to the pool.

Our early approach to reinitialization was to unload and reload all objects in the
libset by calling \texttt{dlclose()} followed by \texttt{dlmopen()}.  While this
approach theoretically allowed us to delegate the work to the dynamic linker, in
practice it introduced significant complications.\footnote{Most notably, some shared
libraries are marked with a special configuration flag, \texttt{DF\_1\_NODELETE},
which prevents the dynamic linker from ever removing them once they have been loaded.
Because almost all libraries depend on libc, the presence of even one such library
would prevent us from reinitializing a libset.  The flag is mostly used on libraries
that need to monkey-patch some other loaded library, such that the two subsequently
have a circular dependency.  Fortunately, this was not generally a problem for us
because when we unload one library from a libset, we then unload the rest.  Whenever
we encountered a \texttt{NODELETE} object file, we would make a special copy with the
flag cleared, for loading into every namespace except the main one.}  Worse, it
required the dynamic linker to reprocess all relocations throughout the libset, which
introduced prohibitive runtime latency.  We measured reinitialization taking almost 5
ms (over 10 million cycles on modern processors) on even small minimal example
programs~\cite{boucher:atc2020}.  With such delays, the only reasonable way for the
control library to handle cancellation was to delegate the reinitialization to a
separate thread to take it off the critical path; of course, this approach only works
as long as the number of libsets is not a bottleneck.

We have since redesigned reinitialization around a significantly faster approach:\@
checkpointing only portions of each module.  The key insight is that, as we saw in
Section~\ref{sec:libgotcha:link}, only some sections are writeable at runtime.  We
can therefore assume that these are the only memory regions of each module that can
change.  After populating the libset pool at application start, \textit{libgotcha}
iterates through each module of each libset and makes a backup copy of all its
writeable regions.  When a control library calls \texttt{libset\_reinit()},
\textit{libgotcha} restores each such region from the backup before returning the
affected libset to the pool.  We summarize this approach, which has reduced the
latency of reinitialization by two orders of magnitude, in
Figure~\ref{fig:reinit}.\footnote{We will address the version watermark alluded to
therein later in this chapter.}  To avoid having to repeat relocations and rerun any
module constructors, we capture the backup after dynamic relocation is complete and
all constructors have run; the tradeoff is that we actually have to copy memory,
rather than leveraging copy on write to later restore to the version on disk.

\begin{figure}
\includegraphics[width=\columnwidth]{figs/reinit-crop}
\caption{Libset reinitialization to support asynchronous cancellation}
\label{fig:reinit}
\end{figure}


\input[functions]{gotcha_goot}

\solb{\textbf{Subsections?}}

\input[functions]{gotcha_plot}
\hspace{-1.5em}
\input[functions]{gotcha_lazy}

\solb{Rewrite this now that we already describe lazy relocations earlier}

\input[functions]{gotcha_globals}

\solb{\textbf{Subsection on thread-local storage}}

\solb{Diagram of thread-local data layout}

\solb{Reinitialization diagram (TLS part)}


\input[functions]{gotcha_uncopyable}

\solb{Add the effects in the above TODO to the UML diagram?}

\solb{Stipulate that libgotcha itself is always uncopyable}

\solb{Say that the hook function runs in the interrupted module's namespace}

\solb{Mention pre-call hooks}

\solb{Give the limitations of each type of hook}

\solb{\textbf{Section More on control libraries}}

\solb{Types of control libraries from the frontmatter}

\solb{Address monomorphization}

\solb{Monkey patching of ld.so and library header rewriting trick}


\input[functions]{gotcha_tls}

\solb{Drop TLS stuff, incorporating whatever is salvageable earlier}

\input[functions]{gotcha_linker}

\solb{Recompiling glibc from the frontmatter}

\input[functions]{gotcha_relocations}

\end{swallowsections}
\end{promotesubsections}


\section{Evaluation}

\input[functions]{eval_ugotcha}

\input[functions]{eval_testbed}

\solb{Thread creation, TLS allocation, and libtlsblock}

\chapter{Rethinking POSIX safety: \\ \textit{libas-safe} and \textit{libac-safe}}
\label{chap:safety}

\ifdefined\chapquotes
\vspace{-1in}
\begin{chapquote}[1.75in]{Douglas Adams, \textit{The Hitchhiker's Guide to the Galaxy}}
`Ah!  This is obviously some strange use of the word \textit{safe} \\
that I wasn't previously aware of.'
\end{chapquote}
\fi

\chapter{Function calls with timeouts, revisited: \\ the \textit{libinger} library}
\label{chap:libinger}

\ifdefined\chapquotes
\vspace{-0.5in}
\begin{chapquote}[1.5in]{David Mitchell, \textit{Cloud Atlas}}
A half-read book is a half-finished love affair.
\end{chapquote}
\fi

\solb{Reflow this paragraph because we moved safe concurrency to after it}

Chapter~\ref{chap:functions} motivated the need for lightweight preemptible functions
and introduced the shared state problem.  Therein, we concluded that explicitly
handling state shared between a preemptible function and the rest of the program must
necessarily be the responsibility of the programmer using the preemptible function,
as is the case for any concurrency abstraction.  However, we also reasoned that
requiring the programmer to reason about all shared state in the program and its
dependencies was not a tenable approach to solving the other half of the problem.
Instead, we introduced the \textit{libgotcha} runtime for automatically establishing
memory isolation boundaries within the process.  Now that we have covered how said
runtime works in Chapter~\ref{chap:libgotcha}, we are ready to discuss the
implementation of the preemptible function abstraction itself.

In this chapter, we turn our attention to the \textit{libinger} library that provides
the wrapper functions for invoking and working with preemptible functions.  We begin
by showing a summary of the library's API surface, expanded from that first presented
in Section~\ref{sec:libinger}.  Figure~TODO shows the C interface; the Rust interface
differs in a few ways:

\thesis{Add listing of the interface, including \texttt{cancel()} and \texttt{pause()}.}

\thesis{Maybe list it in both languages so we can refer back to it for safety discussion}

\begin{figure}
\begin{lstlisting}[label=lst:ingerfullapi,caption=Preemptible functions extended interface]
struct linger_t {
	bool is_complete;
	cont_t continuation;
};

linger_t launch(Function func,
                  u64 time_us,
                  void *args);
void resume(linger_t *cont, u64 time_us);
\end{lstlisting}
\end{figure}

\paragraph{Closure support.}
We leverage Rust's first-class closures to enable the caller to pass
\texttt{launch()} a function that captures state from its environment.  We expect the
caller to provide any inputs to the function in this manner, so there is no need to
wrap the arguments or pass an empty value when the function expects no inputs.


\paragraph{Type safety.}
The aforementioned interface change means that the Rust wrapper functions do not
erase the types of the preemptible function's parameters by diluting them to a
\texttt{void *}; thus, the preemptible function does not have to perform an unsafe
cast before using them, and any type mismatch is caught at compile time.
Furthermore, both \texttt{launch()} and \texttt{resume()} are generic on the
preemptible function's return type:\@ instead of a \texttt{linger\_t}, they return a
tagged union.  Once the preemptible function runs to completion, the caller may
destructure this type to retrieve the function's return value.


\paragraph{Safe concurrency.}
The \texttt{launch()} function requires that the preemptible function closure
implement the language's automatic \texttt{Send} trait.  This means that any attempt
to pass a closure that shares external state without the use of concurrency control
will cause a compilation failure.


\paragraph{Flexibility.}
The requirement that Rust preemptible functions be \texttt{Send} is similar to the
restrictions imposed by the standard library's thread \texttt{spawn()} interface.
However, \texttt{launch()} differs in an important way:\@ unlike a thread, a
preemptible function is not restricted to the \texttt{'static} lifetime, and so is
able to accept references to local variables and other dynamically-allocated data.
Attempts to transfer such references to a thread would result in a compile error.
The difference that makes it safe to use all lifetimes with preemptible functions is
the fact that they execute synchronously, and therefore cannot outlive the calling
context without becoming paused.  If a preemptible function times out, the Rust
compiler knows the lifetime of any references it has captured, so any attempt to pass
the paused closure to a scope where its shared data no longer exists will be met with
a compile error.


\paragraph{RAII.}
Our Rust interface adheres to the RAII (Resource Allocation Is Initialization) idiom,
allowing continuation deallocation to happen automatically.  Unlike the C interface,
the Rust one has no \texttt{cancel()} function; instead, its continuation objects
implement the \texttt{Drop} trait, and their destructor performs cancellation
whenever they go out of scope.
\\

We now turn our attention from how one interacts with \textit{libinger} to how it
works.

\solb{Interface is generic}

\solb{Use of Rust enum (tagged union)}


\begin{promotesubsections}
\input[functions]{inger_pause}
\end{promotesubsections}

\solb{Reflow because this was moved from another chapter}

\solb{Be sure to emphasize that concurrency has always been hard}

\solb{Ability to query whether it yielded}

\solb{\textbf{(Sub)section on custom mutex idea}}


\begin{promotesubsections}
\input[functions]{inger_stacks}

\solb{Mention that this preallocation is not fundamental}

\solb{Cover cool disaggregation trick for growable stacks}

\solb{\textbf{Section on POSIX contexts} (which might have once existed in paper)}

\thesis{Give a tour of \textit{libtimetravel}?}

\solb{Fix (mis)uses of the term continuation}


\input[functions]{inger_interrupts}
\end{promotesubsections}

\solb{\textbf{Section on deferred preemption}}

\solb{Include specialized version UML call diagram}


\section{Cancellation}

Should a caller decide not to finish running a timed-out preemptible function, it
must deallocate it.  In Rust, deallocation happens implicitly via the
\texttt{linger\_t} type's destructor, whereas users of the C interface are responsible
for explicitly calling the \textit{libinger} \texttt{cancel()} function.

As discussed in Chapter~\ref{chap:libgotcha}, \textit{libgotcha} returns a
preemptible function's libset to the pool for reuse when that function returns
normally.  However, when a function is cancelled before it finishes, none of the
modules in its libset is safe to reuse in general:\@ a library function might have
been in the middle of executing.  To avoid future problems with the libset, as part
of a cancellation, \textit{libinger} instructs \textit{libgotcha} to reinitialize the
function's libset before returning it to the pool.

Cancellation cleans up \textit{libinger} resources allocated by \texttt{launch()};
however, the current implementation does not automatically release resources already
claimed by the preemptible function itself.  However, we have prototyped a solution
that demonstrates the feasibility of such cleanup in RAII languages such as Rust.
We detail this approach in Chapter~\ref{chap:ingerc}.


\section{Evaluation}

\input[functions]{eval_testbed}


\subsection{Microbenchmarks}

\input[functions]{eval_uinger}


\subsection{Image decompression}

Unlike state of the art approaches, lightweight preemptible functions support
cancellation.
\begin{sloppypar}
\input[functions]{eval_cancel}
\end{sloppypar}

\chapter{Resource cleanup and async unwinding: \\ the \textit{ingerc} compiler}
\label{chap:ingerc}

\ifdefined\chapquotes
\vspace{-1in}
\begin{chapquote}[1.75in]{J. R. R. Tolkien, \textit{The Fellowship of the Ring}}
And with that Gl\'oin embarked on a long account of the doings of the \\
Dwarf-Kingdom.  He was delighted to have found so polite a listener...
\end{chapquote}
\fi

As described so far, one of the facilities that \textit{libinger} enables is
asynchronous function cancellation.  As we saw in Chapters~\ref{chap:functions} and
\ref{chap:safety}, this is a significant achievement that is only possible under the
POSIX safety model thanks to selective relinking.  However, one missing piece of
functionality is automatic cleanup of any resources the cancelled function had
allocated.

The resource leaks associated with cancelling a function are a significant problem:\@
they make cancellation infeasible for long-running applications, which would
experience the cumulative leakage of the resources allocated by all such cancelled
functions.  While a garbage collector would be able to find the leaked resources,
deallocating them might still prove challenging because, without a record of the
interruption point where cancellation occurred, it would not be safe to run object
finalizers.  Of course, our system targets unmanaged languages, so we must accomplish
resource cleanup without a garbage collector.


\section{Languages with unstructured resource management}

In languages such as C, resource lifetimes are completely unstructured, with each
allocation and deallocation performed via an ad-hoc function call.  Some such
functions are well-known because they are prescribed by the C and/or POSIX standards:
\texttt{malloc()}/\texttt{free()}, \texttt{open()}/\texttt{close()}, etc.  However,
applications and libraries can provide their own resource-allocation interfaces, so
it is not possible to identify or track resource management in general.  Worse, there
is no standardization of deallocation functions' interface.  These language
properties mean that automating cleanup would require hand-annotating all custom
allocation and deallocation functions throughout the application and its
dependencies; such annotations would have to provide associations between each
allocator and its corresponding deallocator, as well as information about how to call
the latter.

Were one to build a system to support this, one would need to use an approach like
that of Valgrind's Memcheck~\cite{seward:usenix2005} or LLVM's
AddressSanitizer~\cite{serebryany:usenix2012}, which
instrument the application's allocation and deallocation calls.  Neither system could
be imported wholesale:  Both assume at a design level that memory is the only
resource whose allocations are being tracked.  Valgrind depends on expensive dynamic
instruction translation that is not suitable for production use.  AddressSanitizer
does not track how each resource was allocated unless paired with the separate
MemorySanitizer system~\cite{stepanov:cgo2015} run in origin-tracking mode; this adds
another 2--7x slowdown on top of AddressSanitizer's own 2x execution time penalty.

For rolling one's own allocation and deallocation tracker, \textit{libgotcha}'s
existing ability to intercept function calls might prove useful.  The bookkeeping
structures would need to be mutable, so care would have to be taken to avoid
designing around data structures with
amortized time complexities, as this would introduce undesirable unpredictable pauses
in preemptible function execution reminiscent of garbage collection.\footnote{The
\textit{libgotcha} runtime itself does not suffer from this problem because its
symbol lookup tables are immutable once process initialization is complete.}  For
instance, storing allocation records in a hash table would require periodic
rebalancing.

Because of the above limitations, we have not pursued automatic resource cleanup for
preemptible functions written in C.  Developers of long-running C applications should
always write a cleanup handler for each preemptible function they might need to
cancel (Section~\ref{sec:libinger:cancellation}).


\section{Languages following the RAII principle}
\label{sec:ingerc:raii}

The situation is more promising in Rust.  Like C++, it adheres to the RAII (Resource
Allocation Is Initialization) idiom that associates each resource's lifetime with
that of some object.  Whenever an object goes out of scope, the program invokes its
destructor and those of its members, freeing the associated resources.  Thus, the
problem of releasing the resources associated with a cancellation can be reduced to
that of invoking the destructors of the objects that are alive at the interruption
point.  Notice that, in contrast to garbage collection, such a model does not divorce
the problem of deallocation from the cancelled function's code; as such, it is not
subject to the safety problems of invoking finalizers, as only the destructors of
objects whose initialization is already complete can be invoked.

Faced with the challenge of safely preempting in the presence of shared state caused
by nonreentrant library interfaces, we found that we could leverage dynamic linking
to solve the problem automatically, and built the \textit{libgotcha} runtime to do
just that.  Here again, we are fortunate to find an existing runtime facility that
can be repurposed to call destructors at an arbitrary position in the program:\@
the Rust language already supports exceptions (which it calls ``panics'').  One
significant advantage to building on top of exceptions rather than implementing
separate resource tracking is that exception handling is already designed to add no
overhead to the non-exceptional execution path.  With the exception of adding one
function call to each function that owns objects with destructors, we believe it is
possible to provide automatic cleanup without imposing runtime overhead on tasks that
are never cancelled.


\section{A brief tour of exception handling}

Whenever a program throws an exception, the language runtime must find the point in
the program that handles that exception.  To prevent resource leaks, deadlocks, and
other bugs, it must then invoke the destructors of all objects that are in scope at
the point where the exception was thrown, but out of scope at the point where it is
caught.  This feature of exception handling is perfectly suited to our use case.

It is possible for a function to throw an exception that is then caught by one of its
callers, so the language runtime must be able to ``unwind'' the stack, locating the
stack frame of each function's caller.  Code for the x86 architecture used to
maintain a frame pointer that made it easy to find the bounds of a function's stack
frame, but with the advent of x86-64, this is no longer standard; thus, the runtime
needs some other way to find the next frame.  Debuggers have long faced this very
problem on other architectures, and the common approach is to rely on extra debugging
information stored in the executable or library on the disk.  On Unix operating
systems, most debuggers use the CFI (Call Frame Information) facility of the standard
DWARF debugging format~\cite{eager:spec2012}.

Modern exception runtimes repurpose this debugging information to unwind the stack
once an exception has been thrown.  The compiler produces the requisite information
by generating CFI pseudoinstructions, which the assembler then transcribes into DWARF
format and stores in the \texttt{.eh\_frame} section of the object file.  This
section is present in non-debug builds and stripped object files and gets loaded into
the process's memory image by the ELF loader or dynamic linker, in contrast to the
CFI's more traditional home, the \texttt{.debug\_frame} section.  With the complexity
of this approach comes the advantage that the application no longer has to update
frame pointers during normal execution.

Call Frame Information alone is not a sufficient primitive to implement exception
handling:\@ the runtime must also be able to find the exception handler(s) present in
each call frame and the destructors to invoke based on where in the function the
exception was thrown.  The compiler must supply this information, which it does by
emitting pseudoinstructions that describe a metadata region known as the LSDA
(Language-Specific Data Area); the assembler stores this in the object file's
\texttt{.gcc\_except\_table} section.  For each function, the LSDA contains a table
mapping instruction address ranges to landing pads, code regions within the function
that serve either to catch exceptions or to invoke destructors.  Our discussion will
focus on the latter type, known as cleanup landing pads.


\section{Asynchronous exception handling}
\label{sec:ingerc:async}

Because exceptions are generated synchronously, they can only occur on calls to
functions that can throw.  Since compilers know which functions can throw, they
generally only output LSDA entries that are accurate for those functions' call sites.
But since \textit{libinger} interrupts functions preemptively, we need to trigger
unwinding and cleanup at whatever arbitrary point the function was paused at before
being cancelled.

Triggering unwinding is a simple matter of tweaking the stack pointer and
instruction pointer of the preemptible function to be cancelled in order to forge a
call to a function that raises an exception using Rust's \texttt{panic!()} macro.
But providing instruction-accurate cleanup information requires us to address the
following challenges:
\begin{enumerate}
\item \textbf{Optimized builds remove some functions' LSDA tables and landing pads.}
We have noticed that enabling optimizations via the Rust compiler's \texttt{-O}
switch causes some functions that have exception-handling support in debug builds to
instead be compiled without it.  We describe our workaround for this issue in
Section~\ref{sec:ingerc:skip}.

\item \textbf{Functions that ``return'' values via pointer parameters lack
exception-handling information.}  We have noticed that such ``sret'' functions tend
to lack any exception information at the LLVM IR level, even if they operate on
objects with destructors.  This is a problem because, although the objects exist in
the caller's stack frame, they must still be treated as owned by the function that is
``returning'' them, so that we will clean them up if cancellation occurs between the
time they are allocated and that function returns.  Such functions are more common
than one might expect and include most constructors:\@ the Rust compiler prefers to
compile functions that return large objects in this manner to avoid moving them to
the caller's stack frame immediately afterward.  See
Section~\ref{sec:ingerc:optimization}.

\item \textbf{Many LSDA entries associate the landing pad with too few instructions
following or preceding a function call site.}  Injecting
an exception in such execution regions results in leaks or deallocating before
allocation, respectively.  Our investigation revealed that these discrepancies result
from changing instruction boundaries during lowering from LLVM IR to the platform's
assembly language; in particular, the backend does not account for the \texttt{mov}
and \texttt{lea} instructions that perform argument passing before most
\texttt{call}s. See Section~\ref{sec:ingerc:codegen}.

\item \textbf{The runtime does not discriminate between being in the middle of
executing a function and having just retired its \texttt{ret} instruction and jumped
back to the call site.}  In either case, it will not invoke any cleanup landing pads
in the caller.  The two scenarios are indistinguishable under the assumption that no
exception can occur at these points in the function.  However, the fact that we can
inject one there creates an important distinction for our purposes:\@ until the
function returns, it still has ownership of its live variables and its landing pads
are responsible for cleaning them up, whereas after it has returned, it is impossible
for those landing pads to be invoked and cleanup must necessarily be up to the
caller.  See Section~\ref{sec:ingerc:return}.

\item \textbf{Unwinding on the first instruction of a function fails because the
runtime consults the LSDA table for the function whose definition precedes it in
memory.}  This issue turns out to have the same cause as the previous one, but the
two situations demand different solutions.  See Section~\ref{sec:ingerc:start}.

\item \textbf{Performing function calls during the prologue or epilogue of a function
is unsafe.}  The x86-64 ABI (Application Binary Interface) specifies that the stack
must always be 16-byte aligned before calling a function, which is not true until the
function has reserved space for an odd number of 8-byte values (excluding the return
address) in its stack frame.  See Section~\ref{sec:ingerc:realign}.

\item \textbf{If attempted on the instruction just after one that repositions the
stack pointer, unwinding miscalculates the frame address.}  While this behavior
appears consistent between the libgcc and libunwind (LLVM) unwind implementations, we
suspect it exists because exceptions ordinarily never occur in the prologue or
epilogue of the function.  GCC has an \texttt{-fasynchronous-unwind-tables} switch
that is intended to make the frame information accurate down to the instruction, but
Clang only includes this switch for command-line compatibility and doesn't actually
implement this feature.  As a likely consequence of this lack of support from the
LLVM project itself, the Rust compiler also makes no attempt to offer it.  See
Section~\ref{sec:ingerc:boundary}.

\item \textbf{Cleanup landing pads do not work reliably if associated with the
function epilogue.}  This happens because the epilogue adjusts the stack pointer,
in many cases causing any synthetic function call (e.g., to inject an exception) to
clobber the very stack values the landing pad is trying to clean up.  Incidentally,
a Web search for ``LLVM unwind function epilogue'' reveals that the unwind info is
not trustworthy during the epilogue in the general case.  Indeed, there have been
several patchsets attempting to fix this, some of which were merged, but each of
which was subsequently reverted for breaking some other architecture.  So it would
appear not only that this is the primary design issue blocking LLVM support for
asynchronous unwind tables, but also that we must avoid injecting exceptions in
epilogues altogether.  See Section~\ref{sec:ingerc:epilogue}.

\end{enumerate}

Rather than integrate a fully general resource cleanup solution into
\textit{libinger}, we have prototyped the components to solve these problems and used
these to build a proof of concept implementation of the compiler transformations
necessary to support asynchronous exception handling.  This prototype represents
preliminary evidence that our approach is feasible, although it would take additional
engineering effort to achieve compatibility with nontrivial applications.

The below numbered sections describe our approach to solving each of the
challenges listed above.  The product of the work described in this section is a
shell script, \textit{ingerc}, that wraps rustc and applies all the described
transformations to produce an output program or library ready for runtime-assisted
cancellation cleanup.


\subsection{Skipping optimization passes that remove exception handling}
\label{sec:ingerc:skip}

Testing with rustc 1.56.0, we have found that the \texttt{prune-eh},
\texttt{function-attrs}, and \texttt{inline} LLVM optimization passes are
responsible for stripping the LSDA tables and landing pads from some functions in
optimized builds.  We have developed a shell script to invoke rustc without these
passes, a task that is unfortunately complicated by the compiler's command-line
interface, which only accepts a list of all the passes to run.

We recognize that disabling the inlining pass is likely to reduce the efficiency of
compiled code, but we leave it to future work to investigate why this pass is
removing exception information from functions otherwise unaffected by inlining.


\subsection{Adding exception-handling support to functions' LLVM IR}
\label{sec:ingerc:optimization}

The above script does not address functions for which the compiler emits no
exception-handling information even in debug builds.  As before, this problem is
easiest to address in the intermediate representation, where the addition of an
exception-handling personality and a \texttt{landingpad} instruction will cause the
LLVM backend to emit an LSDA table and landing pad for the function.

To reduce implementation complexity, we do not attempt to detect which functions own
objects with destructors, and instead introduce exception handling into any functions
that do not already have it.  This saves us from having to query complex properties
of the IR and reduces our task to one of simple text transformations.  We implement
these in a TypeScript script performing regular expression replacements.

The landing pads we insert at this stage are empty skeletons that do not actually
invoke any destructors.  We describe how we identify which destructor(s) to invoke
(if any) and add the calls at the end of Section~\ref{sec:ingerc:codegen}.


\subsection{Adjusting LSDA entries}
\label{sec:ingerc:codegen}

The possibility that cancellation injects an exception during the argument-passing
instructions
preceding a call violates a design assumption of LLVM's LSDA generation.  IR
instructions such as \texttt{call} often expand to multiple machine instructions,
most commonly to perform argument passing before the function dispatch.  However, the
backend generates the address ranges for LSDA entries using labels in the IR.  This
means that ordinary optimization and transform passes cannot associate landing pads
with some but not all of the instructions comprising a function call sequence.

To get around this problem, we had to implement a plugin that loads a code generation
pass into \texttt{llc}, the LLVM static compiler.  The pass works at the x86-64
machine instruction level to reposition LSDA-related labels and resize the code
region
on the normal execution path with which a cleanup landing pad is associated.  To
prevent leaks, if the ending label falls before a destructor call, the pass moves it
downward
to just before the machine \texttt{call} instruction; otherwise, the pass moves it
downward
to just before the function epilogue.  To prevent issuing destructor calls before
construction, if the starting label falls before the function call that produces the
object to be cleaned up, the pass moves it downward to just after that call.

The pass also identifies functions with parameters annotated as
\texttt{sret} in the LLVM IR.  These correspond to functions where the script from
Section~\ref{sec:ingerc:optimization} added synthetic landing pads.  The pass checks
to see
whether the involved type(s) have destructors; if so, it adds destructor calls to the
landing pad.


\subsection{Detecting whether a function has just returned}
\label{sec:ingerc:return}

Regardless of whether a function has returned, LLVM's libunwind treats the caller
frame as sitting within the call instruction, rather than on the subsequent
instruction located at the return address.  Here is the offending libunwind code:
\begin{lstlisting}
  // If the last line of a function is a "throw" the compiler sometimes
  // emits no instructions after the call to __cxa_throw.  This means
  // the return address is actually the start of the next function.
  // To disambiguate this, back up the pc when we know it is a return
  // address.
  if (isReturnAddress)
  	--pc;
\end{lstlisting}

Since we propose to inject the exception by forging a function call, libunwind always
assumes
the frame where we did this is a return address and performs the decrement.  The
obvious workaround would be to remove this code from libunwind, at the cost of
potentially breaking unwinding through C++ code that might be present in the program.
Indeed, a glance through the disassembly of the Rust standard library shows that
rustc emits \texttt{ud2} (invalid) instructions following the call sites in the
scenario described in the comment, so Rust code is unaffected by the problem.

However, it turns out that the above code has another important effect beyond that
documented in the comment:\@ it avoids running the cleanup landing pad associated
with the code region following the call site if the called function was still
executing at the time the exception was thrown.  This is essential because in this
case, the called function still has ownership over any objects requiring cleanup, and
their state is undefined from the perspective of the caller.  The safe and correct
thing for the runtime to do is to invoke the called function's landing pad but not
the caller's.

For this reason, we need to override this libunwind behavior only at the instruction
where we injected the exception, and only if that instruction immediately
follows a call (so that libunwind would confuse the situation with one
where the called function was still executing).  We propose to accomplish that by
applying
a heuristic-based tweak just before the \textit{libinger} cancellation code injects
the exception call:\@ if the instruction pointer is equal to the 8-byte value offset
--8 bytes from the stack pointer, the function return just completed and we should
add one to the instruction pointer to counteract the described libunwind behavior for
this stack frame only.


\subsection{Unwinding from the first instruction of a function}
\label{sec:ingerc:start}

Another consequence of the libunwind implementation detail described in
Section~\ref{sec:ingerc:return} is that unwinding with the instruction pointer
positioned on the first instruction of a function results in an address associated
with the preceding function in memory.  Therefore, the runtime does not find the
correct LSDA for the function (if it finds one at all).

It is hard to detect this problem without consulting the LSDA, so we implement a fix
by patching libunwind, which already decodes this information.  We insert a check
whether the current instruction pointer falls at the very beginning of its function;
if so, we set the \texttt{isReturnAddress} flag to skip the instruction pointer
adjustment.


\subsection{Calling functions when the stack is misaligned}
\label{sec:ingerc:realign}

Our standard cancellation response of forging a call to a function that panics
causes crashes when the stack is misaligned, as during a function
prologue.  Fortunately, we can easily solve this by instead selectively calling a
function that does not allocate any space in its own stack frame.  This has the
effect of restoring the stack alignment (because of the return address pushed by the
\texttt{call} instruction) before calling any complex code that relies on alignment.
Crucially, it does so without introducing any invalid stack frames that would break
unwinding.  The following function suffices:
\begin{lstlisting}[language={[x86masm]Assembler},morekeywords=ud2]
  	.globl realign
  realign:
  	.cfi_startproc
  	call panic
  	ud2
  	.cfi_endproc
\end{lstlisting}

\noindent
(where \texttt{panic} is the function that would ordinarily inject the exception).


\subsection{Unwinding after an instruction that moves the stack pointer}
\label{sec:ingerc:boundary}

To compute the offset from the stack pointer to the return address, libunwind
contains a function called \texttt{parseFDEInstructions()}.  It loops through the CFI
instructions in the \texttt{.eh\_frame} section, continuing as long as
\texttt{codeOffset < pcoffset} to process the stack pointer adjustments for the
instructions that have executed so far.  Unfortunately, this appears to fall one CFI
instruction short when the instruction that has just retired repositioned the stack
pointer.  Changing the \texttt{<} to a \texttt{<=} fixes the problem.

Section 6.4.3 of the DWARF specification~\cite{www-dwarf-spec} seems to agree with
this sign change.  We hypothesize that libunwind inherited this off-by-one error from
libgcc in its effort to replicate the older library's behavior.  The libunwind test
suite continues to pass after making the change, suggesting that a noncomprehensive
test suite has
allowed the mistake to avoid detection.  Furthermore, Clang's lack of support for
asynchronous unwind tables has probably prevented the community from encountering the
unwind failures we have.


\subsection{Unwinding in the epilogue of a function}
\label{sec:ingerc:epilogue}

As discussed in Section~\ref{sec:ingerc:async}, function epilogues are perilous for
exception injection, and even unwinding in them is currently unreliable.  To work
around these limitations, we have developed our own compiler-assisted runtime
component.

That epilogues pop values off the stack might suggest that it is no longer possible
to clean up a function's resources once its epilogue has started executing;
fortunately, they have an important property that refutes this intuition.  Although
the epilogue removes elements such as saved register values from the stack, it only
moves the stack pointer and does not overwrite the contents of the stack frame.
Thus, if we could undo the epilogue's effects, we could then inject an exception and
it would be handled as if the epilogue had never executed at all.  This is precisely
our approach.

To support time traveling backward to just before a function's epilogue, we need the
program to record its instruction pointer just before it enters the epilogue.  We
accomplish this by having the script introduced in
Section~\ref{sec:ingerc:optimization} and our LLVM pass cooperate to insert a call to
a custom function, \texttt{ingerc\_epilogue\_start()}, before each function's
epilogue(s).

In addition to saving the instruction pointer, \texttt{ingerc\_epilogue\_start()}
informs \textit{libinger} that an epilogue is currently running, so that cancelling a
preemptible function will not inject an exception in the usual way.  This means that
we must also be able to inform \textit{libinger} once the epilogue is finished, so
\texttt{ingerc\_epilogue\_start()} overwrites the function's return address with the
location of another function, \texttt{ingerc\_epilogue\_end()}, that performs this
notification before returning to the real return address.

\begin{sloppypar}
There are a few other values we need to save before starting the epilogue:  (1)~When
\texttt{ingerc\_epilogue\_end()} runs, it will need to know the function's original
return address.  (2)~If either of the functions we introduce at the beginning and end
of the epilogue is running when the function is cancelled, the stack pointer will be
different than at the point we intend to travel to, so we always store the original
stack pointer as well.  (3)~To restore the return address in
\texttt{ingerc\_epilogue\_end()} or when time traveling, we need the frame pointer.
The frame address is not normally accessible at runtime, so
we have our LLVM plugin insert code to pass it to \texttt{ingerc\_epilogue\_begin()}.
\end{sloppypar}

The functions we introduce run just before the function returns, so they must not
overwrite any return registers.  Because of this, we have hand coded them in
assembly.  We give their implementations in Listing~\ref{lst:epilogue}.  The
\texttt{ingerc\_epilogue\_begin()} function saves all values into globals so they are
accessible by the runtime.  This allows us to use the stored return address
(\texttt{ingerc\_epilogue\_ra}) to notify the runtime that the epilogue is currently
executing, so we are careful to set that last and reset it to null in
\texttt{ingerc\_epilogue\_end()}.

\begin{figure}
\begin{lstlisting}[label=lst:epilogue,language={[x86masm]Assembler},caption={[Code to support time travel out of the epilogue]Code to support time travel out of the epilogue. The \texttt{@gotpcrel} relocations are position-independent GOT lookups of the globals' addresses.}]
	.globl	ingerc_epilogue_start
ingerc_epilogue_start:
	# Save our return address as the destination instruction pointer.
	mov	ingerc_epilogue_ip@gotpcrel(%rip), %rsi
	mov	(%rsp), %rcx
	mov	%rcx, (%rsi)
	# Save the stack pointer as it was before we were called.
	mov	ingerc_epilogue_sp@gotpcrel(%rip), %rsi
	lea	8(%rsp), %rcx
	mov	%rcx, (%rsi)
	# Save the frame pointer, which we received as an argument.
	mov	ingerc_epilogue_fp@gotpcrel(%rip), %rsi
	mov	%rdi, %rcx
	mov	%rcx, (%rsi)
	# Save the return address of our caller.
	mov	ingerc_epilogue_ra@gotpcrel(%rip), %rsi
	mov	(%rdi), %rcx
	mov	%rcx, (%rsi)
	# Make our caller return to ingerc_epilogue_end().
	mov	ingerc_epilogue_end@gotpcrel(%rip), %rsi
	mov	%rsi, (%rdi)
	# Return.
	ret

	.globl	ingerc_epilogue_end
ingerc_epilogue_end:
	# Put the original return address on the stack.
	mov	ingerc_epilogue_ra@gotpcrel(%rip), %rsi
	mov	(%rsi), %rdi
	push	%rdi
	# Clear the saved return address.
	xor	%rcx, %rcx
	mov	%rcx, (%rsi)
	# Return to the original caller.
	ret
\end{lstlisting}
\end{figure}

\begin{sloppypar}
There is one other thing that these functions have to be careful about:\@
if cancellation occurs while they are running but outside the region where
\texttt{ingerc\_epilogue\_ra} is set, unwinding must be safe and invoke the correct
cleanup landing pads.  This is why \texttt{ingerc\_epilogue\_end()} returns to the
real caller using a \texttt{push} and a \texttt{ret} instead of an unconditional
branch.  The \texttt{ingerc\_epilogue\_start()} implementation is compatible with
ordinary unwinding, but its invocation can be troublesome because it is
intentionally the last instruction before the epilogue.  This would mean that cleanup
in the caller frame would invoke the epilogue's landing pad, but it never has one.
To prevent a leak in this situation, our LLVM plugin inserts a \texttt{nop}
instruction after the call and before the epilogue label, in order to associate the
function's return address with the landing pad for the
preceding basic block.
\end{sloppypar}


\section{Preemptible function cancellation}
\label{sec:ingerc:cancellation}

While we have not integrated resource cleanup support into \textit{libinger}, our
work on asynchronous exception handling suggests a design.  We have prototyped the
approach in isolation using a set of scripts that use GDB to interrupt execution
after an arbitrary number of instructions have retired and inject an exception at
that point.  In this section, we give the algorithm and how it would integrate with
the existing \textit{libinger} codebase.  We conclude by reasoning about its
correctness based on where the preemptible function is in its execution at the time
it is cancelled and discussing performance considerations.

\begin{sloppypar}
Section~\ref{sec:ingerc:async} introduced our fundamental approach to asynchronous
cleanup:\@ the runtime should inject a synthetic exception at an arbitrary point in
the preemptible function.  It also presented a compiler wrapper script,
\textit{ingerc}, that applies a series of transformations to the code to make this
safe and correct.  In this section, we assume that the preemptible functions being
cancelled are written in Rust, and that the application and all its Rust dependencies
have been compiled with \textit{ingerc} instead of rustc.  We believe the latter
requirement is reasonable because the Cargo build system already expects to have the
source of all dependencies available.  Indeed, new languages such as Rust and Go
follow a growing trend of having unstable ABIs that preclude linking against
precompiled build artifacts generated by a different compiler version.
\end{sloppypar}

Whereas the \textit{libinger} C bindings implement the \texttt{cancel()} operation as
a standalone function, the Rust interface performs cancellation in the destructor.
Whenever a paused preemptible function goes out of scope, the destructor notices that
it has not run to completion and reinitializes its libset to prepare it for reuse.
Listing~\ref{lst:cleanup} gives pseudocode for a function that the destructor would
call right before this reinitialization to clean up the preemptible function's own
resources.  This works because, for safety, \texttt{resume()} catches all panics
before they can cross the FFI (Foreign Function Interface) boundary
(Section~\ref{sec:libinger:jumps}).  To prevent the
Rust runtime from outputting a diagnostic message when the panic occurs, it is
advisable to first disable the Rust standard library's panic handler in the
preemptible function's libset.  The standard library exposes the
\texttt{panic::set\_hook()} function for doing this.

\begin{figure}
\begin{lstlisting}[label=lst:cleanup,escapeinside={(*}{*)},caption=Resource cleanup for cancelled preemptible functions (pseudocode)]
function cleanup(linger_t func):
	ucontext_t snapshot = func.continuation;

	// Check whether some callee just returned (section (*{\color{Green} \ref{sec:ingerc:return}}*))
	uint64_t retaddr = 8 bytes preceding snapshot.uc_mcontext[REG_RSP]
	if snapshot.uc_mcontext[REG_RIP] == retaddr:
		increment snapshot.uc_mcontext[REG_RIP]

	// If in epilogue, time travel to before (section (*{\color{Green} \ref{sec:ingerc:epilogue}}*))
	if ingerc_epilogue_ra != NULL:
		snapshot.uc_mcontext[REG_RIP] = ingerc_epilogue_ip
		snapshot.uc_mcontext[REG_RSP] = ingerc_epilogue_sp
		location ingerc_epilogue_fp = ingerc_epilogue_ra
		ingerc_epilogue_ra = NULL

	if 16 divides snapshot.uc_mcontext[REG_RSP]:
		// Inject an exception using panic!() (section (*{\color{Green} \ref{sec:ingerc:async}}*))
		snapshot.uc_mcontext[REG_RIP] = panic
	else:
		// Realign the stack and inject exception (section (*{\color{Green} \ref{sec:ingerc:realign}}*))
		snapshot.uc_mcontext[REG_RIP] = realign

	// Throw the exception and let the cleanup landing pads run
	resume(func, UNLIMITED_TIME)
\end{lstlisting}
\end{figure}

Table~\ref{tab:cleanup} summarizes our method of resource cleanup, showing the
actions taken by our proposed runtime at each possible point the preemptible function
(or any of its callees) might be paused when cancellation occurs.  It can be seen
that we have handled all possible points within the body of an ordinary function.
The case we have scoped out of our investigation is cancelling a preemptible function
while it is running a destructor; instead of attempting this, we suggest implementing
a mechanism to detect this case (e.g., unwinding the stack or hooking into the Rust
standard library) and using a return address trick similar to that from
Section~\ref{sec:ingerc:epilogue} to delay resource cleanup until the destructor has
finished.

\begin{table}
\begin{center}
\begin{tabular}{c | p{0.175\textwidth} p{0.15\textwidth} c c}
Position within && Possible to & \multicolumn{2}{c}{Handling} \\
running function & Indicator & unwind here? & Normal execution & When cancelling \\
\hline
First instruction & Stack alignment & No (stack misalignment) & -- & \textsection~\ref{sec:ingerc:start}, \ref{sec:ingerc:realign} \\
Function prologue & Stack alignment & No (stack misalignment) & -- & \textsection~\ref{sec:ingerc:realign} \\
Function body & -- & Yes & -- & Raise exception \\
After call site & Return address & Yes (but leaks) & -- & \textsection~\ref{sec:ingerc:return} \\
Before epilogue & -- & Yes & \textsection~\ref{sec:ingerc:epilogue} & Raise exception \\
Function epilogue & Saved return address & No (calls could clobber stack frame) & -- & \textsection~\ref{sec:ingerc:epilogue} \\
After return & -- & -- & \textsection~\ref{sec:ingerc:epilogue} & Raise exception \\
\end{tabular}
\end{center}
\caption{Cancelled function resource cleanup by position within the running function}
\label{tab:cleanup}
\end{table}


\section{Performance considerations}

While our approach mostly avoids adding operations to the common execution path, the
exception is epilogue handling.  To support that case, we add one function call and
six global variable accesses.  We saw in Chapter~\ref{chap:libgotcha} that in a
normal application, each of these operations has a negligible cost of just a few
cycles.  However, we also found that \textit{libgotcha} slows down accesses to
dynamically-linked global variables considerably.
Fortunately, the names of the symbols we introduce are well known, so
when integrating the new runtime components, we could introduce special cases to
prevent \textit{libgotcha} from intercepting their uses.  This would make the symbols
local to each libset and thereby obviate the need to pay the expensive reference
costs.

\section{Thread library: \textit{libturquoise}}
\label{sec:libturquoise}

\textit{Shinjuku} observes that ``there have been several efforts to implement
efficient, user-space thread libraries.  They all focus on cooperative
scheduling''~\cite{Kaffes:nsdi2019}.  We agree that such libraries are rare, and
attribute this to a lack of natural abstractions to support them.  (Though
\textit{RT} from Section~\ref{sec:related} could be a counterexample, its lack of
nonreentrancy support makes it far from general purpose.)

Although the preemptible function is a fundamentally synchronous abstraction, its
simplicity makes it readily composable, and indeed, well-suited to implementing
preemptive threading.  As a proof of concept, we have created a
preemptively-scheduled userland thread library, \textit{libturquoise}\footnote{so
called because it implements ``green threading with a twist''}, by modifying the
\textit{tokio-threadpool}~\cite{www-tokio-threadpool} work-stealing scheduler from
the Rust futures ecosystem.

To migrate the thread pool workers to preemptive scheduling, we made them poll each
task future from within a preemptible function.  We did this in
just 120 new lines of Rust, 50 of them added to version 0.1.16 of the thread library
and
70 spent augmenting \textit{libinger}'s Rust API with a reusable \textbf{preemptible
futures} adapter.

Currently, \textit{libturquoise} assigns each future it launches or resumes the same
fixed time budget, although this design could be extended to support
multiple job priorities.  When a task times out, the scheduler pops it from its
worker thread's job queue and pushes it to the incoming queue,
offering it to any available worker for rescheduling after all other waiting jobs
have had a turn.


\subsection{Preemptible futures}

For seamless interoperation between preemptible functions and the futures ecosystem,
we built a preemptible future adapter that wraps the \textit{libinger} API.  This
can be used to pass preemptible functions into a platform designed to process
futures.

Because
of languages' differing futures, this integration is not portable like the core API.
Fortunately, its implementation is a straightforward application of
\texttt{pause()} to propagate cooperative yields across the preemptive function
boundary:\@ we present the general construction of the preemptible future
type and an algorithm for polling one in Listing~\ref{lst:future}.

\begin{figure}
\begin{lstlisting}[label=lst:future,caption=Futures adapter type (pseudocode)]
function PreemptibleFuture(Future fut,
                              Num timeout):
  function adapt():
    while poll(fut) == NotReady:
      pause()
  fut.linger = launch(adapt, 0)
  fut.timeout = timeout
  return fut

function poll(PreemptibleFuture fut):
  resume(fut.linger, fut.timeout);
  if has_finished(fut.linger):
    return Ready
  else
    if called_pause(fut.linger):
      notify_unblocked(fut.subscribers)
    return NotReady
\end{lstlisting}
\end{figure}


\renewcommand{\paper}{chapter\xspace}
\chapter{Microsecond-scale microservices}
\label{chap:microservices}

\ifdefined\chapquotes
\begin{chapquote}{Douglas Adams, \textit{The Hitchhiker's Guide to the Galaxy}}
The ships hung in the sky in much the same way that bricks don't.
\end{chapquote}
\fi

This chapter provides a case study of how lightweight preemptible functions could be
used to create the serverless platform of the future.  We have not prototyped any of
its ideas on \textit{libinger}; rather, they formed the initial motivation for
fine-grained preemption and the inspiration for the preemptible function abstraction.

\begin{namespacereferences}{microservices:}
\input[microservices]{abstract}


\input[microservices]{intro}


\input[microservices]{motivation}
\end{namespacereferences}
\hspace{-2em}
Section~\ref{sec:libinger:quantum} demonstrates that interval timers are capable of
delivering signals with this frequency.


\begin{namespacereferences}{microservices:}
\input[microservices]{isolation}
\end{namespacereferences}

Given our performance goals, there is a crucial isolation aspect that Rust does not
provide:\@ there is nothing to stop users from monopolizing the CPU.  Our system,
however, must be preemptive.  We can apply preemptible functions
(Chapter~\ref{chap:functions}) here to impose a time quota on each microservice.
This has the added benefit of automatically providing memory isolation between
library dependencies (Chapter~\ref{chap:libgotcha}), insulating unsafe platform code
from affecting the state of other microservices or the rest of the worker process.

\begin{namespacereferences}{microservices:}
\input[microservices]{isolation_depth}


\input[microservices]{eval}


\input[microservices]{future}
\input[microservices]{future_rest}


\input[microservices]{concl}
\end{namespacereferences}

Since we published this benchmark, Cloudflare has built and deployed a production
FaaS platform called Workers.  Like our proposed architecture, this system omits
containers and virtual machines by running user code in
process~\cite{www-cloudflare-workers}.  It accomplishes this by requiring users to
submit JavaScript code (or WebAssembly) and running each task as a separate Isolate
under the V8 JavaScript engine~\cite{www-javascript-v8}, thereby halving major
competitors' cold-start latencies.  While running under a JavaScript engine confers
some practical benefits such as not having to audit dependencies, we believe that a
shift to native code will be necessary to further reduce cold-start latencies from
the millisecond range to the tens or hundreds of microseconds.

\renewcommand{\paper}{thesis\xspace}


\backmatter

\chapter{Proposed work}

\begin{swallowsections}
\input[functions]{concl}
\end{swallowsections}


\section{Remaining work}

I propose extending the work already completed in all three of the following ways:

\paragraph{Cancellation resource cleanup for the Rust interface}
Although we currently support asynchronous cancellation of timed-out preemptible
functions via \textit{libinger}'s \texttt{cancel()} facility, we do not yet perform
any automatic cleanup of already-allocated resources.  This is impossible
in general for C programs because the language lacks a destructor mechanism; however,
I intend to add at least partial support for doing so for Rust programs.  The basic
principle will be to throw an artificial exception (Rust panic) on the preemptible
function's execution stack, thereby unwinding the stack and invoking local variables'
destructors.  This approach alone will not guarantee comprehensive cleanup, because
such variables may not have been fully initialized at the time of preemption; for the
same reason, it may have safety ramifications that will merit additional study.  I
have some ideas about how to augment the technique, such as by extending
\textit{libgotcha} to keep a running cache of the several most recent allocations and
deallocations of common resources (e.g., memory and file descriptors) from each
libset.  While exhaustive cleanup may prove difficult to achieve, I hope to at least
catalogue situations where we guarantee not to leak certain classes of resource, and
perhaps provide a tunable allowing users to request full resource tracking if needed.

\paragraph{Automatic selection and variation of timer frequency}
For simplicity, the current \textit{libinger} implementation subscribes to timer
signals spaced a globally constant interval apart throughout the entire duration of
each preemptible function.  To improve efficiency while preserving preemption
granularity, I plan to dynamically determine this interval based on the requested
timeout.  For long-running functions, this will include delaying the first of these
signals until shortly before the timeout would expire.  Maintaining accuracy across
multiple CPUs will probably require building in a calibration routine to infer
configuration parameters.

\paragraph{Support OpenSSL and benchmark hyper with HTTPS}
Earlier in \textit{libgotcha}'s development history, it was able to run nginx with
OpenSSL.  Recent attempts at getting hyper to run with OpenSSL have ended in crashes,
yet this configuration is desirable for measurement because it exhibits a far greater
number of dynamic function calls, of which \textit{libgotcha} alters the performance
characteristics.  I aim to support and benchmark the configuration, which will likely
involve regression testing using the old nginx setup.

\hfill \\
\noindent
Additionally, I will complete one of the following projects, as selected by the
committee:

\paragraph{Achieve even finer--grained preemption}
The \textit{Shinjuku} authors included IPI microbenchmarks that suggest it should be
possible for us to achieve interrupt latencies and spacing within a small constant
factor of theirs.  I could attempt to achieve this by reusing some of their
optimizations to POSIX contexts and adding a specialized timer signal delivery
mechanism to the kernel to reduce the number of ISR instructions at the expense of
generality.

\paragraph{Implement ``\textit{libingerOS}'' container framework}
Generalizing the serverless platform case study, it should be possible to generate a
utility that takes two or more position-independent executables and combines them
into a single process whose thread(s) are timeshared between the ``programs'' using
preemptible functions.  Obviously, this would only provide memory isolation for
programs implemented in memory-safe languages.  This would be implemented on top of
\textit{libinger} as a sample application.

\paragraph{Implement a caching RPC framework}
The preemptible functions API naturally lends itself to problems of caching partial
computations, and one interesting case study would be a caching RPC framework.  I
imagine this working as follows:  Clients would enclose alongside each request the
timeout they were using when listening.  The server would process each request inside
its own preemptible function.  When the client timed out waiting for a response, the
server would simultaneously time out and cache the continuation.  Subsequent requests
for the same procedure with identical inputs would resume the interrupted computation
from where it left off.


\section{Timeline}

\begin{itemize}
\item 24 April 2020: ATC '20 author notification
\item 4 June 2020: ATC '20 camera-ready deadline (if applicable)
\item June 2020: Start cancellation resource cleanup, fix OpenSSL support
\item July 2020: Continue cancellation resource cleanup, fix OpenSSL support
\item August 2020: Finish cancellation resource cleanup
\item September 2020: Speaking skills talk, start automatic preemption intervals
\item October 2020: Finish and evaluate automatic preemption intervals
\item November 2020: Start committee-selected project
\item December 2020: Finish and evaluate committee-selected project
\item January--March 2021: Thesis writing, run any final experiments
\item April 2021: Finish thesis
\item May 2021: Defense
\end{itemize}


\cleardoublepage
\addcontentsline{toc}{chapter}{Bibliography}
\bibliography{ref}

\end{document}
